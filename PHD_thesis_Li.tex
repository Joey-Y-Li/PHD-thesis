%%%%%%%%%%%%%%%%%%%%%%%%%%%%%%%%%%%%%%%%%%%%%%%%%%%%%%%%%%%%%%%%%%%%%%%%%%%%%%%%
% University of Western Ontario Thesis Template
% By: Justin Quinn Veenstra, 2010
% With thanks to Mr. (soon to be Dr.) Will Robertson.


\documentclass[12pt,twoside]{report}
%% Decomment next line to use PostScript fonts
%%\UsePackage{times}
%%%%%%%%%%%%%%%%%%%%%%%%%%%%%%%%%%%%%%%%%%%%%%%%%%%%%%%%%%%%%%%%%%%%%%%%
%%                                                                    %%
%%                    ***   I M P O R T A N T   ***                   %%
%%                                                                    %%
%% Fill in the following fields with the required information:        %%
%%  - \department{...}  name of the graduate department               %%
%%  - \degree{...}      name of the degree obtained                   %%
%%  - \author{...}      name of the author                            %%
%%  - \title{...}       title of the thesis                           %%
%%  - \gyear{...}       year of graduation                            %%
%%  - \super{...}    supervisor
%%  - \firstname, \middlename, \lastname... there is additional documentation by the actual fields, so I'll leave it at that
%%%%%%%%%%%%%%%%%%%%%%%%%%%%%%%%%%%%%%%%%%%%%%%%%%%%%%%%%%%%%%%%%%%%%%%%
\usepackage{listings}
\usepackage[table]{xcolor}
\usepackage{booktabs}
\usepackage{xcolor}
\usepackage{setspace}
\usepackage{float}
\usepackage{appendix}
\usepackage{graphicx}
\usepackage{amsmath}
\usepackage[byname]{smartref}
\usepackage[hidelinks]{hyperref}
\usepackage{subcaption}
\usepackage{graphicx}
\usepackage{multirow}
\usepackage{changepage}
\usepackage{pseudocode}
\usepackage[lined,ruled,linesnumbered]{algorithm2e}
\newcounter{algccc}
\newcommand{\ccc}{\addtocounter{algccc}{1}\thealgccc.\ }

\newcommand{\qqa}{\hspace*{10pt}}
\newcommand{\qqb}{\qqa\qqa}
\newcommand{\qqc}{\qqb\qqa}
\newcommand{\qqd}{\qqc\qqa}
\newcommand{\qqe}{\qqd\qqa}
\newcommand{\qqf}{\qqe\qqa}
\newcommand{\qqg}{\qqf\qqa}
\newcommand{\qqh}{\qqg\qqa}
\newcommand{\qqi}{\qqh\qqa}
\newcommand{\qqj}{\qqi\qqa}

\DeclareMathOperator{\Thit}{T_{\rm hit}}
\DeclareMathOperator{\Tsim}{T_{\rm sim}}
\DeclareMathOperator{\Thc}{T_{\rm hc}}
\DeclareMathOperator{\Msim}{M_{\rm sim}}
\def\pp{\mathinner{\ldotp\ldotp}}
%\usepackage{hyperref} %comment out for hardcopy
% \usepackage{txfonts}
\usepackage{tocloft}
\usepackage [english]{babel}
\usepackage [autostyle, english = american]{csquotes}
\MakeOuterQuote{"}
\makeatletter
\numberwithin{figure}{chapter}
\newenvironment{acknowledgements}%
{\clearemptydoublepage
 \begin{center}
  \section*{Acknowledgements}
 \end{center}
 \begingroup
}{\newpage\endgroup}

\newenvironment{dedication}%
{\clearemptydoublepage 
 \begin{center}
  \section*{Dedication}
 \end{center}
 \begingroup
}{\newpage\endgroup}

\newenvironment{preliminary}%
{\pagestyle{plain}\pagenumbering{roman}}%
{\pagenumbering{arabic}}

\addtoreflist{chapter}
\newtheorem{theorem}{Theorem}[section]
\newtheorem{lemma}[theorem]{Lemma}
\newtheorem{proposition}[theorem]{Proposition}
\newtheorem{corollary}[theorem]{Corollary}

\newenvironment{proof}[1][Proof]{\begin{trivlist}
\item[\hskip \labelsep {\bfseries #1}]}{\end{trivlist}}
\newenvironment{definition}[1][Definition]{\begin{trivlist}
\item[\hskip \labelsep {\bfseries #1}]}{\end{trivlist}}
\newenvironment{example}[1][Example]{\begin{trivlist}
\item[\hskip \labelsep {\bfseries #1}]}{\end{trivlist}}
\newenvironment{remark}[1][Remark]{\begin{trivlist}
\item[\hskip \labelsep {\bfseries #1}]}{\end{trivlist}}

\newcommand{\qed}{\nobreak \ifvmode \relax \else
      \ifdim\lastskip<1.5em \hskip-\lastskip
      \hskip1.5em plus0em minus0.5em \fi \nobreak
      \vrule height0.75em width0.5em depth0.25em\fi}

% Default values for title page.

%% To produce output with the desired line spacing, the argument of
%% \spacing should be multiplied by 5/6 = 0.8333, so that 1 1/2 spaced
%% corresponds to \spacing{1.5} and double spaced is \spacing{1.66}.
\def\normalspacing{1.5} % default line spacing


%% Define the "thesis" page style.
\if@twoside % If two-sided printing.
\def\ps@thesis{\let\@mkboth\markboth
   \def\@oddfoot{}
   \let\@evenfoot\@oddfoot
   \def\@oddhead{
      {\sc\rightmark} \hfil \rm\thepage
      }
   \def\@evenhead{
      \rm\thepage \hfil {\sc\leftmark}
      }
   \def\chaptermark##1{\markboth{\ifnum \c@secnumdepth >\m@ne
      Chapter\ \thechapter. \ \fi ##1}{}}
   \def\sectionmark##1{\markright{\ifnum \c@secnumdepth >\z@
      \thesection. \ \fi ##1}}}
\else % If one-sided printing.
\def\ps@thesis{\let\@mkboth\markboth
   \def\@oddfoot{}
   \def\@oddhead{
      {\sc\rightmark} \hfil \rm\thepage
      }
   \def\chaptermark##1{\markright{\ifnum \c@secnumdepth >\m@ne
      Chapter\ \thechapter. \ \fi ##1}}}
\fi

\pagestyle{thesis}
% Set up page layout.
\setlength{\textheight}{9in} % Height of the main body of the text
\setlength{\topmargin}{-.5in} % .5" margin on top of page
\setlength{\headsep}{.5in}  % space between header and top of body
\addtolength{\headsep}{-\headheight} % See The LaTeX Companion, p 85
\setlength{\footskip}{.5in}  % space between footer and bottom of body
\setlength{\textwidth}{6.25in} % width of the body of the text
\setlength{\oddsidemargin}{.25in} % 1.25" margin on the left for odd pages
\setlength{\evensidemargin}{0in} % 1.25"  margin on the right for even pages

% Marginal notes
\setlength{\marginparwidth}{.75in} % width of marginal notes
\setlength{\marginparsep}{.125in} % space between marginal notes and text

% Make each page fill up the entire page. comment this out if you
% prefer. 
\flushbottom

\setcounter{tocdepth}{3} % Number the subsubsections 
% \def\normalspacing{1.5} % default line spacing

\newcommand\isco[1]{%
  \edef\@tempa{#1}%
  \def\@tempb{}%
  \ifx\@tempa\@tempb
	\else \\\underline{Co-Supervisor:}\vspace{0.35in}\\\dots\dots\dots\dots\dots\dots\dots\\{#1}\\
  \fi
}

\newcommand\isjoint[1]{%
  \edef\@tempa{#1}%
  \def\@tempb{}%
  \ifx\@tempa\@tempb
	\else \\\underline{Joint Supervisor:}\vspace{0.35in}\\\dots\dots\dots\dots\dots\dots\dots\\{#1}\\
  \fi
}

\newcommand\isalt[1]{%
  \edef\@tempa{#1}%
  \def\@tempb{}%
  \ifx\@tempa\@tempb
	\else \\\underline{Alternate Supervisor:}\vspace{0.35in}\\\dots\dots\dots\dots\dots\dots\dots\\{#1}\\
  \fi
}

\newcommand\isdefinedsig[1]{%
  \edef\@tempa{#1}%
  \def\@tempb{}%
  \ifx\@tempa\@tempb
	\else \\ \dots\dots\dots\dots\dots\dots\dots\\{#1}\\
  \fi
}
\newcommand\isdefinedspinetitle[1]{%
  \edef\@tempa{#1}%
  \def\@tempb{}%
  \ifx\@tempa\@tempb
	\else (Spine title: #1)\\
  \fi
}
\newcommand\coauthor[1]{%
  \edef\@tempa{#1}%
  \def\@tempb{}%
  \ifx\@tempa\@tempb
	\else \newpage \Large Co-Authorship Statement\normalsize\\\indent\\#1\\
  \fi
}

\newcommand\acknowlege[1]{%
  \edef\@tempa{#1}%
  \def\@tempb{}%
  \ifx\@tempa\@tempb
	\else \newpage \Large Acknowledgements\normalsize\\\indent\\#1\newpage
  \fi
}

%\renewcommand{\appendixtocname}{\Huge \textbf{List of Appendices} \normalsize}
\newcommand{\blank}{\hspace{-2mm}}
\newcommand{\super}{Dr. Lucian Ilie} %supervisor
\newcommand{\superj}{} %joint supervisor, if there is one, leave blank if not (lbin)... only one of the three.
\newcommand{\superc}{} %co-supervisor, if there is one, leave blank if not (lbin)
\newcommand{\supera}{} %alternate supervisor, if there is one, leave blank if not (lbin)
\newcommand{\sco}{Dr. W. J. Braun}  %member of supervisory committee
\newcommand{\sct}{Dr. A. Bing}  %other member of supervisory committee (lbin)
\newcommand{\examo}{Dr. Q. Ring}  %examining committee (up to four, if less leave blank)
\newcommand{\examt}{Dr. W. Fing}
\newcommand{\examth}{Dr. G. Hing}
\newcommand{\examf}{}
\newcommand{\department}{Computer Science}
\newcommand{\degree}{Doctor of Philosophy}
\newcommand{\firstname}{Yiwei}
\newcommand{\middlename}{}
\newcommand{\lastname}{Li}
%\renewcommand{\author}[1]{\ifx\empty#1\else\gdef\@author{#1}\fi} 
\newcommand{\authorname}{{\firstname} {\middlename} {\lastname}}
\newcommand{\titl}{Computational Methods for Predicting Protein-protein Interactions and Binding Sties}
\newcommand{\spinetitle}{}%only if the above is more than 60 characters
\newcommand{\thesisformat}{Monograph} %or Integrated Article
\newcommand{\gyear}{\number\year}
\newcommand{\makecoauthor}{
%Type information about coauthorship here/
I would like to acknowledge my imaginary friend, Jummi for doing all the work. 
}
\newcommand{\makeacknowlege} {
TODO ...
}
% \newcommand{\listappendixname}{List of Appendices}
% \newlistof{myappendices}{app}{\listappendixname}
% \newcommand{\myappendices}[1]{%
% \addcontentsline{app}{myappendices}{#1}\par}

\renewcommand{\maketitle}
{\begin{titlepage}
   \setcounter{page}{1}
   %% Set the line spacing to 1 for the title page.
   %\begin{spacing}{1} 
   \begin{large}
   \begin{center}
      \mbox{}
      \vfill
      {\MakeUppercase{\titl}}\\
      \isdefinedspinetitle{\spinetitle}
      (Thesis format: \thesisformat)\\
      \vfill
      by \\
      \vfill
      {\firstname} \underline{\lastname}\\
      \vfill
      Graduate Program in {\department}\\
      \vfill
		A thesis submitted in partial fulfillment\\
		of the requirements for the degree of\\
		\degree\\
		\vfill
		The School of Graduate and Postdoctoral Studies\\
		The University of Western Ontario\\
		London, Ontario, Canada\\
		\vfill
      {\copyright} {\authorname} {\gyear}  \\
      \vspace*{.2in}
   \end{center}
   \end{large}
%   \end{spacing}
   \end{titlepage}

}%\maketitle

\newcommand{\makecert}{
   \setcounter{page}{2}
\vfill
\begin{center}
\large
THE UNIVERSITY OF WESTERN ONTARIO\\
School of Graduate and Postdoctoral Studies\\
\vfill
\textbf{CERTIFICATE OF EXAMINATION}
\end{center}

\vfill
\begin{table}[ht]
\begin{minipage}[t]{0.5\linewidth} %tabular instead?
\begin{tabular}{l}
\underline{Supervisor:}\vspace{0.35in}
\isdefinedsig{\super}
\isco{\superc}
\isjoint{\superj}
\isalt{\supera}
\\
\underline{Supervisory Committee:}\vspace{0.35in}
\isdefinedsig{\sco}\vspace{0.15in}
\isdefinedsig{\sct}
\end{tabular}
\vfill
\end{minipage}
\hspace{0.5in}
\begin{minipage}[t]{0.5\linewidth}
\begin{tabular}{l}
\underline{Examiners:} \\\vspace{.5cm}
\isdefinedsig{\examo}\\
\isdefinedsig{\examt}\\
\isdefinedsig{\examth}\\
\isdefinedsig{\examf}
\end{tabular}
\vfill
\end{minipage}
\vfill
\end{table}
\vfill
\begin{center}
The thesis by \\ \vfill
\textbf{\firstname{} \middlename{} \underline{\lastname}}\\
\vfill
entitled:\\\vfill
\textbf{\titl}\\\vfill
is accepted in partial fulfillment of the \\
requirements for the degree of\\
\degree\\
\end{center}
\begin{table}[ht]
\begin{minipage}[t]{0.5\linewidth}
\begin{tabular}{l}
\dots\dots\dots\dots\dots\\
Date
\end{tabular}
\end{minipage}
\hspace{0.5in}
\begin{minipage}[t]{0.5\linewidth}
\begin{tabular}{l}
\dots\dots\dots\dots\dots\dots\dots\dots\dots\dots\\
Chair of the Thesis Examination Board
\end{tabular}
\end{minipage}
\end{table}

}

\makeatother
\begin{document}
\begin{spacing}{\normalspacing}

%% ***   NOTE   ***
%% You should put all of your '\newcommand', '\newenvironment', and
%% '\newtheorem's (in other words, all the global definitions that you
%% will need throughout your thesis) in a separate file and use
%% "\input{filename}" to input it here.


%% This sets the page style and numbering for preliminary sections.
\begin{preliminary}

%% This generates the title page from the information given above.
\maketitle
\addcontentsline{toc}{chapter}{Certificate of Examination}
\makecert
\newpage
%\addcontentsline{toc}{chapter}{Co-Authorship Statement}
%\coauthor{\makecoauthor}  %comment this out if none
%\newpage
\addcontentsline{toc}{chapter}{Acknowlegements}
\acknowlege{\makeacknowlege}	%as above
\addcontentsline{toc}{chapter}{Abstract}
\Large\begin{center}\textbf{Abstract}\end{center}\normalsize
%%  ***  Put your Abstract here.   ***
%% (150 words for M.Sc. and 350 words for Ph.D.)

Proteins are essential to organisms and participate in virtually every  process within cells. Quite often, they keep the cells functioning by interacting with other proteins. This process is called protein-protein interaction (PPI). The bonding amino acid residues during the process of protein-protein interactions are called PPI binding sites. Identifying PPIs and PPI binding sites are fundamental problems in system biology.

Experimental methods for solving these two problems are slow and expensive. Therefore, great efforts are being made towards increasing the performance of computational methods.

We present DELPHI \cite{li2020delphi}, a deep learning based program for PPI site prediction and SPRINT \cite{li2017sprint, li2020predicting}, a algorithmic based program for PPI prediction. Both programs have been compared to the state-of-the-art programs on several datasets. Both DELPHI and SPRINT are more accurate than the competing method. SPRINT is also orders of magnitudes faster while using very little memory.

The dataset and source code for both DELPHI and SPRINT are publicly available at:
\text{github.com/lucian-ilie} and
and \text{www.csd.uwo.ca/\~{}ilie/software.html}

\vfill
\textbf{Keywords:} Bioinformatics, SPRINT, DELPHI, Protein-protein interaction, deep learning, Protein-protein interaction prediction, Protein-protein interaction binding sites prediction
\newpage
\tableofcontents\newpage
\newpage
\addcontentsline{toc}{chapter}{List of Figures}
\listoffigures
\newpage
\addcontentsline{toc}{chapter}{List of Tables}
\listoftables\newpage
% \addcontentsline{toc}{chapter}{List of Appendices}
% \listofmyappendices\newpage
%\addcontentsline{toc}{chapter}{List of Abbreviations, Symbols, and Nomenclature}
%\large List of Abbreviations, Symbols, and Nomenclature \normalsize
%\newpage
\end{preliminary}
%% End of the preliminary sections: reset page style and numbering.

%%%%%%%%%%%%%%%%%%%%%%%%%%%%%%%%%%%%%%%%%%%%%%%%%%%%%%%%%%%%%%%%%%%%%%%%
%%                                                                    %%
%%                    ***   I M P O R T A N T   ***                   %%
%%                                                                    %%
%% Put your Chapters here; the easiest way to do this is to keep each %%
%% chapter in a separate file and \include all the files right here.  %%
%% Note that each chapter file should start with the line             %%
%% "\chapter{ChapterName}".  Note that using "\include" instead of    %%
%% "\input" makes each chapter start on a new page.                   %%
%%%%%%%%%%%%%%%%%%%%%%%%%%%%%%%%%%%%%%%%%%%%%%%%%%%%%%%%%%%%%%%%%%%%%%%%

\chapter{Introduction \label{chap_1}}

\section{DNA}
DNA is the code of life. Almost all living organisms (exception: some viruses) are coded by four nucleotides: adenine (A), thymine (T), guanine (G), and cytosine (C). DNA has a double helix structure which is composed of sugar molecules, phosphate group, and bases (A, G, C, T). In DNA strands, A is always matched with T and G is always matched with C (\cite{jones2004introduction}).
DNA can copy itself through the replication process. It can also transcript into RNA. During transcription, the information in DNA pairs is passed 
to corresponding RNA, which will use uracil (U) instead of thymine (T) to match adenine. After transcription, RNA will be translated into amino acid chains then further form proteins. In the process of transcription, every three nucleotides (codon) will determine one kind of amino acid. Proteins are strings of amino acids. Figure \ref{fig_trans_trans} shows the transcription and translation process in a cell. 
\begin{figure}[!h]
\begin{center}
\includegraphics[height = 9cm, width = 9cm]{img/transcp_transla.jpg}
\caption[Illustration of transcription and Translation]{Illustration of transcription and Translation. DNA is transcripted to RNA. RNA is then translated to amino acid chains. From: tokresource.org\label{fig_trans_trans}}
\end{center}
\end{figure}

\section{Protein}
One or more amino acid chain forms a protein. Proteins are large biomolecules, or macromolecules. In general, proteins are made of twenty kinds of amino acids. The function of proteins varies including working as antibodies, contractile proteins, enzymes, hormonal proteins, structural, storage proteins, transport proteins, etc. (\cite{white1959principles}). Proteins are essential to organisms and participate in virtually every  process within cells. It is estimated that human body has 1 to 3 billion proteins \cite{milo2013total}.

As shown in Figure \ref{fig_pro_se}, the primary structure of a protein is its amino acid sequence. This information is the most available data we can obtain through publicly available databases such as Uniprot (\url{https://www.uniprot.org/}) \cite{uniprot2014uniprot}. For example, manually annotated all human proteins and their sequences can be downloaded by clicking on``Swiss-Prot" and then ``Human" and then ``Download". Uniprot contains additional information such as function and gene ontology information which can be filtered by editing ``Columns" before downloading. The downloaded dataset  can be further processed based on specific needs.
\begin{figure}[!ht]
\begin{center}
\includegraphics[ width = 8cm]{img/pro_amin.jpg}
\caption[The primary structure of a protein]{The primary structure of a protein. Each small circle indicates an amino acid, and the primary structure of a protein is the amino acid sequence. From: wikimedia.org \label{fig_pro_se}}
\end{center}
\end{figure}

As shown in Figure \ref{fig_pro_1234}, proteins also have second, tertiary, and quaternary structures. Protein secondary structure represents local structures stabilized by hydrogen bonds within a protein chain itself. The most common local structures are alpha helix and beta pleated sheet. The tertiary structure contains the overall shape of a single protein molecule, in other words, the spatial relationship of the secondary structure of multiple chains. The term "tertiary structure" is often used as interchangeably with the term "fold". The tertiary structure controls the basic function of a protein. The quaternary structure describes the structure among several protein molecules, which is a protein complex. 
\begin{figure}[!h]
\begin{center}
\includegraphics[ width = 12cm]{img/protein_1234_structure.png}
\caption[The primary, secondary, tertiary, and quaternary structure of proteins]{The primary, secondary, tertiary, and quaternary structure of proteins. From: https://www.thoughtco.com/ \label{fig_pro_1234}}
\end{center}
\end{figure}

As mentioned above, the most widely available data is the primary structure data of organisms. FASTA format is used to represent such data. The FASTA file uses single-letter codes to represent peptide sequences of each protein. Each protein has two lines. The first line is the protein name, and the second line is its amino acid sequence. The line containing protein names starts with a '\textgreater' sign and is followed by the protein name. In the sequence line, each letter encodes an amino acid. An example is given in Fig \ref{fig:fasta}, the amino acid sequence of protein P32479 is MKVVKFPWLAHREESRKYEIYTVDVSHDGKRLA. 

\begin{figure}
\begin{center}
\includegraphics[width=0.8\textwidth]{img/fasta.png}
\caption[A FASTA file example]{A FASTA file example. Each protein consists of the protein name which starts with `\textgreater' and the amino acid sequence.}
\label{fig:fasta}  
\end{center}
\end{figure}

\section{Protein-protein Interaction Prediction \label{PPI_site_intro}}
Proteins are the most important molecules in cells (\cite{schleif1993genetics}). 
They carry out most of the cellular processes. Quite often, they keep the cells functioning by interacting with other proteins in stable or transient protein complexes (\cite{eisenberg2000protein}).
This process is called protein-protein interaction (PPI). This is a vital process because of the accepted idea that PPIs are responsible for the behaviour of cells under different stimuli (\cite{bader2003functional, pandey2000proteomics, schwikowski2000network}).
Protein complexes are groups of proteins that interact together to perform certain functions. Figure \ref{fig_pro_comp} shows an example of a protein complex. Protein pathways and modules are another two functional groups connected through PPIs. 
\begin{figure}[h!]
\begin{center}
\includegraphics[height = 8cm]{img/pro_comp.png}
\caption[Protein complex]{Protein complex. (From: bioproximity.com)  \label{fig_pro_comp}}
\end{center}
\end{figure}
 Scientists believe the reason that advanced organisms like humans are more complicated than lower organisms like the worms is not only because of large number of genes, but also because of sophisticated PPI networks \cite{pitre2008computational}. Figure \ref{fig_ppi_net} shows a illustration of human PPIs.
Understanding the potential of unknown proteins is becoming possible by looking into their PPI information \cite{sharan2007network}. As well, ientifying PPI information helps improve the system-level understanding of molecular processes \cite{levy2008evolution}.
Therefore, understanding and mapping PPIs is an important current area of research.

\begin{figure}[h!]
\begin{center}
\includegraphics[height = 10cm, width = 10cm]{img/pro_inter_net.jpg}
\caption[A human protein-protein interaction network]{A human protein-protein interaction network. (From: \href{https://www.mdc-berlin.de/}{www.mdc-berlin.de})  \label{fig_ppi_net}}
\end{center}
\end{figure}

However, there are a lot of unknown facts about PPIs to be discovered. For example, in a simple organism such as Saccharomyces Cerevisiae, there are less than 40,000 estimated PPIs. Comparing with 19,000,000 potential interacting pair, it is believed that there is still a significant gap between discovered PPIs and all PPIs of existence \cite{jessulat2011recent, sprinzak2003reliable}. Precise, fast and affordable protein-protein interaction prediction methods are needed.

The idea of PPI prediction is shown in Figure \ref{fig_ppi_pred}. Proteins and interactions are indicated using nodes and edges respectively. Green edges are known interactions. Red edges are unknown interactions. The task of predicting PPIs is to add the red edges. 
\begin{figure}[h!]
\begin{center}
\includegraphics[width=12cm]{img/network.png}
\end{center}
\caption[The idea of protein-protein interaction prediction]{The idea of protein-protein interaction prediction. The red dots indicate proteins and edges mean interactions. Green edges are discovered interactions stored in PPI databases. The idea of predicting PPIs is to add the red edges, which indicate interactions that exist but are missing in the database. \label{fig_ppi_pred}}
\end{figure}

In additional to the protein primary sequences, stored in FASTA file (see Figure \ref{fig:fasta}), protein-protein interaction file is also needed for PPI prediction (see Figure \ref{fig_ppi_input_output} left). PPI files contain known interactions from databases that help PPI predictors learn to classify new interactions, together with the sequence information. Most machine learning based programs also require as an additional input, the negative interactions, that is protein pairs that are known not to interact. The output of PPI predictors are scores for each new protein pair (see Figure \ref{fig_ppi_input_output} right). The higher the score, the more confident a predictor claims a pair of proteins interact.
\begin{figure}[h!]
\begin{center}
\includegraphics[width=7cm]{img/ppi_file.png}
\includegraphics[width=7cm]{img/result_interactome.png}
\end{center}
\caption[Examples of an input PPI file and output PPI score file]{Examples of an input PPI file (left) and output PPI score file (right). \label{fig_ppi_input_output}}
\end{figure}

\section{Protein-protein Interaction Binding Sites Prediction}
Proteins interact with each other by binding together. Figure \ref{fig_ppi_bind} shows two binding proteins. The bonding amino acid residues are protein-protein interaction binding sites. Detecting PPI binding sites helps understand cell regulatory mechanisms, locating drug targets, predicting protein functions \cite{bonetta2010interactome}. There are noticeable industrial efforts putting into this area as well. For example, RemediumAI (remediumai.com) is developing machine learning predictors to accelerate the drug design process. Databases like PDB \cite{berman2002protein} store protein binding sites information deriving from the 3D structure of each protein, but the available protein structures are limited. Predicting the binding residues in each protein have been attempted by many researcher. However, the prediction performance is still far from satisfaction. 
\begin{figure}[h!]
\begin{center}
\includegraphics[width=12cm]{img/binding_sites.png}
\end{center}
\caption[Illustration of protein-protein interaction binding sites]{Protein-protein interaction binding sites. Two interaction proteins (red and blue) forms a binding structure (rightmost). The contacting residues are binding sites. (From: \cite{townshend2018generalizable}) \label{fig_ppi_bind}}
\end{figure}

Similar to the PPI prediction problem, in the realm of PPI binding sites prediction, experimental approaches, for instance, mutagenesis, are labor and time consuming, and thus, many computational PPI sites prediction programs have been developed as they are faster and cheaper. Again, there are mainly two categories: sequence based and structure based methods \cite{esmaielbeiki2016progress}. For the same reason disccussed in Section \ref{PPI_site_intro}, we focus on sequence based methods.

The illustration of PPI biding site prediction is shown in Figure \ref{fig_ppi_bind_work_flow}. Programs are trained using the primary sequence and discovered binding sites and then predict binding residues on new proteins.
\begin{figure}[h!]
\begin{center}
\includegraphics[width = 15cm]{img/PPI_site_pred_input.png}
\end{center}
\caption[Illustration of protein-protein interaction binding sites prediction]{Illustration of the input and output in the protein-protein interaction binding sites prediction task. On the left, the primary sequence and the biding site information are used as inputs. They are stored in a FASTA-like format where an additional line is added for each protein. 1 indicates binding sites and 0 means non-binding sites for the corresponding location. The binding sites for new proteins are predicted (red circles on the right). The figures of amino acid sequence and the binding proteins are from customequinenutrition.com. \label{fig_ppi_bind_work_flow}}
\end{figure}

\section{Thesis Overview}
The thesis is organized as follows. Chapter \ref{chap_1} describes the DNA, protein, and defines the problems. Chapter \ref{chap_2} describes in details our PPI sites prediction program, DELPHI \cite{li2020delphi, li2020delphi_ISMB}. Chapter \ref{chap_3} describes in details our PPI prediction program, SPRINT \cite{li2017sprint, li2020predicting}. Chapter \ref{chap_4} concludes our research contribution, summarizes some common methodologies in working in bioinformatics, proposes some future work.

\chapter{DELPHI \label{chap_2}}
This chapter introduces our recent publication DELPHI \cite{li2020delphi, li2020delphi_ISMB}, a deep learning based program for protein-protein interaction site prediction. We first describe the deep learning prerequisites used in DELPHI as well as the state-of-the-art methods. Then we describe both the algorithm and the implementation of DELPHI in details. We comprehensively compare DELPHI to nine state-of-the-art programs on five datasets, and DELPHI outperforms the competing methods in all metrics even though its training dataset shares the least similarities with the testing datasets. In the most important metrics, AUPRC and MCC, it surpasses the second best programs by as much as 18.5\% and 27.7\%, resp. We also demonstrated that the improvement is essentially due to using the ensemble model and, especially, the three new features. Using DELPHI it is shown that there is a strong correlation with protein-binding residues (PBRs) and sites with strong evolutionary conservation.  In addition DELPHI's predicted PBR sites closely match known data from Pfam.
DELPHI is available as open sourced standalone software and web server.
\section{Background}
\subsection{Deep Learning in Bioinformatics} \label{set_AI_in_bioinfor}
Deep Learning is a branch of machine learning \cite{goodfellow2016deep}. It is inspired by the structure and function of the brain called artificial neural networks. From a high level, any computational programs can be considered mathematical transformation on given inputs. Traditional programs handcraft these mathematical computations while machine learning algorithms learn them. The mathematical transformation is learned from known examples, generally without being programmed with specific rules.

For example, in the Handwritten Digit Recognition problem, researchers used to design specific rules to to capture the characteristics of each number. While in deep learning, only the structure of a network is built. The network "learns" the ability of telling a handwritten number by seeing a large amount of labeled data as shown in Fig. \ref{fig_dig_rec}.
\begin{figure}[!h]
\begin{center}
\includegraphics[width = 9cm]{img/digit_recog.png}
\caption[The labeled data in the Handwritten Digit Recognition problem]{The labeled data in the Handwritten Digit Recognition problem. Each handwritten number (inside the black boxes) is labeled with its true value (above each handwritten number). From medium.com \label{fig_dig_rec}}
\end{center}
\end{figure}
%We illustrate this using a simple house price predictor as an example shown in Fig. \ref{fig_house_price}. Assume size, number of bedrooms, zip code, and wealth are the four factors to determine the house price. Traditional method need to hard code the mathematical relationship between the four inputs and the final predicting price y. For example, a linear function with a weight $\theta$ to each factor $x_i$. Written in vector notation is $y=\theta^Tx$. $\theta$ in this case is handcrafted. In contrast, deep learning methods define the computation flow (edges) between each value (nodes). The actual computation is learned by seeing sample data, for example 100 known house price and their four attributes. 
% \begin{figure}[!h]
% \begin{center}
% \includegraphics[width = 9cm]{img/house_price.png}
% \caption{Example of predicting house price using computational methods. The x1 - x4 indicate the four factors in determine the house price. Circles indicate computations and edges indicate data flows in the computations. \label{fig_house_price}}
% \end{center}
% \end{figure}

Many bioinformatics problems have been transformed into well defined mathematical problems. For example, sequence alignment, tree comparison, similarity search etc.These problems often have good algorithmic solutions. However, there are also large amount of bioinformatics problems are far from understanding, and many of them are fundamental problems such as coding region identification, structure prediction, interaction identification, biomedical image classification \cite{larranaga2006machine}. The biggest obstacle of solving these problems are the underlying mechanism is biologically complicated so that even biologists can not give definitive explanation of the factors and how they are related. 

With the increasing amount of biological data, more and more hard biological problems have been attempted to solve using deep learning methods. As shown in Fig. \ref{fig_deelLearning_papers}, since early 2010s, the number of deep learning bioinformatics methods has increased drastically. This is due to several reasons. First, significant amounts of biomedical data have been accumulated which enables significant performance improvement deep learning methods. For example, as shown in Fig. \ref{fig_pdb_stats}, from year 2000 to 2020 April, The total number of PDB \cite{berman2002protein} released structures increased from 13,589 to 162,529, which is almost 12 times bigger as two decades ago. As shown in Fig. \ref{fig_data_performance}, the performance of deep learning methods generally increases with more data while traditional methods reach its maximum performance after being fed with certain amount of data. Second, great amount of efforts have been made in deep learning software hardware co-design to accelerate the deep learning process especially for accelerating training. On the software side, numerous deep learning frameworks and compilers have been released and well maintained. Many of them of opensource programs with tight community integration. Popular frameworks include Google's Tensorflow \cite{tensorflow2015-whitepaper}, PyTorch \cite{NEURIPS2019_9015}, Caffe \cite{jia2014caffe}, high level API libraries like Keras \cite{chollet2015keras}, and recently opensourced MindSpore \cite{liao2019davinci} from Huawei. Compiler optimizations have also been explored greatly. For instance, Google's MLIR \cite{lattner2020mlir} project aims to unify the deep learning intermediate representation so that frontend APIs and the hardware can integrate more smoothly. The TVM \cite{chen2018tvm} project aims to automate the kernel implementation on various hardware.  These software makes deep learning easy to use and efficient during runtime. On the hardware side, general purpose CPU was initially used for deep learning computation. Soon GPU became very popular because of its vector computation unit that greatly accelerates the vector computations in deep learning. Recent years, specialized hardware focusing on the cube unit have also been developed by major companies. Cube unit performs matrix multiply in a single cycle which further accelerates deep learning computations. Well known examples include Google's Tensor Processing Unit (TPU), chips produced by Habana Lab (acuired by Intel), and Huawei's Ascend chips etc. As shown in Fig. \ref{fig_hardware_speedup}, the full setup of TPU could potentially speedup 200 times of the training on Resnet50. Third, biological problems are often too complicated for hand-crafted algorithms. Deep learning on the other hand can takes advantage of the growing number of data and captures the underlying patterns buried in high dimentionla data that human cannot.  Fourth, the influence of events and completions has attracted many researchers into the field of AI. To name a few, in ImageNet Large Scale Visual Recognition Challenge in 2012, AlexNet \cite{krizhevsky2012imagenet} ranked the first place and surpassed the second place by a huge 10.8\% margin. In October 2015, AlphaGo \cite{silver2017mastering} beat the professional Go player Lee Sedol on a full-sized board. Deepmind’s AlphaFold \cite{senior2020improved} won CASP13 protein-folding competition in 2018. 
\begin{figure}[h!]
\begin{center}
\includegraphics[width = 9cm]{img/deepLearning_in_bioinfo.png}
\caption[Approximate number of published deep learning and deep learning bioinformatics papers by year]{Approximate number of published deep learning and deep learning bioinformatics papers by year. From \cite{min2017deep}.  \label{fig_deelLearning_papers}}
\end{center}
\end{figure}

\begin{figure}[!h]
\begin{center}
\includegraphics[height = 6cm, width = 9cm]{img/pdb_stats.png}
\caption[PDB Statistics: Overall Growth of Released Structures Per Year]{PDB Statistics: Overall Growth of Released Structures Per Year. From rcsb.org \cite{berman2002protein} \label{fig_pdb_stats}}
\end{center}
\end{figure}

\begin{figure}[!h]
\begin{center}
\includegraphics[height = 7cm, width = 9cm]{img/data_deeplearning.png}
\caption[Learning performance with the increment of data]{Learning performance with the increment of data. From deepai.org/  \label{fig_data_performance}}
\end{center}
\end{figure}

\begin{figure}[h!]
\begin{center}
\includegraphics[height = 8cm, width = 13cm]{img/hardware_speedup.png}
\caption[Google's TPU speedup on Resnet50 training]{Google's TPU speedup on Resnet50 training. From cloud.google.com/tpu \label{fig_hardware_speedup}}
\end{center}
\end{figure}

In the area of PPI prediction and PPI binding site prediction, most of methods published within a past decade are machine learning based. Seeing the trend, we designed and implemented our program DELPHI using deep learning techniques as well.

\subsection{Basic Notions and Definitions}
\subsubsection{Deep Neural Networks}
Neural networks learn complex mathematical transformations using layers of neurons. Fig. \ref{fig_mlp} shows the classic structure of a multilayer perceptron (MLP). It consists of at least three layers. The input layer, the output layer, and the hidden layers. The input layer is the initial state of the data. In bioinformatics, often raw data such as DNA or protein sequences is converted to features representations like embedding or position-specific scoring matrix (PSSM). Sequence itself contains useful but often limited information and needs to be represented further using features. The input layer is then passed to hidden layers for further computation. The term “deep learning” comes from the fact that the number of the hidden layers is big.
\begin{figure}[h!]
\begin{center}
\includegraphics[width = 13cm]{img/multiplayer_perceptron.png}
\caption[Multilayer perceptron (MLP)]{Multilayer perceptron (MLP). An MLP must have at least three layers: the input layer, a hidden layer and the output layer. From medium.com \label{fig_mlp}}
\end{center}
\end{figure}

The term neuron, or often referred to as perceptron, means a mathematical function. As shown in Fig. \ref{fig_neuron}, a neuron takes in one or more inputs and apply mathematical functions on them and obtain an output called hypothesis. Each input is multiplied by a weight with the intuition of weighing the importance of each input.
% one or more inputs are multiplied by values called weights and then combined together with a bias value to better fit the model. Simple models often Linear Regression and Logistic Regression. Linear regression is a linear function while logistic regression is non linear.
\begin{figure}[h!]
\begin{center}
\includegraphics[height = 6cm, width = 7cm]{img/neuron.png}
\caption[A neuron in deep learning networks]{A neuron in deep learning networks. $x_1$ - $x_3$ are inputs. $b$ is the bias value. $w_1$ - $w_3$ are the weights. $Z$ is the value before applying activation functions. $a$ is the final output value.From /towardsdatascience.com. \label{fig_neuron}}
\end{center}
\end{figure}
% For inputs $x1, x2 ... xi$, the vector notation for Linear Regression and Logistic Regression are \ref{equ_linear_regression} and \ref{equ_logistic_regression} respectively where $h_\theta(x)$ are the output, also known as hypothesis, $\theta^T$ is the weights and b is the bias. 
% \begin{equation}
% h_\theta(x)=\theta^Tx + b \label{equ_linear_regression}
% \end{equation}

% \begin{equation}
% h_\theta(x)=\frac{1}{1+e^{-\theta^Tx}} \label{equ_logistic_regression}
% \end{equation}
After calculating the hypothesis, an non-linear function is often applied to it, and this is called an activation function. The intuition of the the term "activation" is that each neuron calculates the hypothesis of inputs and if the result is greater than a threshold, then the neuron activates and send a signal to the next neuron. Popular activation function include Sigmoid, tanh, ReLU, LeakyReLU, Maxout, ELU \cite{goodfellow2016deep}. Their function and plots are shown in Fig. \ref{fig_activation_func}.
\begin{figure}[h!]
\begin{center}
\includegraphics[width = 13cm]{img/activation_function.png}
\caption[Popular activation functions and their plots]{Popular activation functions and their plots. From /towardsdatascience.com. \label{fig_activation_func}}
\end{center}
\end{figure}

\subsubsection{Training}
The output of a network may not be able to predict the true vale of a sample. The weights in network need to be tuned to achieve better prediction. This process is called training a neuron network.

First, we need to measure how well the current network is doing by having calculating the the difference, or cost, between the current predicted value and the true value of the output. This is done by having a loss function. A commonly used loss function for is least square which is shown in Equation \ref{equ_least_square} in vector notation, where $J(\theta)$ is the cost, $h_\theta(x^{(j)})$ is the predicted value or the hypothesis, $y^{(j)}$ is the ground truth, and $m$ is the number of samples. Other popular loss functions include mean squared error, cross-entropy, hellinger distance, logcosh etc \cite{goodfellow2016deep}.
\begin{equation}
J(\theta)=\frac{1}{2}\sum^m_{j=1}(h_\theta(x^{(j)}) - y^{(j)})^2
 \label{equ_least_square}
\end{equation}

The training process aims to minimize the loss function. The weights will be adjusted gradually. The most popular algorithm for this is gradient descent. As shown in Fig. \ref{fig_gradient_descent}, the initial value of the weight $w$ is the left-most blue circle. The yellow point on the plot is the best choice of the weight because it renders the minimal loss. Denoting in vector representation, given an initial guess for the weights $\theta$, the weights are updated such that the cost moves in the direction of steepest descent \cite{robbins1951stochastic}. The updated weights $\theta_j$ are computed as in Equation \ref{equ_gradient_descent}.
\begin{equation}
\theta_j=\theta_j - \alpha\frac{\partial}{\partial\theta_j}J(\theta)
 \label{equ_gradient_descent}
\end{equation}

\begin{figure}[h!]
\begin{center}
\includegraphics[width = 11cm]{img/gradient_descent.jpg}
\caption[The illustration of gradient descent]{The illustration of gradient descent. The x-axis is the value of the weight, and the y-axis is the corresponding cost. The aim is to choose a $w$ that minimizes the cost. From saugatbhattarai.com. \label{fig_gradient_descent}}
\end{center}
\end{figure}

In Keras, training can be done using the Model Class API as follows.
\begin{lstlisting}[language=python,frame=single]
model.fit(x = input_data, y = labels, epochs=10, ...)
\end{lstlisting}

\subsubsection{Inference}
After training the network, all the weights are fixed and the network is ready to predict on new data. This predicting process is called inference. Fig. \ref{fig_train_vs_infer} illustrates the process of the training and inference process of a neuron network. Performing inference on a network is significantly faster than training that network.
\begin{figure}[h!]
\begin{center}
\includegraphics[width = 13cm]{img/train_inference.jpg}
\caption[The illustration of training and inference processes]{The illustration of training and inference processes. From nvidia.com.\label{fig_train_vs_infer}}
\end{center}
\end{figure}

In Keras, inference can be done using the Model Class API as follows.
\begin{lstlisting}[language=python,frame=single]
model.predict(x = input_data, ...)
\end{lstlisting}

\subsubsection{Convolutional Neural Networks}
One of the shiniest algorithms in deep learning is Convolutional Neural Network (CNN). In recent decades, it has led the major advancements in the field of computer vision, image and video classification, media recreation, and recommendation systems. CNN is the core layer of many state-of-the-art deep learning networks such as AlexNet (2012) \cite{krizhevsky2012imagenet}, ZFNet (2013) \cite{zeiler2014visualizing}, GoogLeNet/InceptionV1 (2014) \cite{szegedy2015going}, VGGNet (2014) \cite{simonyan2014very}, ResNet50 (2015) \cite{he2016deep}, Xception (2016) \cite{chollet2017xception}, ResNeXt-50 (2017) \cite{xie2017aggregated} and many of their variants. The basic building blocks of a CNN are convolutional layers, pooling layers, and fully connected layers, as shown in Fig. \ref{fig_CNN}. 

\begin{figure}[h!]
\begin{center}
\includegraphics[width = 13cm]{img/convlution.png}
\caption[The basic building blocks of a Convolutional Neural Network]{The basic building blocks of a Convolutional Neural Network: convolutional layers, pooling layers, and fully connected layers. From superdatascience.com \label{fig_CNN}}
\end{center}
\end{figure}

The computation of a convolution applied on a 2D image is as follows. As shown in Fig. \ref{fig_Conv2d}, the input is an image, often called a feature map, which is represented as a 2D matrix $I$. A sliding window, in this example of size 3x3 is placed on top and shifted through the entire image. Elementwise multiplication is conducted on each sub-matrix and another matrix called filter or kernel, denoted as $K$. The summation of the elementwise multiplication is placed as one value in the output matrix $I*K$. The values in the kernel matrix is learned during training. The intuition of the convolution operation is to extract the higher level features such as edges, from the input image. A CNN is not limited to one layer of convolution operation. Each convolution layer is meant to extract more abstract information from its previous input.
\begin{figure}[h!]
\begin{center}
\includegraphics[width = 13cm]{img/convolution_computation.png}
\caption[The computation of a convolution operation]{The computation of a convolution operation. A sliding window is applied on the input image $i$. Each window multiplies the kernel matrix $K$ elementwise. The summation of the product is placed as one value in the output matrix. For example, the summation of the elementwise multiplication between a sub-matrix (red) and the kernel matrix (blue) is 4 (green). This computation is $1*1+0*0+0*1+1*0+1*1+0*0+1*1+1*0+1*1 = 4$.   From deepai.org. \label{fig_Conv2d}}
\end{center}
\end{figure}

The convolution layer can be implemented in Keras in either 1D or 2D depending on the shape of the input tensor.
\begin{lstlisting}[language=python,frame=single]
keras.layers.Conv2D(filters, kernel_size, ...)
\end{lstlisting}

The convolution layer is often followed by a pooling layer. The Pooling layer is responsible for reducing the size of the extracted features from the previous convolution operation. As shown in Fig. \ref{fig_pooling}, there are two popular types of pooling operations: max pooling and average pooling. Max Pooling returns the maximum value from the portion of the input, and average pooling returns the average value. In most applications, max pooling has a better performance than average pooling. One interpretation is that the max pooling operation performs noise suppressant by discarding all noisy values while average pooling retaining them.
\begin{figure}[h!]
\begin{center}
\includegraphics[width = 13cm]{img/pooling.png}
\caption[Examples of the pooling operation]{Examples of the pooling operation. The input matrix is divided into four sub-matrices, and each of them is performed a pooling operation. For example, the purple sub-matrix has four values 4, 9, 5, 6. Max pooling computes their max value, which is 9, and the average pooling computes the average, whcih is 6. From towardsdatascience.com \label{fig_pooling}}
\end{center}
\end{figure}

The intuition behind pooling layers is that the reduced image still contains the dominant features that are adequate to represent the original image. As shown in Fig. \ref{fig_pooling_blurry}, the image after the maxpooling operation (right) is reduced in size comparing to its input image (left), but it is still clear that the image content is a car. 
\begin{figure}[h!]
\begin{center}
\includegraphics[width = 13cm]{img/pooling_blurry.png}
\caption[The intuition of the pooling operation]{The intuition of the pooling operation. The original image after a convolution operation is on the left, and the image after the max pooling operation is on the right. From analyticsvidhya.com \label{fig_pooling_blurry}}
\end{center}
\end{figure}

In Keras, pooling layers can be implemented using the layers API.
\begin{lstlisting}[language=python,frame=single]
# Max Pooling
keras.layers.MaxPooling2D(pool_size=(2, 2), ...)
# Average Pooling
keras.layers.AveragePooling2D(pool_size=(2, 2), ...)
\end{lstlisting}

The last layer(s) of a CNN is usually fully connected layers. The role of a fully connected layer is to transform the results of the convolution or pooling operations to one or few values. It is called fully connected layers because every neuron is connected to all the neurons in the next layer. As shown in Fig. \ref{fig_fc}, each node in a fully connected layers is the weighted summation of all its input. The weighs are trained during training. The computation denoted in vector notation is shown in Equation \ref{equ_linear_regression} where $h_\theta(x)$ are is the outputs, $x$ is the inputs, $\theta^T$ is the weights and b is the bias. 
\begin{equation}
h_\theta(x)=\theta^Tx + b \label{equ_linear_regression}
\end{equation}
\begin{figure}[h!]
\begin{center}
\includegraphics[height=7cm, width = 9cm]{img/fully_connected.png}
\caption[The illustration of fully connected layers]{The illustration of fully connected layers. Each neuron is connected to all the neurons in the previous layer. From oreilly.com \label{fig_fc}}
\end{center}
\end{figure}

The fully connected layer can be implemented in Keras using the Dense layer API.
\begin{lstlisting}[language=python,frame=single]
keras.layers.Dense(units=1, activation='sigmoid', ...)
\end{lstlisting}
\subsubsection{Recurrent Neural Networks}
Another star of the deep learning algorithms is Recurrent Neural Networks (RNN). Recurrent neural networks are good at modeling sequential data. The input of an RNN is a sequence. Each chunk of the sequence is called a timestep. For example, in Fig. \ref{fig_RNN_timestep}, the network input is the sentence "What time is it". Each word or punctuation in the sentence is fed to a computation unit denoted by a circle. A single occurrence of the unit is called a timestep. RNN allows each hidden state to remember certain information from previous timesteps. The ability of carrying previous information makes RNN good at speech recognition, language translation, stock predictions etc. 
\begin{figure}[h!]
\begin{center}
\includegraphics[width = 9cm]{img/RNN.png}
\caption[Time steps in a recurrent neural network]{Time steps in a recurrent neural network. The network input is the sentence "What time is it?". Each word is fed to a unit in the RNN network. The occurrence of each unit is called a timestep. From towardsdatascience.com \label{fig_RNN_timestep}}
\end{center}
\end{figure}

 A single RNN cell is shown in Fig. \ref{fig_RNN}. Notice that the major difference between a RNN layer and a fully connected layer is that there are two inputs in each RNN cell, the input at the $t$th position $x^{<t>}$ and the output of the previous cell $a^{<t-1>}$. This way, the network manages to memories the information from previous sequences. $a^{<t-1>}$ and $x^{<t>}$ are multiply by their weights in the cell, and the weights are learned during training. 

\begin{figure}[h!]
\begin{center}
\includegraphics[width = 13cm]{img/basic_rnn.png}
\caption[A basic RNN cell]{A basic RNN cell. The inputs are: output from the previous unit $a^{<t-1>}$ and $t$th input $x^{<t>}$. From stanford.edu. \label{fig_RNN}}
\end{center}
\end{figure}

A major concern with the basic RNN is Vanishing Gradient \cite{pascanu2013difficulty} which is also called short term memory problem in RNN. When doing back propagation, each node in a layer calculates its gradient with respect to the effects of the gradients, in the layer before it. The adjustment is smaller and smaller through the chain rule. However, the importance of beginning part of input sequence could be dominant. For example, in the sentence "What time is it?", the words "what time" are very important even though it is at the beginning part of the sentence. Long short-term memory (LSTM) \cite{hochreiter1997long} and Gated Recurrent Unit \cite{cho2014learning} are developed to solve the Vanishing Gradient problem in RNN.

As shown in Fig. \ref{fig_LSTM_GRU}, LSTM and GRU cells are built based on the basic RNN cell. They are computationally more expensive than the basic RNN cell but able to maintain the information from the first time step till the last time step by introducing forget gates, input gates, output gates, cell state, update gates, reset gates. Gates contains an activation function that decides whether to remember or to forget the input value. LSTM and GRU have similar cell structure, and usually researchers empirically determine which one to use for a specific deep learning application.
\begin{figure}[h!]
\begin{center}
\includegraphics[width = 13cm]{img/LSTM_GRU.png}
\caption[A single LSTM and GRU uni]{A single LSTM and GRU unit. From towardsdatascience.com \label{fig_LSTM_GRU}}
\end{center}
\end{figure}

In Keras, LSTM, GRU can be implemented as follows.
\begin{lstlisting}[language=python,frame=single]
# LSTM
keras.layers.LSTM(units = 32, activation='tanh', ...)
# GRU
keras.layers.LSTM(units = 32, activation='tanh', ...)
\end{lstlisting}
\subsubsection{Ensemble Networks}
Deep learning neural networks are nonlinear methods. This offers great flexibility in terms of fitting the training data, but a downside of this flexibility is that they are sensitive to the specifics of the training data and may find a set of weights each for a set of training data but does not apply to other data. This problem is called overfitting and often referred to as having a high variance \cite{goodfellow2016deep} as shown in Fig. \ref{fig_over_under_fitting}.
\begin{figure}[h!]
\begin{center}
\includegraphics[width = 13cm]{img/over_under_fiting.png}
\caption[The illustration of overfitting, underfitting, and good balance]{The illustration of overfitting, underfitting, and good balance. From towardsdatascience.com \label{fig_over_under_fitting}}
\end{center}
\end{figure}

One effective way of reducing the high variance is to train multiple models instead of a single model and then combine the predictions from all models, as shown in Fig. \ref{fig_ensemble_learning}. This is called ensemble learning.
\begin{figure}[h!]
\begin{center}
\includegraphics[width = 8cm]{img/ensemble_learning.png}
\caption[The illustration of ensemble learning]{The illustration of ensemble learning. Three models A, B, and C are trained using the same training data, and the final prediction is the combination of the output from all three models. From towardsdatascience.com \label{fig_ensemble_learning}}
\end{center}
\end{figure}
\subsubsection{Dropout Layers}
Ensemble learning can effectively reduce overfitting, but it is computationally more expensive as more models need to be trained. A single model can simulate the process of having different network architectures by dropping out nodes during training \cite{srivastava2014dropout}. As shown in Fig. \ref{fig_dropout}, the output of some nodes are randomly "dropped" to 0. This computationally cheap way is proven to be a remarkably effective regularization method to reduce overfitting. Dropout can be applied to all types of layers. In Keras, it is either integrated as a layer parameter or used directly as an independent layer. Below are some examples Keras dropout code.
\begin{lstlisting}[language=python,frame=single]
# 1. adding a dropout layer between two fully connected layers
keras.layers.Dense(units=64, ...)
keras.layers.Dropout(rate=0.5, ...)
keras.layers.Dense(units=1)
# 2. applying dropout in an LSTM layer
keras.layers.LSTM(units=32, dropout=0.5, ...)
\end{lstlisting}
\begin{figure}[h!]
\begin{center}
\includegraphics[height = 6cm, width = 10cm]{img/dropout.png}
\caption[An example of applying dropout in fully connected layers]{An example of applying dropout in fully connected layers. During training, the two nodes in the middle layer are dropped out meaning that their outputs to the next layer become 0. From towardsdatascience.com.\label{fig_dropout}}
\end{center}
\end{figure}
\subsubsection{Training, Validation, and Testing Dataset}
Preparing the dataset is one of the most important steps in developing deep learning applications. Generally, in order to compare with other programs, the comparative performance on one or several gold standard testing datasets are needed. These testing datasets are often called benchmark data. One can easily train and overfit the benchmark and report good performance that fails to generalize to other datasets, so during the development process, the model should not be trained on the benchmark dataset or similar data. However, we still need a way is to tell if the model is trained well. The most popular way is to split the training dataset into two parts, training and validation. During the model development cycle, the model is trained on the training data. If the performance on the training data is satisfactory, the model is evaluated against the validation data. Results obtained on training and validation data should be similar, otherwise it is an indication of overfitting. This can be done for several rounds until the achievement of good performance. Then the performance on the testing data is reported. All adjustments to the model should be based on training and validation data only, not on benchmark data.
\subsubsection{Data Augmentations and Sampling}
As discussed earlier in Fig. \ref{fig_data_performance}, generally speaking, the more data to train a model, the better performance it renders. However, trainig data can be limited due to various reason. Many techniques are designed to synthetically create more data. This is called data augmentation. In computer vision, popular data augmentations include image mirroring, random cropping, and color shifting.  

Many machine learning algorithms are designed to train on classification data with balanced data, but biological data is often unbalanced. For example, in a tumor image classification dataset, most of the images are benign, and only a small portion is malignant. Data sampling is a collection of techniques that transform a training dataset in order to balance or better balance the class distribution \cite{chawla2004special}. Popular techniques include randomly under-sampling the majority class, randomly over-sampling the minority class, and Sythetic Minority Oversampling Technique (SMOT) \cite{chawla2002smote}.
\subsection{Previous Methods}
Similar to PPI predictions, the experimental methods for PPI binding sites identification are labor and time intensive. Computational methods are needed to bridge the gap, and many have been developed \cite{cao2006enhanced, ofran2007isis, du2009improved, chen2009sequence, london2010structural, chen2010sequence, murakami2010applying, xue2011homppi, amos2011binding, jones2012psicov, asadabadi2013predictions, singh2014springs, wang2014fast, geng2015prediction, laine2015local, hwang2016hybrid, maheshwari2015prediction, liu2016prediction, wei2016protein, maheshwari2016template, jia2016ippbs, zhang2019sequence, wang2019protein, zhang2019scriber, zeng2019protein, xie2020prediction}. Out of the above mentioned twenty six computational methods, all but one are machine learning based. Computational methods can be classified into three categories, sequenced based, structure based, and combined. Among them, sequence based approaches are usually faster and cheaper. They are also more universal because comparing to sequence information, structure information is still limited. 

Machine learning methods use feature groups to represent each protein sequence. Widely used features such as position-specific scoring matrix (PSSM), evolutionary conservation (ECO), putative relative solvent accessibility (RSA) have been assessed in \cite{zhang2019comprehensive}. High-scoring segment pair (HSP) has been used in previous methods for PPI prediction \cite{li2017sprint}. One-hot vectors \cite{zhang2019sequence, zeng2019protein} and amino acid embedding \cite{asgari2015continuous, heinzinger2019modeling, asgari2019probabilistic} have also been empirically explored to represent protein sequences.

The learning structure is crucial to PPI binding sites classification problems. Previously explored architectures include random forest \cite{wei2016protein, wang2019protein}, SVM \cite{wei2016protein}, logistic regression \cite{zhang2019scriber}, Bayes classifier \cite{murakami2010applying}, artificial neural networks \cite{singh2014springs}. Recently, convolutional neural network (CNN) \cite{zeng2019protein} and recurrent neural network (RNN) \cite{zhang2019sequence} have also been applied to solve this problem. 

\subsubsection{PIPE-sites}
PIPE-Sites \cite{amos2011binding} is a algorithmic based PPI binding site program based on the PPI prediction program PIPE \cite{Pitre06_PIPE}. It requires only protein sequences and known interactions. As one of the few partner-specific sites predictors, PIPE-Sites is able to predict different binding sites for the same protein but with different partners. For example, protein A interacts with protein B and C, and the binding sites between A-B and A-C are different. PIPE-sites is able to predict these different sites in protein A.

The algorithm of PIPE-Sites is shown using an example in Figure \ref{fig_PIPE-Sites}. Using the PIPE program, the scoring matrix is built for a protein pair, the program first finds the peak location. Then it extend from the peak towards all four directions until the value drops below a predefined threshold, percentPeak. Residues within the window are considered binding sites. If there are multiple peaks, PIPE-sites will generate a ranked list, in descending order, of potential binding sites.

\begin{figure}[h!]
\begin{center}
\includegraphics[height =9 cm]{img/PIPE_site_cut.JPG}
\caption[The walk algorithm of PIPE-Sites]{The walk algorithm of PIPE-Sites. PIPE-Sites detects first the peak in the matrix 89. Then it extends towards all four directions until the score drops below a threshold percentPeak. Depends on the choice of percentPeak, different windows are drawn. Residues within the window are considered binding sites. From: \cite{amos2011binding}  \label{fig_PIPE-Sites}}
\end{center}
\end{figure} 

% In the paper of PIPE-sites \cite{amos2011binding}, the authors claim that PIPE-sites is accurate by checking the predicted sites against some domain databases. Figure \ref{fig_PIPE-Sites_eg} shows examples of two matrices that PIPE produces. The left one shows a highly possible binding site between protein Q13444 and protein O15259. The peak location is the predicted site (location 700-800 in Q13444, location 100-200 in O15259). The right picture indicates that there is no interaction between YGL055W and YBL090W as there is no high peak in the matrix. 
% \begin{figure}[h!]
% \begin{center}
% \includegraphics[height = 6 cm]{img/pipe_sites_ex.png}
% \caption{Examples of PIPE-Sites matrices. (From: \cite{amos2011binding})  \label{fig_PIPE-Sites_eg}}
% \end{center}
% \end{figure} 

\subsubsection{DLPred}
DLPred \cite{zhang2019sequence} is a sequence based protein binding predictor published in 2019. Its architecture is shown in Fig. \ref{fig_DLPred}. The main layers in DLPred model are simplified LSTM layers and fully connected layers. The simplified LSTM is developed by the authors, and the main purpose is to reduce the time consumption. The features used are PSSM, physical properties, hydropathy index, physicochemical characteristics, PKx (the negative of the logarithm of the dissociation con- stant for any other group in the molecule), conservation score, and one-hot encoding for protein sequences. DLPred also adopts a filtered sampling technique such that sequences with more binding residue are picked. They empirically show that this helps deal with the imbalance of the protein biding problem.
\begin{figure}[h!]
\begin{center}
\includegraphics[height = 6 cm]{img/DLPred.png}
\caption[The architecture of DLPred]{The architecture of DLPred. Three BSLSTM layers are followed by two fully connected layers with highway connection.  (From: \cite{zhang2019sequence})  \label{fig_DLPred}}
\end{center}
\end{figure} 

\subsubsection{DeepPPISP}
DeepPPISP \cite{zeng2019protein} is a recent protein binding prediction utilizing secondary structure. As shown in Fig. \ref{fig_DeepPPISP}, the network structure of DeepPPISP consists of a TextCNN component to extract feature features and two fully connected layers to classify protein binding residue. The TextCNN is a variation of the convolution layers meant to handle lower dimension tensor. DeepPPISP is highlighted for the usage of both local and global information. For each given amino acid, the local information is the neighboring seven amino acids of the target amino acid, and the global information is its     neighboring 500 amino acids. It is shown empirically in the paper that the usage of global information largely helps the prediction. 
\begin{figure}[h!]
\begin{center}
\includegraphics[width = 14cm]{img/DeepPPISP.png}
\caption[The architecture of DeepPPISP]{The architecture of DeepPPISP. The inputs are both sub-sequence (local) and whole protein sequence. For local features, sequence embedding, PSSM, and secondary structure are combined and then flattened a local feature vector. For global features, sequence embedding, PSSM, and secondary structure are concatenated and then passed through TextCNN layers. The features extracted from local and global sequence are further concatenated and passed through fully connected layers.  (From: \cite{zeng2019protein})  \label{fig_DeepPPISP}}
\end{center}
\end{figure} 

\subsubsection{SCRIBER}
SCRIBER \cite{zhang2019scriber} is another sequence based protein binding residue predictor published in 2019. It uses the following feature: putative relative solvent accessibility (RSA), evolutionary conservation (ECO), relative amino acid propensity (RAAP) for binding, and the selected relevant physiochemical properties, putative protein binding intrinsically disordered regions, putative secondary structure (SS), and selected physicochemical properties of amino acids (aliphaticity, aromaticity, acidity and size). The core layer of the model is a simple logistic regression. The key innovative idea of SCRIBER is that it is trained using not only protein-protein binding information, but also protein binding information with RNA, DNA, ligand, and other type. As shown in Fig. \ref{fig_SCRIBER}, in the first layer of SCRIBER, five different models are trained using five binding information. In the second layer, the prediction values of the five models are used as input to train another model aiming to predict protein-protein binding sites only. The intuition is that models make false predictions because they mistakenly cross-predict other types of binding. For example, a non-protein-protein binding residue is predicted to have high propensity for protein-protein binding, but it is actually a residue for DNA binding. The mistake happens because of the different types of bindings share some similar properties. Because SCRIBER is not a deep learning based method, it uses a relatively small training dataset containing 843 proteins.
\begin{figure}[h!]
\begin{center}
\includegraphics[height = 7cm]{img/SCRIBER.png}
\caption[The architecture of SCRIBER]{The architecture of SCRIBER. The first layer predicts the propensities for DNA-binding (red), RNA-binding (violet), small ligand-binding (orange), other-binding (blue) and protein-binding (green) residues. These propensities are used together in the second layer to predict protein-protein binding sites. (From: \cite{zhang2019scriber})  \label{fig_SCRIBER}}
\end{center}
\end{figure} 
\section{Methods}
This section describes in details our program DELPHI for binding site prediction. It is trained on a big dataset comparing to the competitors. DELPHI combines a CNN and a RNN components with a many-to-one structure. It uses twelve features to represent protein sequences, among them three are used first time for site prediction. 

\subsection{Training Data Preparation}
Training data is one of the most important factors in training a model. An trend in applied machine learning research field is that a good publication always comes good data. 

We construct next our training and validation data such that there are no similarities above 25\% between (i) training and validation data and (ii) between testing (see Section \ref{testing_data} for details) and the union of training and validation. Also, we want relatively large training data as it is beneficial for deep learning models.
    
\subsubsection{Raw Training Data}
We first collected a relatively large amount of raw data. A high quality dataset was provided in \cite{zhang2019comprehensive}, where proteins were solved structurally in complex from the BioLiP \cite{yang2012biolip} database. In BioLip, two residue are considered binding if the distance between the atoms of the 
two residues $<$0.5+sum of the Van der Waal's radii of the two atoms. BioLip IDs are then mapped into UniProt IDs using SIFTS \cite{velankar2012sifts} to insure we work with complete sequences. Then we left with sequences annotated with Protein, DNA, RNA, and small ligands binding information at the residue level. We kept only the sequences with protein-protein binding information to focus on protein-protein binding.
\subsubsection{Similarities Eliminations}
We further processed this dataset by eliminating similarities. We removed any sequences from training dataset sharing more than 25\% similarities, as measured by PSI-CD-hit \cite{li2006cd,fu2012cd}, with any sequences in testing datasets. It is well acknowledged that similar sequences between training and testing datasets negatively affect the generalization of the evaluated performance of a machine learning model. Also, proteins with higher levels of similarity can be accurately predicted by the alignment-based methods \cite{zhang2018review}. The similarity threshold is picked differently by different programs ranging from 25\% to 50\%. We picked the strictest value of 25\% to match to one of the closest competing programs, SCRIBER \cite{zhang2019scriber}, for a fair comparison. We used PSI-CD-HIT because it is fast, accurate and well maintained in the CD-HIT suite. Also, it is able to cluster sequences with similarity at low as 25\%, whereas CD-HIT works only down to 40\%. Finally, we ran PSI-CD-hit again on the rest of the training protein sequences so no sequences shared more than 25\% similarities among training data. This ensures the training data is as diverse as possible. The commands to cluster proteins using PSI-CD-HIT are as follows.
\begin{lstlisting}[language=bash,frame=single]
# convert fasta file to blast database format
$ makeblastdb -in [fasta_file_name] -dbtype prot
# compute protein clusters
$ psi-cd-hit/psi-cd-hit.pl -i [fasta_file_name] -o [out_file_name] -c 0.25
\end{lstlisting}
\subsubsection{Data Split}
After the similarities elimination process, A dataset of 9,982 protein sequences was constructed. From it, we randomly pick eight ninth (8,872) as the training dataset and one ninth (1,110) as the validation dataset. This can be done by using the model\_selection module from sklearn. The commands are as follows.
\begin{lstlisting}[language=python,frame=single]
kfold = KFold(n_splits=9)
for train, test in kfold.split(dataset_all)
    ...
\end{lstlisting}


\subsection{Features}
DELPHI uses 12 features groups, shown in Table \ref{tab_feture}, including high-scoring segment pair (HSP), a variation of 3-mer amino acid embedding (ProtVec1D), position information, position-specific scoring matrix (PSSM), evolutionary conservation (ECO), putative relative solvent accessibility (RSA), relative amino acid propensity (RAA), putative protein-binding disorder, hydropathy index, physicochemical characteristics, physical properties, and PKx. Each input is represented by a 39 dimensional feature vector profile. To the best of our knowledge, this study is the first time that HSP, ProtVec1D, and position information are being used in binding sites classification problems. The computation of each of these two new features is described next.

\begin{table}[htbp]
  \centering
  \caption[The feature groups used by DELPHI]{The feature groups used by DELPHI. The first column indicates the name of each feature. The second column describes the program used to obtain the feature. ``Load'' means the value for a specific amino acid is known from previous work, and it is loaded in the DELPHI program. ``Compute'' means DELPHI performs additional computation to that feature. The last column shows the dimension of each feature group. Full details are given in the text.}
    \begin{tabular}{p{20.93em}p{11em}r}
    \toprule
    Feature & Program & \multicolumn{1}{p{5.145em}}{Dimension} \\
    \midrule
    High-scoring segment pair (HSP) & SPRINT and compute & 1 \\
    % \midrule
    3-mer amino acid embedding (ProtVec1D) & Load and compute & 1 \\
    % \midrule
    Position information & Compute & 1 \\
    Position-specific scoring matrix (PSSM) & Psi-Blast & 20 \\
    % \midrule
    Evolutionary conservation (ECO) & Hhblits & 1 \\
    % \midrule
    Putative relative solvent accessibility (RSA) & ASAquick & 1 \\
    % \midrule
    Relative amino acid propensity (RAA) & Load  & 1 \\
    % \midrule
    Putative protein-binding disorder & ANCHOR & 1 \\
    % \midrule
    Hydropathy index & Load  & 1 \\
    % \midrule
    Physicochemical characteristics & Load  & 3 \\
    % \midrule
    Physical properties & Load  & 7 \\
    % \midrule
    PKx   & Load  & 1 \\
    \bottomrule
    \end{tabular}%
  \label{tab_feture}%
\end{table}%

High-scoring segment pair (HSP): An HSP is a pair of similar sub-sequences between two proteins. The similarities between two sub-sequence of the same length are measured by scoring matrices such as PAM and BLOSUM. SPRINT \cite{li2017sprint} is used for computing all HSPs as it detects similarities fast and accurately among all proteins in training and testing. The detailed description of computing HSP in SPRINT is discussed in Section \ref{sec_detect_simi}. After obtaining the HSPs, the score for the $i$th residue, $P[i]$, of a testing protein $P$, denoted $\text{HSP}_{\text{score}}(P[i])$, is calculated as follows. Assume we have an HSP, $(u,v)$, between $P$ and a training protein $Q$ such that $u$ covers the residue $P[i]$, that is, position $i$ in $P$ is within the range covered by $u$. Let $j$ be the position in $Q$ that corresponds to $i$, that is, the distance in $P$ from the beginning of $u$ to $i$ is the same as the distance in $Q$ from the beginning of $v$ to $j$. If $Q[j]$ is a known interacting residue, then we add the PAM120 score between $P[i]$ and $Q[j]$ to the HSP score of $P[i]$:
\[
\text{HSP}_{\text{score}}(P[i]) = \!\!\!\!\!\!\sum_{\stackrel{\text{\tiny HSPs covering $P[i]$}}{\text{\tiny $Q[j]$ interacting residue}}}\!\!\!\!\!\! \max(0, \text{PAM120}(P[i], Q[j])) \ .
\]

SPRINT has a branch specifically developed for computing HSPs for binding sites.\\ \href{https://github.com/lucian-ilie/SPRINT/tree/produce_interface_HSP}{https://github.com/lucian-ilie/SPRINT/tree/produce\_interface\_HSP}.\\ The default parameters and the following command are used.
\begin{lstlisting}[language=bash,frame=single]
# ${PRO_SEQ}: all sequences in training and testing in one file, 
in fasta format
# ${HSP_OUT_FN}: the output HSP file name
bin/compute_HSPs -p ${PRO_SEQ} -h ${HSP_OUT_FN} 
\end{lstlisting}

The 3-mer amino acid embedding (ProtVec1D): We developed this feature based on ProtVec \cite{asgari2015continuous}. ProtVec uses word2vec \cite{mikolov2013distributed} to construct a one hundred dimensional embedding for each amino acid 3-mer. It is shown in \cite{asgari2015continuous} that ProtVec can be applied to problems such as protein family classification, protein visualization, structure prediction, disordered protein identification, and protein-protein interaction prediction. Since using the ProtVec embedding in our program slows down significantly the deep learning model, especially during training, we replaced the one hundred dimensional vector by one dimensional value, which is the sum of the one hundred components; we call this ProtVec1D. According to our tests, ProtVec1D achieves, in connection with the other features, the same prediction performance as ProtVec. In Keras, a pre-trained embedding layer can be use as follows.
\begin{lstlisting}[language=python,frame=single]
# initialize the pre-trained embedding as a dictionary
embedding_matrix = {}
...
# add an embedding layer with pre-traind weights
keras.layers.Embedding(input_dim = 31, output_dim = 100,
weights=[embedding_matrix], trainable=False)
\end{lstlisting}

Position information: In natural language processing tasks, position information is shown useful. The popular network Bert \cite{devlin2018bert} utilizes this information to guide its translation process. It is also shown by DeepPPISP \cite{zeng2019protein} that the global information of a protein helps the prediction of interfaces. Inspired by the two networks, we use the position information of each amino acid as an input feature hoping that it provides certain global information of a protein. The position of an amino acid in a protein is in the range of 1 to the length of the protein. Then the position is divided by protein's length so that the value is between 0 to 1.

Position-specific scoring matrix (PSSM): PSSM matrices are widely used in protein interaction related problems. They contain the evolutionary conservation of each amino acid position by aligning an input sequence with protein databases. The PSSM matrices are computed using PSI-Blast \cite{altschul1997gapped} with the expectation value (E-value) set to 0.001 and the number of iterations set to 3. PSI-Blast ((2.6.0+)) performs multiple alignment on each input sequence against the non-redundant database. PSSMs take fairly a long time to compute, but for each sequence, it only needs to be done once. The psiblast command I used is as follows.
\begin{lstlisting}[language=bash,frame=single]
# ${input_protein_fasta}: the input protein sequence.
# ${nr}: the path of the non-redundant database
# ${out_pssm}: the output PSSM file name
psiblast -query ${input_protein_fasta} -db ${nr} -num_threads 3
-out_ascii_pssm ${out_pssm} -num_iterations 3
\end{lstlisting}

Evolutionary conservation (ECO): ECO also contains evolutionary conservation, but in a more compact way. To compute the ECO score, the faster multiple alignment tool HHBlits \cite{remmert2012hhblits} is run against the non-redundant Uniprot20 database with default parameters. The one dimensional conservation value is computed using the formula described in \cite{zhang2019comprehensive}. The commands used for conducting multiple alignment using hhblits as follows.
\begin{lstlisting}[language=bash,frame=single]
$ hhblits -i [input] -ohhm [output]  -d [database] -hide_cons 
-hide_pred -hide_dssp -v 0  -neffmax 1 -n 1
-out_ascii_pssm ${out_pssm} -num_iterations 3
\end{lstlisting}

Putative relative solvent accessibility (RSA): The solvent accessibility is predicted using ASAquick \cite{faraggi2014accurate}. The values are obtained in the from rasaq.pred file in each output directory. The ASAquick command is as follows.
\begin{lstlisting}[language=bash,frame=single]
$ ASAquick [input]
\end{lstlisting}


Relative amino acid propensity (RAA): The AA propensity for binding is quantified as relative difference in abundance of a given amino acid type between binding residues and the corresponding non-binding residues located on the protein surface. The RAA for each amino acid type is computed in \cite{zhang2019comprehensive} by using the program of  \cite{vacic2007composition}.

Putative protein-binding disorder: The putative protein-binding disorder is computed using the ANCHOR program \cite{dosztanyi2009anchor}. The used ANCHOR command is as follows.
\begin{lstlisting}[language=bash,frame=single]
# $input_protein: the input protein sequence, in fasta format
ASAquick $input_protein
\end{lstlisting}
Hydropathy index: Hydrophobicity scales is experimentally determined transfer free energies for each amino acid. It contains energetics information of protein-bilayer interactions \cite{wimley1996experimentally}. The values are computed in \cite{kyte1982simple}.

Physicochemical characteristics: For each protein, this includes three features: the number of atoms, electrostatic charges and potential hydrogen bonds for each amino acid. They are taken from \cite{zhang2019sequence}.

Physical properties: We use a 7-dimensional property of each amino acid type. They are a steric parameter (graph-shape index), polarizability, volume (normalized van der Waals volume), hydrophobicity, isoelectric point, helix probability and sheet probability. The pre-computed values are taken from \cite{zhang2019sequence}.

PKx: This is the negative of the logarithm of the dissociation constant for any other group in the molecule. The values for each amino acid type is taken from \cite{zhang2019sequence}.

After computing all the feature vectors, the values in in each row vector are normalized to a number between 0 to 1 using formula (\ref{eq_normalized}) where \textit{v} is the original feature value, and max and min are the biggest and smallest value observed in the training dataset, resp. This is to ensure each feature group are of the same numeric scale and help the model converge better:

\begin{equation}
v_\text{norm}=\dfrac{v-\text{min}}{\text{max}-\text{min}}\label{eq_normalized}
\end{equation}

\subsection{Model Architecture}
\subsubsection{Architecture Overview}
DELPHI has an architecture that is inspired by ensemble learning. The intuition of the design is that different components of the model capture different information, and another deep neural network is trained to only select the most useful ones. As shown in Fig. \ref{fig_architecture}, the model consists of three parts, a convolutional neural network (CNN) component, an recurrent neural network (RNN) component, and an ensemble component. The core layers of the CNN and RNN components are convolution and bidirectional gated recurrent units (GRU) layers. The ensemble model decodes the output of the first two components.  

\begin{sidewaysfigure}
\centering
\includegraphics[height=8cm]{img/Model_architecture.pdf}
\caption[The architecture of DELPHI]{The architecture of DELPHI. Left: the CNN component of the model. Middle: the RNN component of the model. Right: The ensemble model. 
  \label{fig_architecture}}
\end{sidewaysfigure}
\subsubsection{Many-to-one Structure}
Another very useful characteristic of the model is its many-to-one structure, meaning that the information of many residues are used to prediction the binding propensity of the centered single residue. As illustrated in Fig. \ref{fig_many2one}, for each amino acid as the prediction target, a window of size 31, centred on the amino acid position, is used to collect information from the neighbouring 30 residues to help the prediction. A sliding window is used to capture each 31-mer. The size 31 is determined experimentally. The beginning and the ending part of the sequence are padded with zeros. The many-to-one structure has two advantages. Firstly, it serves as a data augmentation technique. Deep learning models need large amount of data to train, and comparing to image classifiers, models in proteomics have access to orders of magnitude less data. Using each residue multiple times during the training process helps the model learn better. Secondly, it makes the model more robust. The lengths of protein sequence vary from less than one hundred to several thousand, and most a many-to-many models have a fixed input length of near 500. During training, sequences around length 500 are often picked. However, during testing, input sequences are random and need to be either padded or cut into pieces. The different average lengths between training and testing could potentially make the model less general. 
\begin{figure*}
\centering
\includegraphics[width=\textwidth]{img/many_2_one.pdf}
  \caption[The many-to-one prediction]{The many-to-one prediction. Sliding windows of size 31, stride 1 are put on top of an input protein sequence. Each time, a sub-sequence of length 31 is extracted. The model predicts the protein-binding propensity of the middle amino acid for each sub-sequence.
  \label{fig_many2one}}
\end{figure*}
\subsubsection{Architecture of the CNN Network}
The CNN model one has a concise structure: one convolution layer, one maxpooling layer, one flatten layer, and two fully connected layers. For each input sub-sequence of size 31, a 2D feature profile of size 39 by 31 is constructed. The 2D vector is reshaped into 3D and then passed to a convolution 2D layer, followed by a maxpooling layer. The intuition of using the convolution and maxpooling layers is that a 2D protein profile vector can be considered as an image with one channel, and the CNN model captures the combination of several features in a partial image. The results are flattened and then fed into two fully connected layers with dropout for regularization. The last fully connected layer has one unit with activation function sigmoid, so that the output is a single value between 0 to 1. The higher the value, the more confident the CNN model claims that the residue is a PPI binding site.
\subsubsection{Architecture of the RNN Network}
The RNN component has the following structure: one bidirectional GRU layer, one flatten layer, and two fully connected layers. Similar to the CNN component, a 2D feature profile of size 39 by 31 is built for each 31-mer. The feature profile is passed to a bidirectional gated recurrent units (GRU) layer with the intention to memories the dependency and relationship among the 31 residues. We set the GRU layer to return a whole sequence as opposed to return a single value. The results are flattened and fed into two fully connected layers with dropout. The output of the RNN network is also a single value between 0 to 1.
\subsubsection{Architecture of the Ensemble Network \label{section_ensemble}}
The final model combines the core layers of the above mentioned CNN and RNN models and tries to further extract essential information of protein binding. The ensemble network takes a sequence of length 31 as its input. Similar to the CNN and RNN components, a 39 by 31 feature vector is constructed and passed to both a convolution layer and a bidirectional GRU layer. The output of the convolution layer is passed on to a maxpool layer and then flattened. The GRU output is also flattened. Then the outputs of the two flatten layers are concatenated and passed on to two fully connected layers with dropout. The last fully connected layer has one output unit with a sigmoid activation function, so the final output is a single value between 0 to 1, indicating the propensity of being binding sites. This is the final output of the entire model. 

Fine tuning is used in this ensemble model. The convolution layer in the CNN network and the bidirectional GRU layer in the RNN network are tuned separated using the same training/validation dataset. After achieving the best performance on the CNN and the RNN components, the weights of the convolution and the GRU layer are saved to files. In the ensemble model, the convolution and the GRU layer load the saved weights from the file and freeze the weights, so that during the process of training, the convolution and the GRU layer stay unchanged. Training and validation data are used again only to train the fully connected layers in the ensemble model. The code snippet on fine tuning is as follows.
\begin{lstlisting}[language=python,frame=single]
# initialize model checkpoint
mc = ModelCheckpoint("SavedModelAndWeights.h5",
monitor = 'val_loss', mode='min', save_best_only=True)
# save model and weights during training
model.fit([features, labels, callbacks=[mc]
# in the ensemble model, load weights for conlution and GRU
model.load_weights("SavedModelAndWeights", by_name=True)
\end{lstlisting}

\subsection{Parameter/Hyper-parameter Tuning}
Parameters and hyper-parameters are chosen based on the training dataset while applying early stopping \cite{prechelt1998early} on the validation set. As shown in Fig. \ref{fig_earlyStop}, early stopping halts the training process when a performance drop on the validation set is detected. This is to avoid overfitting the training dataset. The early stopping code snippet using Keras callbacks module is as follows.
\begin{figure}[!h]
\begin{center}
\includegraphics[width = 9cm]{img/early_stop.png}
\caption[The illustration of early stopping]{The illustration of early stopping. The training process stops at the arrow point. The right side of the point is overfitting the training dataset. The left side of the point is underfitting. \label{fig_earlyStop}}
\end{center}
\end{figure}

\begin{lstlisting}[language=python,frame=single]
es = EarlyStopping(monitor='val_loss', mode='min', patience=4)
model.fit([features, labels, callbacks=[es]
\end{lstlisting}
We chose all parameters with the purpose to maximize area under the precision-recall curve (AUPRC) of the training data. All testing results are then carried using the already tuned model. All parameters and hyper-parameters used in this model are shown in Table \ref{tab_parameter}.

The DELPHI model takes 2.1 hours to train the CNN component, 0.5 hour to train the RNN component and 1.3 hours to train the ensemble model on our testing cluster.

% Table generated by Excel2LaTeX from sheet 'Sheet1'
\begin{table}[htbp]
  \centering
  \caption[Parameters used in DELPHI]{Parameters used in DELPHI. Parameters are divided into four groups: CNN, RNN, ensemble model, and hyper-parameters.}
    \begin{tabular}{lr}
    Parameter & Value \\
    \midrule
    \midrule
    Epoch in CNN & 8 \\
    Kernel size in CNN & 5 \\
    Stride in CNN & 1 \\
    Padding in CNN & valid \\
    Number of filters in CNN & 64 \\
    Fully connected unit in CNN & 64, 1 \\
    \midrule
    Epoch in RNN & 9 \\
    GRU unit & 32 \\
    Fully connected unit in RNN & 64, 1 \\
    \midrule
    Epoch in ensemble & 5 \\
    Fully connected unit in ensemble & 96, 1 \\
    \midrule
    Batch size & 1024 \\
    Dropout rate & 0.3 \\
    Optimizer & Adam ($\beta_1=0.9, \beta_2=0.999$) \\
    Patience in early stop & 4 \\
    Loss function & binary cross entropy \\
    Learning rate & 0.002 \\
    \bottomrule
    \end{tabular}%
  \label{tab_parameter}%
\end{table}%
\subsection{Implementation}
Many open sourced programs in bioinformatics are not well written and maintained. It is frustrating for future researchers spending weeks even months just to compile and run a program. The engineering quality in academia is often compromised and forgiven. Indeed, research novelty is more important than good coding structure, git practise, and readability, but good engineering always plays a key role in successful projects. Projects like LLVM \cite{lattner2004llvm} and TensorFlow \cite{tensorflow2015-whitepaper} have thrived for years with huge open source communities not only because of the back scene innovational ideas, but also because of the excellent code base and infrastructure. We provide here some environment and implementation details hoping to help users to use DELPHI better, recreate results easier, and improve upon the current code base.

\subsubsection{Environment Configuration}
System environment configuration is a step many researchers ignore or choose to do improperly. Each project has its unique compiler, package, and environmental variable settings. DevOp skills are not trivial for AI engineers without system configure experience. There are several solutions to ensure users running a program under the correct environment. Containers like docker is a complete industrial solution. Docker stores the entire operation system (OS) in an image, and the users can load the image such that the environment is identical. However, containers at the OS level is too heavy weighted. Lighter weighted solutions such as Python Virtual Environment, Anaconda are sufficient at package level, especially for python package management. According to GitHub statistics in 2019, about 50\% of the machine learning languages are written in Python.

DELPHI uses Python (3.5) Virtual Environment to manage all packages. Each virtual environment has its own packages and dependencies installed locally. Users can switch between any project environment without root permission nor modifying system directories. All packages needed by DELPHI is stored in Requirements.txt. The commands to create the virtual environment for DELPHI are as follows.
\begin{lstlisting}[language=bash,frame=single]
# create a python3 environment
$ virtualenv -p [python3.5_path] [environment_name] 
# activate the newly created environment
$ source [environment_name]/bin/activate 
# install all packages neede by DELPHI
$ pip3 install -r Requirements.txt 
\end{lstlisting}

The DELPHI program is written in python, with Keras \cite{chollet2015keras} APIs and TensorFlow GPU back end. All features are computed from sequence only. We alleviate the burden of feature computation from users by providing all computation programs and a pipeline script. We ease the system configuration process by providing users a pip package list which enables one-command installation. 

\subsubsection{Class Weights}
Classifying protein binding residue is an imbalanced problem. To cope with that, different class weights \cite{ting2002instance} are assigned to the positive and negative samples, so that the model pays more attention to the minority class, which is the binding sites.  The values are determined by the inverse of the class distribution in the training datasets. In our program, the weights are 0.55 and 4.97 for the non-binding and binding sites respectively. The code piece for implementing the class weights is as follows.
\begin{lstlisting}[language=python,frame=single,basicstyle=\small]
# compute the class weight in a balanced manner
cw = class_weight.compute_class_weight('balanced',
                np.unique(all_labels),all_labels)
# apply the class weights during training
model.fit(class_weight = cw, ...)
\end{lstlisting}

\subsubsection{Data Shuffling}
During training, we shuffle the data before each epoch. Since the sliding window is used to extract each 31-mer, adjacent data entries are very similar; only the first and the last residue differ from the previous and the next data entry. Shuffling the whole training data diversifies the input in each batch. We experimentally trained the model with and without data shuffling, and shuffling the data rendered better predictions. This can be done by 
\begin{lstlisting}[language=python,frame=single]
  model.fit(shuffle=True, ...)
\end{lstlisting}

\subsection{The DELPHI Web Server}
In order to facilitate the use of DELPHI, we built a web server. The web interface (\texttt{www.csd.uwo.ca/\textasciitilde{}yli922/index.php}) of DELPHI is shown in Fig. \ref{fig_DELPHI_web}. The user can enter protein sequences in FASTA format, and the result will be sent via email upon completion. 
\begin{figure}[h!]
\begin{center}
\includegraphics[width =13 cm]{img/DELPHI_web_server.png}
\caption[The web interface of DELPHI]{The web interface of DELPHI. \label{fig_DELPHI_web}}
\end{center}
\end{figure} 

\subsubsection{The Architecture of the Web Server}
The overall architecture of the DELPHI web server is shown in Fig \ref{fig_DELPHI_web_arch}. The DELPHI web server consists two components, a front end machine which is responsible for hosting web pages and sending emails, and a back end machine which is mainly for the DELPHI computation. The front end server has the public domain name csd.uwo.ca, and the backend machine is a cloud VM provided by Compute Canada. Users will enter their input sequences using the web page interface, and the sequences will be saved in a designated directory (see "saved user input" in Fig. \ref{fig_DELPHI_web_arch}). A Cron job is set to run every five minutes to automatically monitor the directory. If newly added files are detected, it will trigger a transfer of the input data to a directory (the "user input" in Fig. \ref{fig_DELPHI_web_arch}) on the back end machine. Similarly, another Cron job on the back end server will automatically run every five minutes to monitor that directory. If new files are found, then it will start the input validation and the computation of the DELPHI program. Once the results are ready, the results file will be transferred back to the front end server and emailed to the user.
\begin{figure}[h!]
\begin{center}
\includegraphics[width =13 cm]{img/DELPHI_webserver_architecture.pdf}
\caption[The architecture of the DELPHI web server]{The architecture of the DELPHI web server. The DELPHI web server consists of two parts, a front end server and a backend server. The front end server hosts the web page and saves the user input to a directory. The back end server is configured with all DELPHI required environment and dependencies. Cron jobs are used to monitor and trigger the data transfer, computation, and the mail action. \label{fig_DELPHI_web_arch}}
\end{center}
\end{figure} 

\subsubsection{Front End Server Configuration}
The front end server is mainly maintained by the Computer Science Department of the University of Western Ontario. It already has the required components for web web service such as Apache, PHP, and SMTP. With that provided, the main task for hosting the DELPHI web page is to write the HTML web page. The data saving part is implemented in PHP. The input data directory is monitored by a cron job, and it can be edited using the following commands.
\begin{lstlisting}[language=bash,frame=single]
crontab  -e
#run every 5 minutes
*/5 * * * * [path_to_script].sh >> [path_to_log_file] 2>&1
\end{lstlisting}

\subsubsection{Back End Server Configuration}
The back end server is a VM provide by Compute Canada. It has a Ubuntu 18.04 LTS operating system with 12 cores, 24GB of RAM, and 400GB of storage. The could machine needs to be configured first on security settings. The instructions can be found at \texttt{docs.computecanada.ca/wiki/Cloud\_Quick\_Start}. For security reasons, the VM does not allow credential login, so SSH key has to be used. The hard drive is applied separately and need to be mounted to the VM manually. The commands used are as following.
\begin{lstlisting}[language=bash,frame=single]
# list available volumes
sudo fdisk -l
#mount volume to a directory
mount [path_to_volume] [path_to_directory]
\end{lstlisting}

All packages and dependencies required by DELPHI need to be installed. The detailed installation instructions can be found at \texttt{github.com/lucian-ilie/DELPHI}.

\subsubsection{Communications between the Front and Back End Servers}
The SSH public key of both servers are written into the file
\texttt{\textasciitilde{}/.ssh/authorized\_keys} on each other, so that transfer can be done without entering passwords. The data transfer process is conducted using rsync which is faster than SCP. The mail command is triggered by appending mail command to SSH command. Both commands are shown as follows.
\begin{lstlisting}[language=bash,frame=single]
# the rsync command
rsync  -r -l -K --progress [source_dir] [destination_dir]
#execute command on remote machines
ssh [username@host] "mail [options for mail]"
\end{lstlisting}

\subsubsection{Pre-computing PSSMs}
The bottle neck of the DELPHI program is the PSSM computation. In order to accelerate the computation speed, we pre-computed the PSSMs for all testing data as well as the human proteome. We stored the PSSM files and on the back end server and built a dictionary, with key as protein sequence and value as the PSSM file location, in JSON format. When running DELPHI, it first checks if the input sequences is in the database (the JSON file). If the PSSM file exists, DELPHI skips the PSSM computation, otherwise DELPHI computes it. The average computation time for a protein of length 500 on the web server is 3 minuets if having pre-computed PSSM and 15 minutes if without pre-computed PSSM.

\subsubsection{Job Scheduling}
As shown in Fig. \ref{fig_DELPHI_web_arch}, the back end server has a scheduler to restrict the number of concurrent running jobs to one. If the number of jobs are too many, depends on the input sequence length, DELPHI might exceed the memory limit. The schedule is implemented as follows. Each time a DELPHI job is running, it creates a temporary file delphiIsRuning.tmp. The file will be removed at the completion of the process. Before running each new job, the scheduler script will check the existence of delphiIsRuning.tmp. If it exits, the sleep command will be invoked, and it will wait ten minutes and check again.



\section{Results}
We have comprehensively compared DELPHI with nine state-of-the-art machine learning based methods. The comparative analysis shows that DELPHI has a better prediction accuracy in all evaluation metrics. 

The methods are selected using the following criteria. First, the program is a sequence based method as sequence information is readily available for most proteins. Second, the program is available in the form of source code or web server. Lastly, the program takes in any input sequence in FASTA format and produces the results on an average-length protein within thirty minutes. Following these criteria, DLpred \cite{zhang2019sequence}, SCRIBER \cite{zhang2019scriber}, SSWRF \cite{wei2016protein}, SPRINT \cite{taherzadeh2016sequence}, CRF-PPI \cite{wei2015cascade}, LORIS \cite{dhole2014sequence}, SPRINGS \cite{singh2014springs}, PSIVER \cite{murakami2010applying}, and SPPIDER \cite{porollo2007prediction} are selected.

Ideally, we would like to have every program trained on sequences of less than 25\% similarity with any testing datasets, but all competing programs are either pre-trained or work as a web server, so there may be some level of similarities for them. Although we had to use other programs as is, we selected DELPHI's training dataset to ensure it meets similarity criteria. Notice that this is to the disadvantage of DELPHI.

All competing methods are pre-trained using their own training and validation datasets. The most recent two programs, DLPred and SCRIBER, use 5719 and 843 training proteins respectively. The training dataset of DLPred is obtained from CullPDB datasets \cite{wang2003pisces} and further filtered by the authors. The SCRIBER training dataset is originally from the BioLip database. This dataset contains also protein binding information with DNA, RNA, and ligand, which is used by SCRIBER.

All tests have been performed on a Linux (Ubuntu 16.04) machine with 24 CPUs (Intel Xeon v4, 3.00GHz), 256GB memory, and a Nvidia Tesla K40c GPU.
\subsection{Testing Datasets \label{testing_data}}
In order to be able to test on common benchmark data, we choose first the testing datasets, then construct the training datasets such that they are highly dissimilar with testing.

Five datasets are used in the comparative assessment. We name them by the size of the data: Dset\_186, Dset\_72, Dset\_164, Dset\_448, and Dset\_355. The first four are publicly available datasets from previous studies \cite{murakami2010applying} \cite{dhole2014sequence} \cite{zhang2019scriber}, and the last one, Dset\_355, is a subset of Dset\_448. Dset\_186, Dset\_72, Dset\_164 have been widely used and explored as benchmark datasets by numerous publications; Dset\_448 is more recent.

Dset\_186 and Dset\_72 were constructed by \cite{murakami2010applying}. Dset\_186 was built based on a PDB collection \cite{berman2002protein}, to which a six-step filtering process was applied to refine the data including removing structures with missing residues, removing chains with the same UniprotKB/Swiss-Prot accession, removing transmembrane proteins, removing dimeric structures, removing proteins with buried surface accessibility and interface polarity under certain range, and similarities elimination. Dset\_72 was constructed based on the protein-protein benchmark set version 3.0 \cite{hwang2008protein} with the similarities with Dset\_186 removed.

Dset\_164 was constructed by \cite{dhole2014sequence} with the same filtering technique as for Dset\_186 and Dset\_72 on newly annotated proteins in PDB since the publication of Dset\_186 (Jun.~2010 to Nov.~2013). 

Dset\_448 was constructed by \cite{zhang2019scriber}. The raw data of Dset\_448 was from the BioLip database \cite{yang2012biolip} where binding sites are defined if the distance between an atom of a residues and an atom of a given protein partner <0.5 \AA{} plus the sum of the Van der Waals radii of the two atoms. The raw data was further processed by removing protein fragments, mapping BioLip sequences to UniProt sequences, and clustering so that no similarities above 25\% are shared within Dset\_448. This dataset is the most recent one as well as the largest. 

Dest\_448 cannot be used to test one of the top competing programs, DLPred, because it contains 93 proteins that share more than 40\% similarity with DLPred's training set. We built Dset\_355 by removing these proteins from Dset\_448.

\begin{table*}[htbp]
    \centering
    \caption[The datasets used for training, validation, and testing]{The datasets used for training, validation, and testing. The first column gives the dataset names. The second column contains the number of proteins in each dataset. The third, fourth, and fifth columns represent the total number of residue, the number of binding, and the number of non-binding residues in each dataset. The last column represents the percentage of the binding residues out of total.}
    \begin{tabular}{p{12em}rrrrr}
    \toprule
    Dataset & Proteins & \multicolumn{3}{c}{Residues} & \multicolumn{1}{c}{\% binding} \\ \cline{3-5}
    & & total & binding & non-binding & \multicolumn{1}{c}{out of total}\\ \hline
    Dset\_448 & 448   & 116,500 & 15,810 & 100,690 & 13.57\% \\
    Dset\_355 & 355   & 95,940 & 11,467 & 84,473 & 11.95\% \\
    Dset\_186 & 186   & 36,219 & 5,517 & 30,702 & 15.23\% \\
    Dset\_72 & 72    & 18,140 & 1,923 & 16,217 & 10.60\% \\
    Dset\_164 & 164   & 33,681 & 6,096 & 27,585 & 18.10\% \\
    Training + validation & 9,982 & 4,254,198 & 427,687 & 3,826,511 & 10.05\% \\
    \hline
    \end{tabular}%
    \label{tab_dataset}%
\end{table*}%

All testing datasets are nearly disjoint as shown in Table \ref{tab:addlabel}.
\begin{table}[htbp]
  \centering
  \caption[Testing datasets analysis]{Testing datasets analysis. The number of identical sequences between each two datasets.}
    \begin{tabular}{|p{4.215em}|r|r|r|r|}
    \toprule
    \multicolumn{1}{|r|}{} & \multicolumn{1}{p{4.215em}|}{Dset\_164} & \multicolumn{1}{p{4.215em}|}{Dset\_186} & \multicolumn{1}{p{4.215em}|}{Dset\_448} & \multicolumn{1}{p{4.215em}|}{Dset\_355} \\
    \midrule
    Dset\_72 & 0     & 0     & 1     & 1 \\
    \midrule
    Dset\_164 &       & 2     & 5     & 5 \\
    \midrule
    Dset\_186 &       &       & 0     & 0 \\
    \midrule
    Dset\_448 &       &       &       & 355 \\
    \bottomrule
    \end{tabular}%
  \label{tab:addlabel}%
\end{table}%
\subsection{Evaluation Scheme \label{sec_evaluation_scheme}}
Similar to previous studies, we use sensitivity, specificity, precision, accuracy, F1-measure, (F1), Matthews correlation coefficient (MCC), area under the receiver operating characteristic curve (AUROC), and area under the precision-recall (AUPRC) to measure the prediction performance. All programs output a prediction value for each amino acid, and thus the receiver operating characteristic (ROC) curve and the precision-recall (PR) curve can be drawn. AUROC and AUPRC are computed based on the curves using Scikit-learn~\cite{scikit-learn}. We focus more on AUROC and AUPRC because they are threshold independent and convey an overall performance measurement of a program. The rest of the metrics are calculated using a binding threshold which is determined after obtaining the prediction scores from each program. Since each program's output is of different scale, for each program, we pick the threshold such that for a given testing dataset, the number of predicted scores above the threshold is equal to the real number of binding sites in the dataset. 

The formulas for calculating the metrics are as follows, where true positives ($TP$) and true negatives ($TN$) are the correctly predicted binding sites and non-binding sites, respectively, and false positives ($FP$) and false negative ($FN$) are incorrectly predicted binding sites and non-binding sites, respectively.
\begin{equation}
\textit{Sensitivity} = \frac{TP}{TP+FN}
\end{equation}

\begin{equation}
\textit{Specificity} = \frac{TN}{TN+FP}
\end{equation}

\begin{equation}
\textit{Precision} = \frac{TP}{TP + FP} 
\end{equation}

\begin{equation}
\textit{Accuracy}=\frac{TP+TN}{TP+FN+TN+FP}
\end{equation}

\begin{equation}
F1=2\times \frac{\textit{Sensitivity}\times \textit{Precision}}{\textit{Sensitivity}+\textit{Precision}}
\end{equation}

\begin{equation}
MCC\!=\!\frac{TP \times TN - FN \times FP}{\sqrt{(TP\!+\!FP)\!\times\! (TP\!+\!FN)\! \times \!(TN\!+\!FP)\!\times\!(TN\!+\!FN)}}
\end{equation}
\subsection{Performance Comparison on Dset\_448 and Dset\_355}
We first compare the DELPHI model with eight programs on Dset\_448. This dataset is the largest and the most recently published. As shown in Table \ref{tab_comp_448_355}, DELPHI surpasses competitors in all metrics with an improvement of 17.4\%, 18.3\%, and 3.08\% on AUPRC, MCC, and AUROC respectively comparing to the second best program SCRIBER.

As mentioned earlier, Dset\_448 cannot be used for DLPred, so we include a comparison of all programs on Dset\_355. As shown in Table \ref{tab_comp_448_355}, the performance of DLPred is very similar to the second best predictor, SCRIBER.  DELPHI still surpasses the second best program by 18.5\% and 20.9\% on AUPRC and MCC.

\begin{table}
  \centering
  \caption[Performance Comparison on Dset\_448 and Dset\_355]{Performance comparison on Dset\_448 and Dset\_355. Programs are sorted in ascending order by AUPRC. Darker colours represent better results. The evaluation of the programs marked with ${}^*$ is by  \cite{zhang2019scriber}.}
    % Table generated by Excel2LaTeX from sheet 'Sheet2'
    \begin{tabular}{@{}l@{\ }*{8}{r}}
    \toprule
    \multicolumn{1}{@{}l}{Predictor} & \multicolumn{1}{c}{$\!$Sens.} & \multicolumn{1}{c}{Spec.} & \multicolumn{1}{c}{Prec.} & \multicolumn{1}{c}{Acc.} & \multicolumn{1}{c}{F1} & \multicolumn{1}{c}{MCC} & \multicolumn{1}{c}{$\!\!\!$AUROC$\!\!\!$} & \multicolumn{1}{c@{}}{$\!$AUPRC} \\
    \hline
    \multicolumn{9}{c}{Dset\_448} \\
    \hline
   SPPIDER* & \cellcolor[rgb]{ .933,  .953,  .922}0.202 & 0.870 & \cellcolor[rgb]{ .961,  .973,  .957}0.194 & 0.781 & \cellcolor[rgb]{ .949,  .965,  .937}0.198 & \cellcolor[rgb]{ .957,  .969,  .949}0.071 & 0.517 & 0.159 \\
    SPRINT* & 0.183 & \cellcolor[rgb]{ .937,  .953,  .925}0.873 & 0.183 & 0.781 & 0.183 & 0.057 & \cellcolor[rgb]{ .839,  .882,  .812}0.570 & \cellcolor[rgb]{ .973,  .98,  .965}0.167 \\
    PSIVER* & \cellcolor[rgb]{ .973,  .98,  .969}0.191 & \cellcolor[rgb]{ .918,  .937,  .902}0.874 & \cellcolor[rgb]{ .973,  .98,  .969}0.191 & \cellcolor[rgb]{ .973,  .98,  .969}0.783 & \cellcolor[rgb]{ .973,  .98,  .969}0.191 & \cellcolor[rgb]{ .973,  .98,  .969}0.066 & \cellcolor[rgb]{ .808,  .859,  .773}0.581 & \cellcolor[rgb]{ .961,  .973,  .953}0.170 \\
    SPRINGS* & \cellcolor[rgb]{ .839,  .882,  .808}0.229 & \cellcolor[rgb]{ .745,  .812,  .698}0.882 & \cellcolor[rgb]{ .843,  .886,  .812}0.228 & \cellcolor[rgb]{ .792,  .847,  .753}0.796 & \cellcolor[rgb]{ .839,  .882,  .808}0.229 & \cellcolor[rgb]{ .835,  .878,  .804}0.111 & \cellcolor[rgb]{ .675,  .761,  .612}0.625 & \cellcolor[rgb]{ .843,  .886,  .816}0.201 \\
    LORIS* & \cellcolor[rgb]{ .714,  .792,  .659}0.264 & \cellcolor[rgb]{ .635,  .733,  .569}0.887 & \cellcolor[rgb]{ .718,  .792,  .663}0.263 & \cellcolor[rgb]{ .667,  .757,  .604}0.805 & \cellcolor[rgb]{ .718,  .792,  .663}0.263 & \cellcolor[rgb]{ .71,  .788,  .655}0.151 & \cellcolor[rgb]{ .58,  .694,  .502}0.656 & \cellcolor[rgb]{ .741,  .812,  .694}0.228 \\
    CRFPPI* & \cellcolor[rgb]{ .698,  .78,  .643}0.268 & \cellcolor[rgb]{ .635,  .733,  .569}0.887 & \cellcolor[rgb]{ .714,  .792,  .659}0.264 & \cellcolor[rgb]{ .667,  .757,  .604}0.805 & \cellcolor[rgb]{ .706,  .784,  .651}0.266 & \cellcolor[rgb]{ .698,  .78,  .643}0.154 & \cellcolor[rgb]{ .502,  .635,  .412}0.681 & \cellcolor[rgb]{ .706,  .784,  .651}0.238 \\
    SSWRF* & \cellcolor[rgb]{ .627,  .729,  .561}0.288 & \cellcolor[rgb]{ .549,  .671,  .471}0.891 & \cellcolor[rgb]{ .635,  .733,  .569}0.286 & \cellcolor[rgb]{ .584,  .694,  .506}0.811 & \cellcolor[rgb]{ .631,  .729,  .565}0.287 & \cellcolor[rgb]{ .624,  .725,  .557}0.178 & \cellcolor[rgb]{ .482,  .624,  .388}0.687 & \cellcolor[rgb]{ .639,  .737,  .573}0.256 \\
    SCRIBER & \cellcolor[rgb]{ .463,  .608,  .365}0.334 & \cellcolor[rgb]{ .443,  .592,  .341}0.896 & \cellcolor[rgb]{ .471,  .612,  .373}0.332 & \cellcolor[rgb]{ .443,  .592,  .341}0.821 & \cellcolor[rgb]{ .467,  .612,  .369}0.333 & \cellcolor[rgb]{ .463,  .608,  .365}0.230 & \cellcolor[rgb]{ .4,  .561,  .29}0.715 & \cellcolor[rgb]{ .522,  .651,  .435}0.287 \\
    DELPHI & \cellcolor[rgb]{ .329,  .51,  .208}0.371 & \cellcolor[rgb]{ .329,  .51,  .208}0.901 & \cellcolor[rgb]{ .329,  .51,  .208}0.371 & \cellcolor[rgb]{ .329,  .51,  .208}0.829 & \cellcolor[rgb]{ .329,  .51,  .208}0.371 & \cellcolor[rgb]{ .329,  .51,  .208}0.272 & \cellcolor[rgb]{ .329,  .51,  .208}0.737 & \cellcolor[rgb]{ .329,  .51,  .208}0.337 \\
    \hline
    \multicolumn{9}{c}{Dset\_355} \\
    \hline
    SPPIDER & \cellcolor[rgb]{ .961,  .973,  .957}0.180 & \cellcolor[rgb]{ .949,  .965,  .941}0.889 & \cellcolor[rgb]{ .961,  .973,  .953}0.180 & \cellcolor[rgb]{ .957,  .969,  .949}0.804 & \cellcolor[rgb]{ .961,  .973,  .953}0.180 & \cellcolor[rgb]{ .961,  .973,  .953}0.068 & 0.515 & 0.138 \\
    SPRINT & 0.168 & 0.886 & 0.167 & 0.801 & 0.168 & 0.054 & \cellcolor[rgb]{ .839,  .882,  .808}0.571 & \cellcolor[rgb]{ .961,  .973,  .953}0.150 \\
    PSIVER & \cellcolor[rgb]{ .969,  .976,  .965}0.178 & \cellcolor[rgb]{ .965,  .976,  .957}0.888 & \cellcolor[rgb]{ .969,  .976,  .965}0.177 & \cellcolor[rgb]{ .969,  .976,  .961}0.803 & \cellcolor[rgb]{ .969,  .976,  .965}0.177 & \cellcolor[rgb]{ .969,  .976,  .965}0.065 & \cellcolor[rgb]{ .804,  .859,  .769}0.583 & \cellcolor[rgb]{ .941,  .957,  .929}0.155 \\
    SPRINGS & \cellcolor[rgb]{ .855,  .894,  .827}0.211 & \cellcolor[rgb]{ .867,  .906,  .843}0.892 & \cellcolor[rgb]{ .859,  .898,  .831}0.210 & \cellcolor[rgb]{ .863,  .898,  .835}0.811 & \cellcolor[rgb]{ .855,  .894,  .827}0.211 & \cellcolor[rgb]{ .855,  .894,  .831}0.103 & \cellcolor[rgb]{ .733,  .804,  .682}0.608 & \cellcolor[rgb]{ .859,  .898,  .831}0.178 \\
    LORIS & \cellcolor[rgb]{ .749,  .816,  .706}0.242 & \cellcolor[rgb]{ .769,  .831,  .725}0.896 & \cellcolor[rgb]{ .757,  .82,  .71}0.240 & \cellcolor[rgb]{ .761,  .824,  .714}0.818 & \cellcolor[rgb]{ .753,  .82,  .706}0.241 & \cellcolor[rgb]{ .753,  .82,  .71}0.137 & \cellcolor[rgb]{ .647,  .745,  .584}0.637 & \cellcolor[rgb]{ .773,  .835,  .729}0.203 \\
    CRFPPI & \cellcolor[rgb]{ .733,  .804,  .686}0.247 & \cellcolor[rgb]{ .741,  .812,  .698}0.897 & \cellcolor[rgb]{ .737,  .808,  .69}0.245 & \cellcolor[rgb]{ .737,  .808,  .69}0.819 & \cellcolor[rgb]{ .733,  .808,  .686}0.246 & \cellcolor[rgb]{ .737,  .808,  .686}0.143 & \cellcolor[rgb]{ .573,  .69,  .498}0.662 & \cellcolor[rgb]{ .733,  .804,  .682}0.214 \\
    SSWRF & \cellcolor[rgb]{ .663,  .753,  .6}0.268 & \cellcolor[rgb]{ .655,  .749,  .592}0.901 & \cellcolor[rgb]{ .659,  .753,  .596}0.268 & \cellcolor[rgb]{ .659,  .749,  .596}0.825 & \cellcolor[rgb]{ .659,  .753,  .6}0.268 & \cellcolor[rgb]{ .659,  .753,  .596}0.168 & \cellcolor[rgb]{ .561,  .678,  .482}0.667 & \cellcolor[rgb]{ .682,  .769,  .624}0.228 \\
    DLPred & \cellcolor[rgb]{ .522,  .651,  .435}0.308 & \cellcolor[rgb]{ .518,  .647,  .431}0.906 & \cellcolor[rgb]{ .522,  .651,  .435}0.308 & \cellcolor[rgb]{ .514,  .647,  .427}0.835 & \cellcolor[rgb]{ .522,  .651,  .435}0.308 & \cellcolor[rgb]{ .522,  .651,  .435}0.214 & \cellcolor[rgb]{ .396,  .561,  .286}0.724 & \cellcolor[rgb]{ .525,  .655,  .439}0.272 \\
    SCRIBER & \cellcolor[rgb]{ .475,  .616,  .38}0.322 & \cellcolor[rgb]{ .475,  .616,  .38}0.908 & \cellcolor[rgb]{ .475,  .616,  .38}0.322 & \cellcolor[rgb]{ .475,  .616,  .376}0.838 & \cellcolor[rgb]{ .475,  .616,  .38}0.322 & \cellcolor[rgb]{ .475,  .616,  .38}0.230 & \cellcolor[rgb]{ .408,  .569,  .302}0.719 & \cellcolor[rgb]{ .514,  .643,  .424}0.275 \\
    DELPHI & \cellcolor[rgb]{ .329,  .51,  .208}0.364 & \cellcolor[rgb]{ .329,  .51,  .208}0.914 & \cellcolor[rgb]{ .329,  .51,  .208}0.364 & \cellcolor[rgb]{ .329,  .51,  .208}0.848 & \cellcolor[rgb]{ .329,  .51,  .208}0.364 & \cellcolor[rgb]{ .329,  .51,  .208}0.278 & \cellcolor[rgb]{ .329,  .51,  .208}0.746 & \cellcolor[rgb]{ .329,  .51,  .208}0.326 \\
    \end{tabular}%
  \label{tab_comp_448_355}%
\end{table}%

\subsection{Performance Comparison on Dset\_186, Dset\_164, and Dset\_72}
To further compare DELPHI with other programs, we used another three previously published datasets: Dset\_186, Dset\_164, and Dset\_72. Based on the availability and usability, We ran SPPIDER, PSIVER, CRFPPI, SCRIBER, DELPred, and DELPHI on them. Note that SSWRF, CRFPPI, and PSIVER use Dset\_186 as their training datasets, so these three programs are excluded on Dset\_186. As shown in Table \ref{tab_ds186_164_72}, in general, the performance rank is the very similar to the ones in Dset\_448 and Dset\_355.

DELPHI clearly outperforms the competitors in all metrics on all datasets although it shares the least similarities to the testing datasets.  The AUPRC is improved by 10.0\%, 0.6\%, 10.2\% comparing to the second best program on each dataset. The improvement on MCC are 18.7\%, 8.9\%, 27.7\% on each dataset. 
\begin{table}
  \centering
  \caption{Performance comparison on Dset\_186, Dset\_164, and Dset\_72 using the same metrics. Darker colours represent better results.}
    \begin{tabular}{@{}l@{\ }*{8}{r}}
    \toprule
    \multicolumn{1}{@{}l}{Predictor} & \multicolumn{1}{c}{$\!$Sens.} & \multicolumn{1}{c}{Spec.} & \multicolumn{1}{c}{Prec.} & \multicolumn{1}{c}{Acc.} & \multicolumn{1}{c}{F1} & \multicolumn{1}{c}{MCC} & \multicolumn{1}{c}{$\!\!\!$AUROC$\!\!\!$} & \multicolumn{1}{c@{}}{$\!$AUPRC} \\
    \hline
    \multicolumn{9}{c}{Dset\_186} \\
    \hline
    SPPIDER & \cellcolor[rgb]{ .988,  .988,  1}0.194 & \cellcolor[rgb]{ .988,  .988,  1}0.848 & \cellcolor[rgb]{ .988,  .988,  1}0.186 & \cellcolor[rgb]{ .988,  .988,  1}0.748 & \cellcolor[rgb]{ .988,  .988,  1}0.190 & \cellcolor[rgb]{ .988,  .988,  1}0.041 & \cellcolor[rgb]{ .988,  .988,  1}0.499 & \cellcolor[rgb]{ .988,  .988,  1}0.165 \\
    SCRIBER & \cellcolor[rgb]{ .631,  .729,  .573}0.279 & \cellcolor[rgb]{ .576,  .69,  .506}0.870 & \cellcolor[rgb]{ .62,  .722,  .557}0.279 & \cellcolor[rgb]{ .604,  .71,  .537}0.780 & \cellcolor[rgb]{ .624,  .725,  .565}0.279 & \cellcolor[rgb]{ .62,  .722,  .557}0.150 & \cellcolor[rgb]{ .529,  .655,  .447}0.647 & \cellcolor[rgb]{ .643,  .741,  .588}0.246 \\
    DLPred & \cellcolor[rgb]{ .459,  .604,  .365}0.320 & \cellcolor[rgb]{ .439,  .592,  .341}0.878 & \cellcolor[rgb]{ .455,  .6,  .357}0.320 & \cellcolor[rgb]{ .451,  .6,  .357}0.793 & \cellcolor[rgb]{ .459,  .604,  .361}0.320 & \cellcolor[rgb]{ .455,  .604,  .361}0.198 & \cellcolor[rgb]{ .38,  .549,  .267}0.694 & \cellcolor[rgb]{ .459,  .604,  .361}0.290 \\
    DELPHI & \cellcolor[rgb]{ .329,  .51,  .208}0.351 & \cellcolor[rgb]{ .329,  .51,  .208}0.884 & \cellcolor[rgb]{ .329,  .51,  .208}0.351 & \cellcolor[rgb]{ .329,  .51,  .208}0.803 & \cellcolor[rgb]{ .329,  .51,  .208}0.351 & \cellcolor[rgb]{ .329,  .51,  .208}0.235 & \cellcolor[rgb]{ .329,  .51,  .208}0.710 & \cellcolor[rgb]{ .329,  .51,  .208}0.319 \\
    \hline
    \multicolumn{9}{c}{Dset\_164} \\
    \hline
    SPPIDER & \cellcolor[rgb]{ .761,  .824,  .729}0.264 & \cellcolor[rgb]{ .949,  .961,  .953}0.828 & \cellcolor[rgb]{ .812,  .859,  .788}0.253 & \cellcolor[rgb]{ .859,  .894,  .843}0.726 & \cellcolor[rgb]{ .788,  .843,  .757}0.258 & \cellcolor[rgb]{ .804,  .855,  .776}0.090 & \cellcolor[rgb]{ .988,  .988,  1}0.528 & \cellcolor[rgb]{ .914,  .933,  .91}0.220 \\
    PSIVER & \cellcolor[rgb]{ .988,  .988,  1}0.217 & \cellcolor[rgb]{ .988,  .988,  1}0.826 & \cellcolor[rgb]{ .988,  .988,  1}0.216 & \cellcolor[rgb]{ .988,  .988,  1}0.716 & \cellcolor[rgb]{ .988,  .988,  1}0.216 & \cellcolor[rgb]{ .988,  .988,  1}0.043 & \cellcolor[rgb]{ .878,  .91,  .867}0.554 & \cellcolor[rgb]{ .988,  .988,  1}0.205 \\
    SSWRF & \cellcolor[rgb]{ .753,  .816,  .714}0.266 & \cellcolor[rgb]{ .741,  .808,  .702}0.838 & \cellcolor[rgb]{ .749,  .816,  .714}0.266 & \cellcolor[rgb]{ .745,  .812,  .71}0.734 & \cellcolor[rgb]{ .749,  .816,  .714}0.266 & \cellcolor[rgb]{ .749,  .816,  .714}0.103 & \cellcolor[rgb]{ .663,  .753,  .608}0.606 & \cellcolor[rgb]{ .796,  .847,  .769}0.243 \\
    CRFPPI & \cellcolor[rgb]{ .682,  .769,  .631}0.280 & \cellcolor[rgb]{ .675,  .761,  .62}0.841 & \cellcolor[rgb]{ .682,  .765,  .631}0.280 & \cellcolor[rgb]{ .678,  .765,  .627}0.739 & \cellcolor[rgb]{ .682,  .765,  .631}0.280 & \cellcolor[rgb]{ .682,  .765,  .631}0.121 & \cellcolor[rgb]{ .655,  .745,  .596}0.608 & \cellcolor[rgb]{ .667,  .757,  .612}0.267 \\
    SCRIBER & \cellcolor[rgb]{ .451,  .6,  .353}0.327 & \cellcolor[rgb]{ .451,  .596,  .353}0.851 & \cellcolor[rgb]{ .451,  .6,  .353}0.327 & \cellcolor[rgb]{ .451,  .6,  .353}0.756 & \cellcolor[rgb]{ .451,  .6,  .353}0.327 & \cellcolor[rgb]{ .451,  .6,  .353}0.179 & \cellcolor[rgb]{ .451,  .596,  .353}0.657 & \cellcolor[rgb]{ .49,  .627,  .4}0.301 \\
    DLPred & \cellcolor[rgb]{ .4,  .561,  .294}0.338 & \cellcolor[rgb]{ .4,  .561,  .29}0.854 & \cellcolor[rgb]{ .4,  .561,  .29}0.338 & \cellcolor[rgb]{ .4,  .561,  .29}0.760 & \cellcolor[rgb]{ .4,  .561,  .294}0.338 & \cellcolor[rgb]{ .4,  .561,  .29}0.192 & \cellcolor[rgb]{ .384,  .549,  .275}0.672 & \cellcolor[rgb]{ .341,  .518,  .22}0.330 \\
    DELPHI & \cellcolor[rgb]{ .329,  .51,  .208}0.352 & \cellcolor[rgb]{ .329,  .51,  .208}0.857 & \cellcolor[rgb]{ .329,  .51,  .208}0.352 & \cellcolor[rgb]{ .329,  .51,  .208}0.765 & \cellcolor[rgb]{ .329,  .51,  .208}0.352 & \cellcolor[rgb]{ .329,  .51,  .208}0.209 & \cellcolor[rgb]{ .329,  .51,  .208}0.685 & \cellcolor[rgb]{ .329,  .51,  .208}0.332 \\
    \hline
    \multicolumn{9}{c}{Dset\_72} \\
    \hline
    SPPIDER & \cellcolor[rgb]{ .8,  .851,  .773}0.188 & \cellcolor[rgb]{ .988,  .988,  1}0.898 & \cellcolor[rgb]{ .843,  .882,  .827}0.179 & \cellcolor[rgb]{ .925,  .945,  .925}0.823 & \cellcolor[rgb]{ .824,  .867,  .8}0.183 & \cellcolor[rgb]{ .835,  .878,  .816}0.084 & \cellcolor[rgb]{ .988,  .988,  1}0.522 & \cellcolor[rgb]{ .988,  .988,  1}0.134 \\
    PSIVER & \cellcolor[rgb]{ .988,  .988,  1}0.152 & \cellcolor[rgb]{ .937,  .953,  .937}0.899 & \cellcolor[rgb]{ .988,  .988,  1}0.152 & \cellcolor[rgb]{ .988,  .988,  1}0.820 & \cellcolor[rgb]{ .988,  .988,  1}0.152 & \cellcolor[rgb]{ .988,  .988,  1}0.052 & \cellcolor[rgb]{ .706,  .784,  .659}0.604 & \cellcolor[rgb]{ .941,  .957,  .945}0.141 \\
    CRFPPI & \cellcolor[rgb]{ .475,  .616,  .384}0.248 & \cellcolor[rgb]{ .478,  .62,  .388}0.911 & \cellcolor[rgb]{ .475,  .616,  .384}0.248 & \cellcolor[rgb]{ .49,  .627,  .4}0.840 & \cellcolor[rgb]{ .475,  .616,  .384}0.248 & \cellcolor[rgb]{ .478,  .616,  .384}0.158 & \cellcolor[rgb]{ .475,  .616,  .384}0.669 & \cellcolor[rgb]{ .565,  .682,  .49}0.200 \\
    SSWRF & \cellcolor[rgb]{ .482,  .624,  .392}0.246 & \cellcolor[rgb]{ .486,  .624,  .396}0.911 & \cellcolor[rgb]{ .482,  .624,  .392}0.246 & \cellcolor[rgb]{ .498,  .635,  .412}0.840 & \cellcolor[rgb]{ .482,  .624,  .392}0.246 & \cellcolor[rgb]{ .486,  .624,  .396}0.157 & \cellcolor[rgb]{ .447,  .596,  .349}0.678 & \cellcolor[rgb]{ .576,  .69,  .506}0.198 \\
    SCRIBER & \cellcolor[rgb]{ .557,  .675,  .482}0.232 & \cellcolor[rgb]{ .553,  .671,  .475}0.909 & \cellcolor[rgb]{ .557,  .675,  .482}0.232 & \cellcolor[rgb]{ .569,  .686,  .498}0.837 & \cellcolor[rgb]{ .557,  .675,  .482}0.232 & \cellcolor[rgb]{ .557,  .678,  .482}0.141 & \cellcolor[rgb]{ .439,  .592,  .341}0.680 & \cellcolor[rgb]{ .576,  .69,  .506}0.198 \\
    DLPred & \cellcolor[rgb]{ .482,  .62,  .392}0.246 & \cellcolor[rgb]{ .859,  .894,  .843}0.901 & \cellcolor[rgb]{ .482,  .62,  .388}0.246 & \cellcolor[rgb]{ .855,  .894,  .839}0.826 & \cellcolor[rgb]{ .482,  .62,  .388}0.246 & \cellcolor[rgb]{ .529,  .655,  .447}0.148 & \cellcolor[rgb]{ .412,  .569,  .306}0.688 & \cellcolor[rgb]{ .467,  .612,  .373}0.215 \\
    DELPHI & \cellcolor[rgb]{ .329,  .51,  .208}0.274 & \cellcolor[rgb]{ .329,  .51,  .208}0.914 & \cellcolor[rgb]{ .329,  .51,  .208}0.274 & \cellcolor[rgb]{ .329,  .51,  .208}0.847 & \cellcolor[rgb]{ .329,  .51,  .208}0.274 & \cellcolor[rgb]{ .329,  .51,  .208}0.189 & \cellcolor[rgb]{ .329,  .51,  .208}0.711 & \cellcolor[rgb]{ .329,  .51,  .208}0.237 \\
\label{tab_ds186_164_72}
    \end{tabular}%
\end{table}%

\subsection{Ablation Study}
\subsubsection{Feature Evaluation}
We conducted an another experiment to show that all twelve feature are necessary for DELPHI. We pruned one feature each time, and the remaining eleven features are used to train and then evaluate the DELPHI model. As shown in Fig. \ref{fig_remove_each_feature}, the performance decreases with the removal of any feature, showing that there are no redundant features. It is perhaps expected that  removing PSSM creates the biggest performance drop, but our newly introduced features, HSP, ProtVec1D, and Position are shown to be  very useful as well. 
\begin{figure}[!h]
\begin{center}
\includegraphics[height = 7cm, width = 9cm]{img/remove_features_individually_Testing.pdf}
\caption[The areas under PR curves with the removal of one out of the twelve features on Dset\_448]{The areas under PR curves with the removal of one out of the twelve features on Dset\_448. One feature is removed each time, and the DELPHI model is trained, validated, and tested using the remaining eleven features. The x-axis shows the removed features where 'None' indicates using all twelve features, and the y-axis is the AUPRC achieved. The features are sorted by the AUPRC values. } \label{fig_remove_each_feature}
\end{center}
\end{figure}

\subsubsection{The Evaluation of the Model Architecture and the Novel Features\label{sec_arch_fea}}
To show that the ensemble architecture and the three novel features improve the performance, we evaluated the CNN, RNN, and the ensemble model separately on Dset\_448 with and without the three new features, a total of six tests. Training is done as before (section \ref{section_ensemble}). In Fig. \ref{fig_CNN_RNN_ensemble}, we plotted the value of AUPRC and MCC of the six tests. Clearly, the ensemble model outperforms the individual CNN and RNN models. Further, the improvement due to the three new features is even higher, with the weakest model, CNN, on 12 features outperforming the ensemble on 9 features. 

\begin{figure}
\centering
\includegraphics[width=\columnwidth, height=7cm]{img/CNN_RNN_ensemble.pdf}
  \caption[The evaluation of the DELPHI model architecture and the three novel features.]{The evaluation of the DELPHI model architecture and the three novel features. The area under PR curves (left) and MCC (right) are plotted separately. Each plot contains the performance of using CNN, RNN, and the ensemble model on Dset\_448. Two different colors indicate with and without the three new features.}
  \label{fig_CNN_RNN_ensemble}
\end{figure}

\subsection{\textcolor{\mySecondColor}{Evolutionary Conservation}}

\textcolor{\mySecondColor}{As an application of DELPHI's predictive power we analyzed three different proteins.  These are the transcription factor SH2D2A (399 amino acids in length), the alpha-subunit of the haemoglobin protein (142 amino acids in length) and the SRY protein (204 amino acids
in length).}

\textcolor{\mySecondColor}{For each protein, BLASTP was used to search for homologues.  The search was restricted to the refseq\_protein database to ensure good quality and
to proteins from mammalian organisms.  The SH2D2A sequences were further
limited to isoform X1 and the haemoglobins sequences were restricted to
proteins labelled as alpha subunits.  Sequences were aligned with MUSCLE
\cite{edgar2004muscle} and any sequences with unusually long protein distances
were further eliminated by hand.}

\textcolor{\mySecondColor}{This resulted in 66 homologous SH2D2A sequences, 178 homologous alpha
subunit haemoglobin sequences and 40 homolgous SRY sequences.  For each
site in the alignment of these proteins, the frequency of the most
conserved amino acid was recorded and compared to the PPI binding sites predicted by
DELPHI in Figure~\ref{fig_conservation_vs_sitePred}.  There are gaps in these figures since only aligned
sites present in more than 10 taxa were included in this comparison.}

\textcolor{\mySecondColor}{In general, Figure~\ref{fig_conservation_vs_sitePred}.a and Figure~\ref{fig_conservation_vs_sitePred}.c shows that the locations where
DELPHI predicted a high probability of a protein-protein interaction are
also the sites with a high degree of sequence conservation.  As expected
this correlation is not perfect since sequence conservation can occur
for many reasons other than PPI.  It will also be noted that the overall
degree of conservation in Figure~\ref{fig_conservation_vs_sitePred}.a for the alpha haemoglobin proteins is
much more conserved than for the other two proteins and is an indication
that these proteins are evolving slower.  Still PPI has a high probability
around 130aa, 185aa and 240aa but has a lower probability at the sites
around 175aa.  Similarly for sites around 350aa and 500aa in Figure~\ref{fig_conservation_vs_sitePred}.c
(protein SH2D2A) the PPI is low but the level of conservation is high.}

\textcolor{\mySecondColor}{The protein SRY in Figure~\ref{fig_conservation_vs_sitePred}.b binds to sites in the DNA rather
than to other proteins or internally.  This is the reason for the
high conservation around 190aa to 260aa.  Except for two sites the
probabilities predicted by DELPHI fluctuate around low values.}

\textcolor{\mySecondColor}{In general the data in Figure~\ref{fig_conservation_vs_sitePred} show that sites with a high probability
of PPI have a high degree of sequence conservation and indicate good
support for the validity of the method.  The opposite is not true.
Sequence conservation can occur for reasons other than the requirement
that the sites are constrained by protein-protein interactions.}

\begin{sidewaysfigure}
\centering
\begin{tabular}{cc}
\includegraphics[height=3.5cm]{img/hemoglobin_top178_conservation.pdf} & 
\includegraphics[height=3.5cm]{img/SRY_conservation.pdf}\\
\small (a) & \small (b)\\
\multicolumn{2}{c}{\includegraphics[height=3.5cm]{img/SH2_conservation.pdf}}\\
\multicolumn{2}{c}{\small (c)}
\end{tabular}
  \caption[Comparison between DELPHI predictions and evolutionary conservation]{Three proteins were evaluated to compare the PPI binding sites predicted by DELPHI (orange) with the degree of site-by-site conservation (blue).  Only sites represented in ten or more taxa are included resulting in some apparent gaps.  The proteins are (a) alpha haemoglobin, (b) SRY and (c) SH2D2A.} 
  \label{fig_conservation_vs_sitePred}
\end{sidewaysfigure}


\begin{figure}
\centering
\includegraphics[width=\columnwidth]{img/plot_DS448_domain.pdf}
  \caption[A comparison of the predicted PBRs from DELPHI and from SCRIBER compared to native PBRs.]{A comparison of the predicted PBRs from DELPHI and from SCRIBER compared to native PBRs. It is apparent that DELPHI fits the curve better except perhaps at the low end where there are few PBR residues per domain.} 
  \label{fig_plot_DS448_domain}
\end{figure}


\subsection{\textcolor{\mySecondColor}{Accuracy of PBR Prediction}}
% I don't think that there is sufficient information here to warrant a separate section.
\textcolor{\mySecondColor}{Following Zhang and Kurgan (2019) we compared the protein-binding residues (PBRs) predicted by SCRIBER and by DELPHI on DSet\_448.  
A total of 600 domains with native PBRs were collected from the Pfam (El-Gebali et al., 2019) annotation of proteins by Zhang and Kurgan (2019).  
Figure \ref{fig_plot_DS448_domain} shows the number of PBRs in these domains and compares them to the numbers predicted by SCRIBER and DELPHI. 
It is apparent that DELPHI has a closer fit to the native data and this is especially 
true with larger numbers of PBRs per domain but DELPHI perhaps over estimates the percentage of domains with PBRs when there are few PBRs per domain.}

\subsection{\textcolor{\mySecondColor}{Human Proteome Prediction}}
\textcolor{\mySecondColor}{
To further assist users, we ran DELPHI on the entire human proteome. The sequences are downloaded from Uniprot in June 2020. All the prediction results are available to download at \texttt{www.csd.uwo.ca/\textasciitilde{}yli922/index.php}.
}

\subsection{Availability}
DELPHI is available both as a open sourced standalone software under the GPLv3 License and web server. The trained model, source code, and data processing pipeline are freely available at \texttt{github.com/lucian-ilie/DELPHI}.\\The web server is at \texttt{www.csd.uwo.ca/\textasciitilde{}yli922/index.php}.
\section{Conclusion}
We introduced our sequence based PPI site prediction program, DELPHI. DELPHI has an ensemble structure which combines a CNN and a RNN component with fine tuning technique. Three novel features, HSP, position information, and ProtVec are used in addition to nine existing ones. We comprehensively compare DELPHI to nine state-of-the-art programs on five datasets, and DELPHI outperforms the competing methods in all metrics even though its training dataset shares the least similarities with the testing datasets. In the most important metrics, AUPRC and MCC, it surpasses the second best programs by as much as 18.5\% and 27.7\%, resp. We also demonstrated that the improvement is essentially due to using the ensemble model and, especially, the three new features. Using DELPHI it is shown that there is a strong correlation with protein-binding residues (PBRs) and sites with strong evolutionary conservation.  In addition DELPHI's predicted PBR sites closely match known data from Pfam.
DELPHI is available as open sourced standalone software and web server.

The contributions of the DELPHI study are as follows. First, a novel fine tuned ensemble model combing CNN and RNN is constructed. Second, three novel features, which are believed to be used the first time in PPI binding site prediction, are introduced. Third, a data processing and feature construction suite, in the form of both source code and web server, is provided, aiming to alleviating the difficulty of tedious feature computation by the users.
\chapter{Protein-protein Interaction Prediction: SPRINT \label{chap_3}}
This chapter describes our two publications regarding SPRINT (Scoring PRotein INTeractions) \cite{li2017sprint, li2020predicting}, a sequence based protein-protein interaction site prediction program. We first introduce the prerequisites used in SPRINT as well as the state-of-the-art methods. Then we describe both the algorithm and the implementation of SPRINT in details.  Last, we compare DELPHI with five state-of-the-art programs on seven datasets showing that it is more accurate while running orders of magnitude faster and using very little memory.
\section{Background}
In addition to the deep learning prerequisites introduced in Section \ref{set_AI_in_bioinfor}, we explain some preliminary algorithmic concepts used in the SPRINT algorithm in this section.
\subsection{Homology Search}
Homology search is a task to detect identical or similar segments among sequences. It is a routinely performed task DNA or protein sequences. Many bioinformatics programs highly depend on homology search, and quite a few algorithms have been proposed for this task. 

The Smith-Waterman algorithm \cite{smith_waterman_1981} was proposed in the early 80's. This classic dynamic programming algorithm detects the optimal local alignment between two sequences. It guarantees to find the best solution, and that comes with the price of high time complexity. As shown in Figure \ref{fig_smith_waterman}, the Smith-Waterman algorithm works as two steps: Filling the scoring matrix and tracing back the optimal alignment. The value $S[i,j]$ in the position  $(i,j)$ is determined by three values around it: $S[i-1,j-1]$, $S[i-1,j]$, and $S[i,j-1]$. The detailed algorithm and analysis can be seen in \cite{smith_waterman_1981}. The worst time complexity is $O(mn)$ where $m$ and $n$ are the lengths of two sequences. This computationally expensive algorithm was soon overwhelmed by the amount of increasing biological data.

\begin{figure}[h!]
     \centering
     \begin{subfigure}[b]{0.45\textwidth}
         \centering
         \includegraphics[width=\textwidth]{img/Smith_waterman_fill_matrix.png}
        %  \caption{$y=3sinx$}
        %  \label{fig_three sin x}
     \end{subfigure}
     \hfill
     \begin{subfigure}[b]{0.45\textwidth}
         \centering
         \includegraphics[width=\textwidth]{img/Smith_waterman_backtra.png}
        %  \caption{$y=5/x$}
        %  \label{fig_five over x}
     \end{subfigure}
        \caption[The Smith-Waterman algorithm.]{The Smith-Waterman algorithm.}
        \label{fig_smith_waterman}
\end{figure}

\subsubsection{BLAST seeds}
Heuristic algorithms were proposed aiming to detecting similarities much faster and at the same time, detecting most of the similarities.

The most noticeable algorithm and its suite is the the BLAST (Basic Local Alignment Search Tool) family \cite{Altschul90_BLAST, altschul1997PSI_BLAST}. BLATS serves as part of the pipelines for thousands of programs. The two papers together have had 157,399 citations till May 2020. 

The key process in BLAST is called hit-and-extension. When identifying similarity, BLAST requires first short consecutive matches between two strings. Then the matched region extends to both the left and the right side to form a longer local alignment. This extension stops until the similarities of the detected region drops below a threshold. The default number for DNA sequences is eleven consecutive matches. Denoting the matches by 1's we obtain what is called a consecutive seed: 11111111111. A pair of identical consecutive matches is called a hit. The homology grows from the consecutive matches therefore they are called the seed. A seed often is used to refer to the way these eleven positions are selected.

Figure \ref{fig_BLAST} shows an example of the hit-and-extension algorithm. When a hit is detected by the consecutive seed, marked with red color, BLAST extends this hit towards both sides (blue) until certain amount of mismatches are detected. 
\\
\begin{figure}[h!]
\begin{center}
\setlength{\tabcolsep}{4.5pt}
\ttfamily{
\begin{tabular}{ccccccccccccccccccccccccccc}
\ & \ & \ & \ & \ & \ & \ & \ & \textbf{|} & $\leftarrow$ & \ & \ & \ & h & i & t & \ & \ & $\rightarrow$ & \textbf{|} & \ & \ & \ & \ & \ & \ & \\
T & T & G & A & C & \textcolor{blue}{A} & \textcolor{blue}{T} & \textcolor{blue}{G} & \textcolor{red}{T} & \textcolor{red}{C} & \textcolor{red}{T} & \textcolor{red}{C} & \textcolor{red}{T} & \textcolor{red}{A} & \textcolor{red}{G} & \textcolor{red}{T} & \textcolor{red}{G} & \textcolor{red}{A} & \textcolor{red}{G} &\textcolor{red}{ T} & \textcolor{blue}{C} & \textcolor{blue}{T} & \textcolor{blue}{C} & T & A & T &\\ 
| & | & \ & \ & \ & | & | & | & | & | & | & | & | & | & | & | & | & | & | & | & | & | & | & \ & \ & \ & \\ 
T & T & A & C & T & \textcolor{blue}{A} & \textcolor{blue}{T} & \textcolor{blue}{G} & \textcolor{red}{T} & \textcolor{red}{C} & \textcolor{red}{T} & \textcolor{red}{C} & \textcolor{red}{T} & \textcolor{red}{A} & \textcolor{red}{G} & \textcolor{red}{T} & \textcolor{red}{G} & \textcolor{red}{A} & \textcolor{red}{G} &\textcolor{red}{ T} & \textcolor{blue}{C} & \textcolor{blue}{T} & \textcolor{blue}{C} & A & T & G &\\ 
\ & \ & \ & \ & \ & \ & \ & \ & 1 & 1 & 1 & 1 & 1 & 1 & 1 & 1 & 1 & 1 & 1 & 1 & \ & \ &\  & \ & \ & \ & \\
\ & \ & \ & \ & \ & \textbf{|} & $\leftarrow$ & \tiny {\textbf{ext.}} & \ & \ & \ & \ & \ & \ & \ & \ & \ & \ & \ & \ & \tiny{\textbf{ext.}} & $\rightarrow$ &\textbf{|}  & \ & \ & \ & \\
\\
\end{tabular} \\
}
\caption[An example of the hit-and-extend algorithm in BLAST]{An example of the hit-and-extend algorithm in BLAST. A hit (red region) is identified first and then extended (blue region) to form a longer local alignment.  \label{fig_BLAST}}
\end{center}
\end{figure}

To reduce the time complexity, fast identification hits is usually implemented by indexing sequences in data structures like hash table, suffix array, or a suffix tree. Details of the SPRINT indexing implementation will be discussed in Section \ref{sec_detect_simi}. 

The sensitivity of a seed is defined by its probability of finding similarities. The length of the consecutive seed in BLAST determines the sensitivity and the speed of the program. In order to detecting more hits, i.e. higher sensitivity, shorter seeds are needed, but shorter seeds also means higher time complexity.

\subsubsection{Spaced seeds}
PatternHunter proposed \cite{Ma02_PatternHunter} the idea of spaced-seed. Denote the five consecutive matches of a BLAST seed by 11111; this is called a consecutive seed of weight five. Spaced  seeds consists of matches interspersed by "don’t care" positions. Matches positions are denoted by 1, and "don't care" positions are denoted by *, sometimes by 0. here is an example of such a spaced seed of weight 11 and length 18: 111*1**1*1**11*111. A spaced match requires only the letters in positions corresponding to 1’s in the seed to match. Figure \ref{fig_hit_spaced_seed} shows an example of a hit using the spaced-seed. The match positions requires identical characters in two sequences while the "don't care" positions do not.

\begin{figure}[h!]
\begin{center}
\includegraphics[height =3.5 cm]{img/spaced_seed_hit.png}
\caption[A hit of a spaced-seed]{A hit of a spaced-seed. 1's in the seed require a match in the two target sequences (blue), but the * positions do not care whether the corresponding positions in the two sequences match or not.  \label{fig_hit_spaced_seed}}
\end{center}
\end{figure}

A spaced-seed is more sensitive than a consecutive seed of the same weight while maintaining the same time complexity. For random sequences $S$ and $T$ with lengths $m$ and $n$, the expected number of hits for weight $w$, length $l$ seed is $(m-l+1)(n-l+1)4^{-w}$. Usually $l$ is much shorter than $m$ and $n$, this value is approximately $4^{-w}mn$. In other words, the expected number of hits in random regions only depends on the weight of the seed, but not the shape of the seed. Therefore, two seeds, one consecutive and one spaced, of the same weight have the same number of expected hits. That is the reason that time complexity does not change when adding "don't" care positions in the seed.

Spaced-seeds outperform consecutive seeds in sensitivity because they avoid hit clustering. As shown in Figure \ref{seeds}, after having a hit, the consecutive seed requires only one additional match when shifting one position to the right. This way, a long similarity region will be hit many times. In contrast, the spaced seed requires six matching characters when shifting one position. As stated above, the total number of hits using two seeds of the same weight is the same, spaced seeds have their hits much better distributed.  Because the hits of a spaced seed are less clustered together, more regions will be hit, thus increasing the probability of finding similarities. As shown in Fig. \ref{fig_seed_sens}, a spaced-seed of weight 11 is more sensitive than a BLAST seed with weight 11. It  is even higher than a BLAST seed with weight 10.
\begin{figure}[h!]
\begin{center}
\setlength{\tabcolsep}{2pt}
\ttfamily{
\begin{tabular}{ccccccccccccccccccccccccccccccccccccccc}
\textcolor{red}{C} & \textcolor{red}{G} & \textcolor{red}{T} & \textcolor{red}{C} & \textcolor{red}{A} & \textcolor{red}{A} & \textcolor{red}{G} & \textcolor{red}{A} & \textcolor{red}{C} & \textcolor{red}{T} & \textcolor{red}{T} & ? & \ & \ & \ & \ & \ &  \ & \ &\textcolor{red}{G} & \textcolor{red}{A} & \textcolor{red}{G} & ? & \textcolor{red}{C} & ? &? &  \textcolor{red}{T} & ? & \textcolor{red}{G} & ? & ? & \textcolor{red}{A} & \textcolor{red}{C} & ? & \textcolor{red}{T} & \textcolor{red}{T} & \textcolor{red}{C} & ? & \\
\textcolor{red}{|} & \textcolor{red}{|} & \textcolor{red}{|} & \textcolor{red}{|} & \textcolor{red}{|} & \textcolor{red}{|} & \textcolor{red}{|} & \textcolor{red}{|} & \textcolor{red}{|} & \textcolor{red}{|} & \textcolor{red}{|} & ? & \ & \ & \ &  \ & \ & \ & \ &\textcolor{red}{|} & \textcolor{red}{|} & \textcolor{red}{|} & ? & \textcolor{red}{|} & ? &? &  \textcolor{red}{|} & ? & \textcolor{red}{|} & ? & ? & \textcolor{red}{|} & \textcolor{red}{|} & ? & \textcolor{red}{|} & \textcolor{red}{|} & \textcolor{red}{|} & ? &\\
\textcolor{red}{C} & \textcolor{red}{G} & \textcolor{red}{T} & \textcolor{red}{C} & \textcolor{red}{A} & \textcolor{red}{A} & \textcolor{red}{G} & \textcolor{red}{A} & \textcolor{red}{C} & \textcolor{red}{T} & \textcolor{red}{T} & ? & \ & \ & \ & \ & \ & \ & \ & \textcolor{red}{G} & \textcolor{red}{A} & \textcolor{red}{G} & ? & \textcolor{red}{C} & ? &? &  \textcolor{red}{T} & ? & \textcolor{red}{G} & ? & ? & \textcolor{red}{A} & \textcolor{red}{C} & ? & \textcolor{red}{T} & \textcolor{red}{T} & \textcolor{red}{C} & ? &\\
1 & 1 & 1 & 1 & 1 & 1 & 1 & 1 & 1 & 1 & 1 & \ & \ & \ & \ & \ & \ &  \ & \ &1 & 1 & 1 & * & 1 & * &* &  1 & * & 1 & * & * & 1 & 1 & * & 1 & 1 & 1 & \\
\ &1 & 1 & 1 & 1 & 1 & 1 & 1 & 1 & 1 & 1 & \textcolor{red}{1} &  \ & \ &\ & \ & \ & \ & \ & \ & 1 & 1 & \textcolor{red}{1} & * & \textcolor{red}{1} & * &* &  \textcolor{red}{1} & * & \textcolor{red}{1} & * & * & 1 & \textcolor{red}{1} & * & 1 & 1 & \textcolor{red}{1} &\\
\end{tabular} \\
}
\caption[Consecutive seed and spaced seed]{Consecutive seed (left) and spaced seed (right). The red 1's are new matches required for a hit at the next position; it is much easier for the consecutive seed to have consecutive hits, hence its hits are more clustered  than those of the spaced seed.}
\label{seeds}
\end{center}
\end{figure}

\begin{figure}[h!]
\begin{center}
\includegraphics[ height= 7cm]{img/seed_sens.png}
\caption[The sensitivity comparison of three spaced-seed and BLAST seed]{The sensitivity comparison of three spaced-seed and BLAST seed. (From \cite{Ma02_PatternHunter}).    \label{fig_seed_sens}}
\end{center}
\end{figure}

\subsubsection{Multiple spaced seeds \label{sec_multiple_spaced_seed}}
Different spaced-seeds detect different homologies. PatternHunterII \cite{Li04_PatternHunterII} proposed the idea of combining multiple spaced-seeds to detect more similarities. A sensitivities comparison is shown in Figure \ref{fig_multi_spaced_seed}. The x-axis, similarity, denotes the percentage of identities in the homology region, and the y-axis, sensitivity, is the probability of having a hit, in the given region. The first observation is that the increment of the seed number improves the sensitivity. It is also notable that typically, doubling the number of seeds gains better sensitivity than decreasing the weight by 1.
\begin{figure}
\begin{center}
\includegraphics[ height= 7cm]{img/multi_spaced_seeds.png}
\caption[The sensitivity comparison on multiple spaced-seed]{The sensitivity comparison on multiple spaced-seed.  From low to high, the solid curves are the sensitivity of multiple spaced seed of weight 11. Denote the number of seeds as $k$, here $k=1,2,4,8,16$. 
The dashed curves are the sensitivity of single optimal spaced-seed of weight 10, 9, 8, 7. (From \cite{Li04_PatternHunterII}).    \label{fig_multi_spaced_seed}}
\end{center}
\end{figure}

Because of the advantage in sensitivity, SPRINT adopts the multiple spaced-seeds. The next natural step is to pick the shape of the seeds. Computing optimal seeds is a hard problem. Even the heuristic algorithms are exponential, except for SpEED \cite{ilie2007multiple, ilie2011speed}. We have thus used SpEED to compute the multiple spaced seeds to be used in our PPI prediction program. 

\subsubsection{Substitution matrix}
In bioinformatics and evolutionary biology, a substitution matrix describes the rate at which one character in a sequence changes to other character states over time. The similarity between sequences depends on their divergence time and the substitution rates as represented in the matrix. Substitution metrics like PAM and BLOSUM are often used as similarity measurement for protein sequences \cite{eddy2004_BLOSUM_PAM}, and thus they are sometimes called similarity matrices. An example, the BLOSUM62 matrix is shown in Figure \ref{fig_BLOSUM62}.

\begin{figure}[h!]
\begin{center}
\includegraphics[ height= 7cm]{img/BLOSUM62.png}
\caption[The BLOSUM62 matrix]{An example of substitution matrix, the BLOSUM62 matrix. BLOSUM 62 is a matrix calculated from comparisons of sequences with a pairwise identity of no more than 62\%. From \text{en.wikipedia.org/wiki/BLOSUM}    \label{fig_BLOSUM62}}
\end{center}
\end{figure}

\subsubsection{Interactome prediction}
The PPI prediction problem, interactome prediction means to score every pair in a proteome of an organism. Denote the total number of protein sequences in an organism as $P$, its interactome has $(1+P)\times P\div 2$ interactions. This is a relatively big number for advanced organisms. For instance, human has about 20,000 proteins, and human's interactome has $(1+20000) \times 20000\div 2 = 200,010,000$ interactions to predict. Comparing to the PPI site prediction problem, which only involves the computation on $P$ sequences, the amount of computation is much higher in the PPI prediction problem.

\subsection{Previous Methods}
As shown in Table \ref{table_experimental_ppi_pred}, various experimental techniques for identifying PPIs have been developed, most notably high throughput procedures such as two-hybrid assay and affinity systems \cite{shoemaker2007deciphering}. 
\begin{table}[h!]
\begin{center}
\includegraphics[ width = 16cm, height= 7cm]{img/experimental_PPI_identification.png}
\caption[Experimental methods for PPI identification]{Experimental methods for PPI identification (From \cite{shoemaker2007deciphering}). High-throughput methods are marked with + in the second column. The fourth column shows whether the method on physically interacting proteins in a complex (‘‘complex’’) or only pairwise interactions (‘‘binary’’).   \label{table_experimental_ppi_pred}}
\end{center}
\end{table}

The yeast two-hybrid (Y2H) approach was proposed by Suter et al. \cite{fields1989novel}. It detects a protein interaction by telling the signal generated by the DNA-binding domain (DBD) and the activation domain (AD) \cite{suter2008two}. Figure \ref{E_M1} shows the basic procedure of the Y2H approach. To detect whether protein 1 (green) and protein 2 (red) interact or not, then DBD and AD are bound, each to one of the proteins. The DBD and AD will connect together only when protein 1 and protein 2 interact with each other. When DBD and AD connect, a reporter will be generated. Thus, protein 1 and protein 2 are interacting if a reporter is detected, and vice versa \cite{suter2008two}.
\begin{figure}[h!]
\begin{center}
\includegraphics[height =7 cm, width = 9cm]{img/E_M1.jpg}
\caption[The Y2H approach]{The Y2H approach. To verify is protein 1 and 2 interact, DBD and AD are attached to them. If the two proteins interact, DBD and AD will bond and a reporter will be generated. The interaction is detected if the reporter is detected., From: \cite{suter2008two} \label{E_M1}}
\end{center}
\end{figure}

Another widely used experimental approach is tandem affinity purification (TAP), which was invented by Rigaut et al. \cite{rigaut1999generic}. It can detect protein complex interaction by adding a "tag" to target proteins. Figure \ref{E_M2} shows the general process of TAP. If we want to test whether interaction exists between protein $A$ and protein $B$, a tag is added to protein $A$. After tagging, the complex is washed twice to remove the tag and unstable bindings. Isolated interacting proteins can be found if there are some. After washing, the mass spectrometry or other methods will be used to detect remaining proteins. The remaining ones are marked as interacting proteins \cite{rigaut1999generic}.
\begin{figure}[h!]
\begin{center}
\includegraphics[height =9 cm, width = 6cm]{img/E_M2.jpeg}
\caption[Tandem affinity purification]{Tandem affinity purification. To verify if Protein A and B interact, a tag is linked to Protein A. A is able to bind the matrix in the affinity column. Molecules that bind with Protein A will also remain in the column. The second wash flushes all molecules, and those molecules considered Protein A's interaction parterner. From: \cite{jessulat2011recent} \label{E_M2}}
\end{center}
\end{figure}

Experimental methods are time and labor intensive and have a high rate of error. Therefore, a variety of computational methods have been designed to help predicting PPIs, employing sequence homology, gene co-expression, phylogenetic \cite{shoemaker2007deciphering, liu2012proteome, zahiri2013computational}.
profiles, etc. [3–5].


In general, based on the type input, computational methods can be classified into two categories: sequence based and structure based \cite{zhang2018review}. Sequence based methods \cite{ Martin05_PPIpred,Pitre06_PIPE,Shen07_PPIpred,Guo08_PPIpred,yu2010predicting, xia2010predicting,shi2010predicting, guo2010pred_ppi,zhang2011adaptive, liu2012spps, yousef2013novel,zahiri2013ppievo, zahiri2014locfuse, you2015detecting, hamp2015evolutionary, jia2015prediction, huang2015using,   you2015predicting, hamp2015more, ding2016predicting,gao2016ens, huang2016sequence, an2016using,you2017improved, hashemifar2018predicting, chen2019multifaceted, chen2019protein} are faster and more universally applicable because comparing to proteins structures, much more protein sequences are available. Here we introduce several state-of-the-art sequence based PPI prediction methods. One of them is traditional algorithmic, and the rest employs machine learning algorithms.

\subsubsection{PIPE}
PIPE (Protein Interaction Prediction Engine)\cite{Pitre06_PIPE} is one of the most widely known traditional algorithmic methods. It predicts interactions based on re-occurring short sub-sequences in an existing interaction database. The  four major steps of PIPE are shown in Figure \ref{fig_PIPE}.
\begin{figure}[h!]
\begin{center}
\includegraphics[height =13 cm]{img/PIPE_procedure.png}
\caption[The algorithm of PIPE]{The algorithm of PIPE. Step 1: building interaction graph from interaction database. Step 2: detecting similar sub-sequences for input $A$. Step 3: detecting similar sub-sequences for input $B$ and build the result matrix for ($A, B$). Step 4: outputing the result based on the result matrix. (From: \cite{Pitre06_PIPE} ) \label{fig_PIPE}}
\end{center}
\end{figure}
The input is a protein pair ($A$, $B$), and the output is the probability of $A$ and $B$ interact. It first puts a 20-sized window $w$ on top of $A$, then compares $w$ with every 20-sized piece in the protein sequence database. PIPE compares the similarity between two 20-sized pieces using PAM120 matrix. If the score between the two pieces are higher than a threshold (default: 35), they are considered as similar pieces. All proteins that contain similar piece with protein $A$, will be stored in a set $R$. 

Based on the interaction database, all proteins that interact with the proteins in $R$, are denoted as Neighbors ($R$). PIPE will then put a 20-sized window $w$ on protein $B$. It compares every piece in $B$ with every 20-sized piece in Neighbors($R$).

A result matrix $M$ is used to denote the score of the query ($A$, $B$). The size of the matrix is length of $A$ times length of $B$. For example, protein $A$ is similar with $C$ at position $i$, and protein $B$ is similar with $D$ at position $j$. We also know that $C$ and $D$ interact from the database, then we will increase the score of $M$($i$, $j$) by one.

In the end, the height of the highest peak in the matrix is the score for query ($A$, $B$). If the score is higher than a set threshold, PIPE will claim $A$ and $B$ interact.

The pseudocode of PIPE is shown in Figure \ref{fig_pseudocode_pipe}. The time complexity of this algorithm is $O(P^3L^2)$ where $P$ is the number of proteins and $L$ is the average protein length. If the human interactome prediction (20117 proteins with average length 557) needs to be performed using a computer that does $10^9$ operations per second, the theoretical estimated running time for PIPE is $(20117^3*557^2)/10^9$s $\approx 80 $ years.  

\begin{figure}[h!]
\begin{center}
\includegraphics[width =17cm]{img/psedo_code_pipe.png}
\caption[The pseudocode of PIPE]{The pseudocode of PIPE.  \label{fig_pseudocode_pipe}}
\end{center}
\end{figure} 

The second version of PIPE, PIPE2 \cite{Pitre08_PIPE2}, increased both the speed and the accuracy from the previous version. 

For speed, instead of computing all similar pieces "on the fly", PIPE2 pre-computes all similar pieces for all protein sequences and stores them locally. PIPE2's pseudocode is shown in Figure \ref{fig_pseudocode_pipe2}. All pieces that are similar with $a$ and $b$ have been pre-computed so that the program does not waste time on computing the same re-occurring piece repeatedly. The time complexity of PIPE2, ignoring the pre-computating part, is improved to $O(P^2L^2R)$ where the size of $R$ is uncertain. After running PIPE2 for more than 80 hours, we estimated its running time on the entire human interactome prediction to be 1520 days.   
\begin{figure}[h!]
\begin{center}
\includegraphics[width =15cm]{img/PIPE2_psedu.png}
\caption[The pseudocode of PIPE2]{The pseudocode of PIPE2.  \label{fig_pseudocode_pipe2}}
\end{center}
\end{figure} 

In terms of accuracy improvement, PIPE2 applies a binary filter which works as shown in Figure \ref{fig_binary_filter}. The binary filter flattens the matrix by checking the surrounding areas. For each cell containing the value $c$, if the surrounding eight cells have more non-zero values than zero values, then it replaces $c$ with 1, otherwise it replaces $c$ with 0. For example, the middle cell 34, has three non-zero values and five zero-values around it, so the filter replaces 34 with 0.

\begin{figure}[h!]
\begin{center}
\includegraphics[height = 4 cm]{img/binary_filter.png}
\caption[The binary filter of PIPE2]{The binary filter of PIPE2. The centered number is replaced with 1 if there are more non-zero values than zero values in the surrounding eight cells, with 0 otherwise.  \label{fig_binary_filter}}
\end{center}
\end{figure} 

PIPE's third version \cite{schoenrock2011mp}, PIPE3, parallels PIPE2. The algorithm is still slow, but it was the first program able to predict the entire human interactome. It took over 3 months to run the entire human proteome on 50 nodes with 12,800 parallel computational threads.

\subsubsection{Martin's program}
Martin \cite{Martin05_PPIpred} is one of the earliest PPI perdition methods that use machine learning. It is highly cited and still considered one of the most effective ones. The workflow of Martin is shown in Figure \ref{fig_Martin}. The core idea of this method is using Support Vector Machine (SVM) \cite{hearst1998support} to classify PPIs. The inputs to the SVM is the encoding of two protein sequences. Encoding protein sequences were not a well studied problem at that time, and their solution was to use signature molecular descriptor \cite{visco2002developing, faulon2003signature1, faulon2003signature2}. The signature of a sequence $A$ is defined as $s(A) = \sigma_iz_i$ where $z_i$ is the all trimers of amino acids and $\sigma_i$ is the number of occurrences of trimer $z_i$. Each trimer put the centered residue at front and the two neighboring residues afterwards. For example, the signature of the sequence $LVMTTM$ is denoted as $s(LVMTTM) = V(LM)+M(TV)+2T(MT)$. The two signatures are further converted into two tensors and then computed the tensor product.
\begin{figure}[h!]
\begin{center}
\includegraphics[height =7 cm]{img/sigProd.png}
\caption[The workflow of Martin]{The workflow of Martin. Two input protein sequences are encoded into signatures. The dot product of the two signatures are then computed. SVM is used to classify the dot product into interaction and non-interaction.  \label{fig_Martin}}
\end{center}
\end{figure} 

\subsubsection{Shen's program}
Shen's program \cite{Shen07_PPIpred} followed the idea of using SVM to classify PPIs. Instead of using the signature to represent protein sequences, ectrostatic (including hydrogen bonding) and hydrophobic information are used.
\subsubsection{Guo's program}
Guo's program \cite{guo2010pred_ppi} further followed the idea of using SVM by incorporating more features and auto covariance. The new features include interaction modes (electrostatic interaction, hydrophobic interaction, steric interaction and hydrogen bond), physicochemical properties. Auto cross covariance (ACC) was used to transform unequal sized feature vectors into uniform matrices. 

\subsubsection{Ding's program}
Ding et al. \cite{ding2016predicting} published a program utilizing random forest classifier for PPI prediction. The flowchart of the program is shown in Figure \ref{fig_Ding}. The protein sequences are encoded into a 638-D matrix consisting k-gram embedding and electrostatic and hydrophobic properties.
\begin{figure}[h!]
\begin{center}
\includegraphics[height =7 cm]{img/Ding.png}
\caption[The workflow of Ding's program]{The workflow of Ding's program. The two input protein sequences are represented using a vector of dimension 638, consisting k-gram embedding and electrostatic and hydrophobic properties. The classifier is random forest. \label{fig_Ding}}
\end{center}
\end{figure} 
\subsubsection{DPPI}
DPPI \cite{hashemifar2018predicting} is a recent PPI prediction program using deep learning. It employs a Siamese-like network structure which learns from two proteins. The input sequences are firstly represented using PSSMs, then both matrices are passed to convolutional layers. Then a random projection module is applied to reduce the dimension of the tensor as well as to improve the prediction performance. In the end a linear layer outputs the prediction result.
\begin{figure}[h!]
\begin{center}
\includegraphics[height =6 cm, width = 17cm]{img/DPPI.png}
\caption[The architecture of DPPI]{The architecture of DPPI. The sequences are encoded using PSSMs, passed into convolutional random projection, and linear layers. From \cite{hashemifar2018predicting} \label{fig_DPPI}}
\end{center}
\end{figure} 
\subsubsection{PIPR}
PIPR \cite{chen2019multifaceted} is one of the most recent PPI predictors. Similar to DPPI, it also employees a Siamese-like network using deep learning algorithms. As shown in Figure \ref{fig_PIPR}, two protein sequences are represented using embedding. The embedding contains two parts: Skip-Gram \cite{mikolov2013distributed} based embedding and electrostaticity and hydrophobicity characteristics. The embedding is then passed to residual RCNN network. The residual CRNN consists convolutional and bidirectional GRU layer. The GRU layer has shortcut connection between the input and the output. The outputs of the RCNN unit is put together using element-wise multiplication, and then the product is passed to fully-connected layers for prediction. 

\begin{figure}[h!]
\centering
\begin{subfigure}[b]{\textwidth}
   \includegraphics[width = 17cm]{img/PIPR_archi.png}
   \caption{}
   \label{fig_Ng1} 
\end{subfigure}

\begin{subfigure}[b]{\textwidth}
   \includegraphics[width = 17cm]{img/PIPR.png}
   \caption{}
   \label{fig_PIPR}
\end{subfigure}

\caption[The architecture of PIPR]{(a) The overall architecture of PIPR. The flowchart consists input sequences, embedding, residual RCNN, element-wise multiplication, multi-layer perceptron. (b) The details of the RCNN unit. The RCNN combines a CNN and a RNN components. The RNN part is bidirectional GRU layer with residual shortcuts that bridge the input and the output of the GRU layer.}
\end{figure}

\section{Methods and Materials}
We describe in this section both the algorithm and the implementation of SPRINT in details.
\subsection{Basic idea}
Proteins similar with interacting proteins are likely to interact as well. That is, if $P_1$ is known to interact with $P_2$ and the sequences of $P_1$ and $P'_1$ are highly similar and the sequences of $P_2$ and $P'_2$ are highly similar, then $P'_1$ and $P'_2$ are likely to interact as well. In a way or another, this is essentially the idea behind the brute force calculation of PIPE as well as the machine learning algorithms of Martin, Shen, and Guo. 

SPRINT uses a complex algorithm to quickly evaluate the contribution of similar subsequences to the likelihood of interaction. SPRINT has two main steps: computing HSP and predicting interactions. 

High-scoring segment pairs (HSPs) are similar subsequences among all input sequences. SPRINT first detects these HSPs from the input protein sequences. Figure \ref{fig_compute_hsp} is the visualization of the process. Four proteins, P1 to P4, are input to the Compute HSPs program. After computation, three HSP pairs are found, that is  subsequences A1, B1, C1 are similar to A2, B2, C2 respectively.  

\begin{figure}
  \includegraphics[width=0.95\textwidth]{img/compute_hsp}
\caption[The idea of computing HSPs]{The idea of computing HSPs. P1 - P4 represent four protein sequences. (A1, A2), (B1, B2), (C1, C2) represent three pairs of HSPs. Blocks with the same colour indicate a pair of HSP.}
\label{fig_compute_hsp}  
\end{figure}

After computing the HSPs, SPRINT takes in both the similarities and known PPIs as inputs and predict interactions. The idea of predicting interactions in SPRINT is shown in Figure \ref{fig_predict_PPI}. A, B, and C are HSP segments, and P1-3 and Q1-3 are proteins. Known PPIs (P1, Q1) and (P2, Q2) are fed to SPRINT along with the HSPs they contain. SPRINT assumes that in an known interaction, all HSPs are possible binding sites that cause the interaction. For example, (A, B) and (A, C) are assumed to be the binding sites in the interactions (P1, Q1) and (P2, Q2) repetitively, and they are expected to behave the same in an novel interaction (P3, Q3). Thus, the interaction score of (P3, Q3) is increased based on the possible biding between (A, B) and (A, C). The higher the score, the more confidently SPRINT claims the interaction. 

\begin{figure}
  \includegraphics[width=0.95\textwidth]{img/predict_PPI.png}
\caption[The idea of predicting PPIs]{The idea of predicting PPIs. A, B, C are HSPs. The blocks with the same colour indicate HSP pairs. (P1, Q1) and (P2, Q2) are known interactions. The interaction score of (P3, Q3) is calculated.}
\label{fig_predict_PPI}  
\end{figure}

% \begin{figure}[h!]
% \centering
% \includegraphics[width=8cm]{img/fig_idea.pdf}
% \caption[Interaction inference]{Interaction inference. The proteins $P_1$ and $Q_1$ are known to interact; blocks of the same colour represent occurrences of similar subsequences. Dashed lines indicate potential contributions to interactions: there are six between $P_1$ and $Q_1$ and they imply two between $P_2$ and $Q_2$ and three between $P_3$ and $Q_3$. \label{fig_idea}}
% \end{figure}

Long similar regions should have a higher weight than short ones. To account for this we assume that all contributing blocks have a fixed length $k$ %(default $k = 20$) 
and that a region of length $\ell$ contributes $\ell-k+1$ blocks. As $k$ is fixed, this grows linearly with $\ell$. The precise score is given later in this section.

We put together the workflow of SPRINT in Figure \ref{fig_SPRINT_workflow}, The details of each step are given in the following sections.

\begin{figure}[h!]
\centering
\includegraphics[width=15cm]{img/SPRINT_workflow.png}
\caption[The workflow of SPRINT]{The workflow of SPRINT. First sequences are indexed with spaced-seeds into multiple hashtables (left). Then HSPs are detected by fast traversing the hashtables (middle). Last, known PPIs are loaded and used together with HSPs for predicting new interactions. \label{fig_SPRINT_workflow}}
\end{figure}

\subsection{Detecting Similarities \label{sec_detect_simi}}
As mentioned above, the first step of SPRINT is the identification of similar subsequences among the input protein sequences. This is done using multiple spaced seeds with the hit-and-extension method.

A spaced match requires only the amino acids in positions corresponding to {\tt 1}'s in the seed to match. For example,  a hit in seed 11****11***1 only needs the amino acids in positions 1, 2, 7, 8, and 12 have to match. Given the spaced seed above, two \textit{exact} spaced matches are underlined in Figure~\ref{fig_s-match}(a).

\begin{figure}[h!] 
\centering
\includegraphics{img/fig_s-match.pdf}
\caption[Spaced-seed hits]{An exact hit (a) and an approximate hit (b) of the same spaced seed.\label{fig_s-match}}
\end{figure}

There is a trade-off between speed and sensitivity. Lower weight has increased sensitivity because it is easier to hit similar regions but lower speed since more random hits are expected and have to be processed. The best value for our problem turned out to be five.

As briefly mentioned in Section \ref{sec_multiple_spaced_seed}, the distribution of matches and don't care positions is crucial for the quality of the seeds and we have used SpEED~\cite{Ilie07_spacedSeeds,Ilie11_SpEED} to compute the following seeds used by SPRINT; we have experimentally determined that four seeds of weight five are the best choice: $\textsc{Seed}_{4,5} = \{$\texttt{11****11***1}, \texttt{1**1*1***1*1}, \texttt{11**1***1**1}, \texttt{1*1******111}$\}$.

In order to further increase the probability of finding similar subsequences, we consider also hits between similar matches, as opposed to exact ones. For example, the two amino acid sequences in Figure~\ref{fig_s-match}(b), though similar, do not have any \textit{exact} spaced matches. In order to capture such similarities, we consider also hits consisting of \textit{similar} spaced matches; an example is shown by the underlined subsequences in Figure~\ref{fig_s-match}(b).

To make this idea precise, we need a few definitions. {\it Spaced-mers} are defined analogously with $k$-mers but using a spaced seed. A $k$-mer is a contiguous sequence of $k$ amino acids. Given a spaced seed, a spaced-mer consists of $k$ amino acids interspersed with spaces, according to the seed. For a spaced seed $s$, we shall call the spaced-mers also $s$-mers. Figure~\ref{fig_s-mers} shows an example of all $s$-mers of a sequence, for $s=${\tt 11****11***1}:

\begin{figure}[h!]
\centering
\includegraphics{img/fig_s-mers.pdf}
\caption{An example of all $s$-mers of a sequence.\label{fig_s-mers}}
\end{figure}

An exact hit therefore consists of two occurrences of the same $s$-mer. An approximate hit, on the other hand, requires two similar $s$-mers.
Assume a similarity matrix $M$ is given. Given a seed $s$ and two $s$-mers $w$ and $z$, the score between the two $s$-mers is given by the sum of the scores of the pairs of amino acids in the two $s$-mers, that is, we sum over indexes corresponding to \texttt{1}'s in the seed:
\begin{equation}\label{eq:s-mer_score}
S_\textrm{$s$-mer}(w,z) = \sum_{s[i]=\texttt{1}}M(w_i,z_i)\ .
\end{equation}
For example, for the $s$-mers $w = \texttt{VL\textvisiblespace\textvisiblespace\textvisiblespace\textvisiblespace KT\textvisiblespace\textvisiblespace\textvisiblespace A}$ and $z = \texttt{HL\textvisiblespace\textvisiblespace\textvisiblespace\textvisiblespace KS\textvisiblespace\textvisiblespace\textvisiblespace A}$ from Figure~\ref{fig_s-match}(b), we have $S_\textrm{$s$-mer}(w,z) = M(\texttt{V},\texttt{H}) + M(\texttt{L},\texttt{L}) + M(\texttt{K},\texttt{K}) + M(\texttt{T},\texttt{S}) + M(\texttt{A},\texttt{A}).$

Using (\ref{eq:s-mer_score}), we define the set of $s$-mers that are \textit{similar} with a given $s$-mer $w$:
\begin{equation}\label{eq:Sim}
\textrm{Sim}(w) = \{z \mid z \text{ $s$-mer}, S_\textrm{$s$-mer}(w,z) \ge \Thit\}\ .
\end{equation}

Note that $\textrm{Sim}(w)$ depends on the parameter $\Thit$ that controls how similar two $s$-mers have to be in order to form a hit. It also depends on the seed $s$ and the similarity matrix $M$ but we do not include them into the notation, for clarity.

All such hits dues to similar $s$-mers are found and then extended both ways in order to identify similar regions. That means, now we have to evaluate the similarity of all the amino acids involved, so we use the regular $k$-mers.
The score between two $k$-mers $A$ and $B$ is computed as the sum of all scores of corresponding amino acids:
\begin{equation}\label{eq_score_kmer}
S_\textrm{$k$-mer}(A,B) = \sum_{i=1}^kM(A_i,B_i)\ ,
\end{equation}
where $A_i$ is the $i$th amino acid of $A$. Given a hit that consists of two $s$-mers $w$ and $z$, we consider the two $k$-mers that contain the occurrences of the two $s$-mers $w$ and $z$ in the center, denoted $\textrm{$k$-mer}(w)$ and $\textrm{$k$-mer}(z)$. If $S_\textrm{$k$-mer}(\textrm{$k$-mer}(w),\textrm{$k$-mer}(z)) \ge \Tsim$, then the two regions are deemed similar. Note the parameter $\Tsim$ that controls, together with $k$-mer size $k$, how similar two regions should be in order to be identified as such.

Details of the fast implementation are given next. The protein sequences are encoded into bits using five bits per amino acid. We encode each amino acid with 5 bits. Blocks of 5 bits can have $2^5$ = 32 different values. So 5 bits are more than enough for encoding twenty amino acids. 
Table \ref{tab_A_A_encoding} shows in detail the encoding of each amino acid. The five bits used for encoding are unrelated with the weight of the spaced seeds employed. It is a coincidence that both numbers are five. Each protein sequence is encoded as an array of unsigned 64-bit integers; each 64-bit integer stores 12 amino acids within 60 bits and 4 bits are unused. Each spaced seed is encoded using also five bits per position, {\tt 11111} for a {\tt 1} (match) and {\tt 00000} for a {\tt *} (don't care), also into an unsigned 64 integer.  

\begin{table}[h!]
\begin{center}
\begin{tabular}[h!]{|c|c|c|} 
%c stands for centre, l for left, r for right; the | puts lines in between, and the hline puts a horizontal line in
\hline
Amino acid name & Notation in FASTA format & 5-bit encoding \\
\hline
Alanine & A & 00000 \\
\hline
Arginine & R & 00001\\
\hline
Asparagine & N & 00010 \\
\hline
Aspartic acid & D & 00011 \\
\hline
Cysteine & C & 00100 \\
\hline
Glutamine & Q & 00101 \\
\hline
Glutamic acid & E & 00110 \\
\hline
Glycine & G & 00111 \\
\hline
Histidine & H & 01000\\
\hline
Isoleucine & I & 01001 \\
\hline
Leucine & L & 01010 \\
\hline
Lysine & K & 01011 \\
\hline
Methionine & M & 01100 \\
\hline
Phenylalanine & F & 01101 \\
\hline
Proline & P & 01110\\
\hline
Serine & S & 01111 \\
\hline
Threonine & T & 10000 \\
\hline
Tryptophan & W & 10001 \\
\hline
Tyrosine & Y & 10010 \\
\hline
Valine & V & 10011 \\
\hline
\end{tabular}
\caption[Amino acid encoding]{Amino acid encoding. Each amino is encoded with 5-bits (last column). \label{tab_A_A_encoding}}
\end{center}
\end{table}

All spaced-mers in all protein sequences are computed and stored in a hash table, together with their location in the protein sequences. Because of our representation, the computation of each spaced-mer requires only one bitwise AND and one bit SHIFT operation which is much faster than string operations. 

Finding HSPs start with finding hits, that is, matching $s$-mers. In order to do this fast, we use a hash table. All extracted $s$-mer will be stored in the hash table. For each seed, we have one hash table. The hash function is shown in \ref{eq_hash_func}. Liner probe is used for hash collision.  
\begin{equation}
entry\_index = s\_mer\_value \text{ }\%\text{ } hash\_size
\label{eq_hash_func}
\end{equation}

where the hash size is determined when encoding proteins. For each seed, the maximum number of $s$-mers is:
\begin{displaymath}
number{\_}of{\_}smers = \sum_{i=1}^{num\_of\_pro} (protein\_length - seed\_length +1).
\end{displaymath}

The program automatically chooses the smallest prime number which is bigger than the number of $s$-mers, from a list of precomputed large prime numbers, as hash size. Hash collisions are solved using linear probing. In each entry of the hash table, we store its $s$-mer value along with a pointer to an array of positions, which keeps a record of protein ids and starting positions of that $s$-mer. The pseudocode for computing the eight hash tables is given in Algorithm  \ref{algo:comp_ht}. Algorithm \ref{algo:insert_ht} shows the insertion into hash table in detail.

\begin{algorithm}[h!]
 \SetAlgoLined 
 \SetKwInOut{Input}{input}
\SetKwInOut{Output}{output}
\caption{Algorithm for computing hash tables \label{algo:comp_ht}}
 \Input{encoded\_proteins $EP$, 8 encoded\_seeds $S$}
 \Output{8 hash tables $HT$}
 \For {each seed $s\in S$}
 {
 	- initialize $ht\in HT$\\
 	- \For {each protein $p\in EP$}
 	{
 		- $temp\_protein\_sequence$ = $p$\\
 		- \For {$position\in [0 , (pro\_length - seed\_length + 1)]$}
 		{
			- $s$-mer = $s$ \& $temp\_protein\_sequence$\\
			- $temp\_protein\_sequence$ shift one position\\
			- insert\_into\_hash\_table ($s$-mer, $position$, $ht$)
 		}
 	}
 }
\end{algorithm}

\begin{algorithm}[h!]
 \SetAlgoLined 
 \SetKwInOut{Input}{input}
\SetKwInOut{Output}{output}
\caption{Algorithm for inserting $s$-mer into hash table \label{algo:insert_ht}}
 \Input{$s$-mer, $s$-mer$\_position$, $hash\_table$}
 \Output{}
	position $p$ = $s$-mer \% $hash\_size$ \\
	\If {$hash\_table[p] == \emptyset$}
	{
		insert ($s$-mer, $s$-mer$\_position$) into $hash\_table$[$p$]\\
	}
	\ElseIf {$s$-mer $== hash\_table[p].s$-mer}
	{
		add ($s$-mer$\_position$) into $hash\_table$[$p$] as a linked list\\
	}
	\ElseIf {$s$-mer $\neq hash\_table[p].s$-mer}
	{
		insert\_$s$-mer\_into\_hashtable($s$-mer, $p$ + 1, $hash\_table$)\\
	}
\end{algorithm}

Once all spaced-mers are stored, for each spaced-mer in the hashtable, all similar spaced-mers are computed and then all hits between the spaced-mer and similar ones are investigated from the table and extended in search for similarities. 

\subsection{Predicting Interactions}
As previously showed in Figure \ref{fig_SPRINT_workflow}, after obtaining the HSPs, the next step is to predict interactions.
\subsubsection{Post-processing similarities}
We first process the similar subsequences we computed in the previous phase to remove those appearing too many times as they are believed to be just repeats that occur very often in the protein sequences without any relevance for the interaction process. We explain the algorithm on the toy example below. For the protein sequence {\tt MVLSPADKTNVKAAWG}, assume we have found the similarities marked by lines in  Figure~\ref{fig_similarities}(a). For example, the top line means that {\tt MVLSP} was found to be similar with another subsequence somewhere else, the bottom line represents the same about the subsequence {\tt KTNVKAAW}, etc. 

\begin{figure}[h!]
\centering
\includegraphics{img/fig_similarities.pdf}
\caption{An example of similarities before (a) and after (b) post-processing.\label{fig_similarities}}
\end{figure}

The counts in the bottom row indicate how many times each position occurs in all similarities found. (In the figure above, this means the number of lines that cover that position.) All positions with a high count, above a threshold $\Thc$, will be eliminated from all similarities, which will be modified accordingly. In our example, assuming the threshold is 5, positions 3, 4, 8, 9, and 10 have counts 5 or higher and are eliminated; see Figure~\ref{fig_similarities}(b). The new similarities are indicated by the lines above the sequence. For example, {\tt MVLSP} has positions 3 and 4 removed and becomes two similarities, {\tt MV} and {\tt P}. The counterpart of each similarity is modified the same way.
\subsubsection{Scoring function}
What we have computed so far are HSPs, that is, pairs of similar subsequences of the same length. We now show how to score an interaction. First, we extend the definition of the score from $k$-mers to arbitrary subsequences of equal length. For two subsequences $X$ and $Y$ of length $n$, the score is given by the sum of the scores of all corresponding $k$-mer pairs; using \eqref{eq_score_kmer}:
\begin{equation}\label{eq_score_nmer}
S_e(X,Y) = \sum_{i=1}^{n-k+1}S_\textrm{$k$-mer}(X[i\pp i+k-1],Y[i\pp i+k-1])\ ,
\end{equation}
where $X[i\pp j] = X_iX_{i+1}\cdots X_j$. It is important to recall that any two similar sequences we find have the same length, therefore the above scoring function can be used.

Finally, we describe how the scores for whole protein sequences are computed. Initially all scores are set to zero. Each pair of proteins $(P_1, P_2)$ that are known to interact has its own contribution to the scores of other pairs. For each computed similarity $(X_1,Y_1)$ between $P_1$ and another protein $Q_1$ ($X_1$ is a subsequence of $P_1$ and $Y_1$ is a subsequence of $Q_1$) and for each similarity $(X_2,Y_2)$ between $P_2$ and another protein $Q_2$, the score between $Q_1$ and $Q_2$, $S_p(Q_1,Q_2)$, is increased, using \eqref{eq_score_nmer}, by:
\begin{equation}\label{eq_score_add}
\begin{array}{l}
S_p(Q_1,Q_2)\gets S_p(Q_1,Q_2)\\

 + \displaystyle{\frac{S_e(X_1,Y_1)(|X_2|-k+1)+ S_e(X_2,Y_2)(|X_1|-k+1)}{|Q_1||Q_2|}},
 \end{array}
\end{equation}
where $|Q|$ denotes the length of the amino acid sequence $Q$. That means, the score of each corresponding $k$-mer pair between $X_1$ and $Y_1$ is multiplied by the number of $k$-mers in $X_2$, that is, the number of times it is used to support the fact that $Q_1$ is interacting with $Q_2$. Similarly, the score of each corresponding $k$-mer pair between $X_2$ and $Y_2$ is multiplied by the number of $k$-mers in $X_1$. The score obtained this way is then normalized by dividing it by the product of the lengths of the proteins involved. 

Once the score are computed, by considering all given interactions and similar subsequences and computing their impact on the other scores as above, predicting interactions is simply done according to the scores. All protein pairs are sorted decreasingly by the scores; higher scores represent higher probability to interact. If a threshold is provided, then those pairs with scores above the threshold are reported as interacting. 

\subsection{Implementations}
\subsubsection{Tuneable parameters}
\label{sec:parameter}
There are some tuneable parameters (see Table \ref{tab_parameter_SPRINT}) in SPRINT. In general, the default values are tuned experimentally and set based on the human datasets used in \cite{li2017sprint}, so it is recommended to use the default values if PPI prediction is performed on human datasets. The default values for the parameters are $k = 20$, $\Thit = 15$, $\Tsim = 35$, $M=PAM120$, and $\Thc = 40$. These values have been experimentally determined using only Park and Marcotte's data set. All the other datasets have been used exclusively for testing. The program is quite stable, the results being almost unaffected by small variations of these parameters.
\begin{table}[]
\begin{adjustwidth}{-2.5cm}{}
\caption{Tuneable Parameters in SPRINT}
\begin{tabular}{@{}lll@{}}
\toprule
Parameter & Default           & Description                                                        \\ \midrule
-Thit      & 15        & The similarity threshold to form a hit                             \\
-Tsim      & 35        & The similarity threshold to form an length-20 HSP                  \\
-M         & PAM120        & The scoring Matrix used to compute similarity between two subsequences \\
-Thc       & 40 & The threshold to consider a position high count                    \\ \bottomrule
\end{tabular}
\label{tab_parameter_SPRINT}
\end{adjustwidth}
\end{table}

\subsubsection{Pseudocode}
We put all the above together to summarize the SPRINT algorithm for predicting PPIs.
The input consists of the proteins sequences and PPIs. The default set of seeds is given by $\textsc{Seed}_{4,5}$ above but any set can be used.


\smallskip

\noindent
\setcounter{algccc}{0}
\underline{$\text{\sc SPRINT}(P_s,P_i)$}\\
\textbf{input:} protein sequences $P_s$, protein interactions $P_i$\\
\textbf{global:} seed set $\textsc{Seed}$\\
\textbf{output:} all protein pairs sorted decreasingly by score\\
\   [Hash spaced-mers]\\
\ccc  \textbf{for} each seed $s$ in $\textsc{Seed}$ \textbf{do}\\
\ccc  \qqa \textbf{for} each protein sequence $p$ in $P_s$ \textbf{do}\\
\ccc  \qqb \textbf{for} $i$ \textbf{from} $0$ \textbf{to} $|p| - |s|$ \textbf{do}   \\
\ccc  \qqc $w$ $\gets$ the $s$-mer at position $i$ in $p$\\
\ccc  \qqc store $w$ in hash table $H_s$\\
\ccc  \qqc store $i$ in the list of $w$ [pos. where $w$ occurs]\\
\   [Compute similarities]\\
\ccc  \textbf{for} each seed $s$ in $\textsc{Seed}$ \textbf{do}\\
\ccc  \qqa \textbf{for} each $s$-mer $w$ in $H_s$ \textbf{do}\\
\ccc  \qqb $\textrm{Sim}(w)\gets$ the set of $s$-mers similar with $w$ \eqref{eq:Sim} \\
\ccc  \qqb \textbf{for} each $z \in \textrm{Sim}(w)$ \textbf{do}\\
\ccc  \qqc \textbf{for} each position $i$ in the list of $w$ \textbf{do}\\
\ccc  \qqd \textbf{for} each position $j$ in the list of $z$ \textbf{do}\\
\ccc  \qqe \textbf{if} $S_\textrm{$k$-mer}(\textrm{$k$-mer}(w),\textrm{$k$-mer}(z)) \ge \Tsim$\\
\ccc  \qqf \textbf{then} extend the similarity both ways\\
\ccc  \qqf \phantom{\textbf{then}} store the subsequence pair \\
\ccc Process similarities: remove pos. with count $\ge\Thc$ \\
\   [Compute scores]   \\
\ccc  \textbf{for} each pair $(P,Q) \in P_s\times P_s$ \textbf{do}\\
\ccc  \qqa $S_p(P,Q) \gets 0$\\
\ccc  \textbf{for} each $(P_1,P_2)\in P_i$ \textbf{do}   \\
\ccc  \qqa \textbf{for} each protein $Q_1$ and \\
\qqd each similarity $(X_1,Y_1)$ in $(P_1,Q_1)$ \textbf{do}   \\
\ccc  \qqb \textbf{for} each protein $Q_2$ and \\
\qqe each similarity $(X_2,Y_2)$ in $(P_2,Q_2)$ \textbf{do}   \\
\ccc  \qqc increase the score $S_p(Q_1,Q_2)$ as in \eqref{eq_score_add}   \\
\   [Predict PPIs]\\
\ccc  sort the pairs in $P_s\times P_s$ decreasingly by score   \\
\ccc  \textbf{if} a threshold is provided  \\
\ccc  \qqa \textbf{then} output those with score above threshold   \\






\subsubsection{System configuration}
The SPRINT program is written in C++ (GCC version 4.8.2) with boost library (version 1.53). The parallel version uses OpenMP library (version 1.8.3). OpenMP is an application program interface that enables programmers to parallel the code from high level. We paralleled the hashtable computation and scoring interactions part to accelerate SPRINT.

\section{Results}
\subsection{Datasets classification}
Park and Marcotte~\cite{ParkMarcotte12_C123} noticed that all methods have significantly higher performance for the protein pairs in the testing data whose sequences appear also in the training data. Three cases are possible, depending on whether both proteins in the test data appear in training (C1), only one appears (C2), or none (C3). They show that essentially all datasets previously used for cross validation are very close to the C1 type, whereas in the HIPPIE meta-database of human PPIs \cite{Schaefer12_HIPPIE} the C1-type human protein pairs accounts for only 19.2\% of these cases, whereas C2-type and C3-type pairs make up 49.2\% and 31.6\%, respectively. Therefore, testing performed on C1-type data is not expected to generalize well to the full population. The authors proceeded to designing three separate human PPI datasets that follow the C1, C2, and C3-type rules. 

\subsection{Datasets}
We first describe the procedure of Park and Marcotte~\cite{ParkMarcotte12_C123} in detail. The protein sequences are from UniProt~\cite{uniprot2012reorganizing}. The interactions were downloaded from the protein interaction network analysis platform \cite{Wu09_PINA} that integrates data from six public PPI databases: IntAct \cite{Kerrien07_IntAct}, MINT \cite{Chatr07_MINT}, BioGRID \cite{Stark11_BioGRID}, DIP \cite{Salwinski04_DIP}, HPRD \cite{Prasad09_HPRD} and MIPS MPact \cite{Guldener06_MIPS_MPact}. The datasets were processed by \cite{ParkMarcotte12_C123} as follows. Proteins in each data set were clustered using CD-HIT2~\cite{li2006cd} such that they shared sequence identity less than 40\%. Proteins with less than 50 amino acids as well as homo-dimeric interactions were removed. Negative PPI data were generated by randomly sampling protein pairs that are not known to interact. See \cite{ParkMarcotte12_C123} for more details.

The total number of proteins used is 20,117, involving 24,718 PPIs. The training and testing datasets are divided into forty splits (from the file human\_random.tar.gz), each consisting of one training file and three testing files, one for each type C1, C2, C3.  Therefore, each C1, C2, or C3 curve produced is the average of forty curves. In addition, they tested also 40-fold cross validation on the entire PPI set. In reality, the ratio between interacting and noninteracting protein pairs is believed to be 1:100 or lower. However, this would make it very slow or impossible to run some of the algorithms. Therefore, Park and Marcotte decided to use ratio 1:1.

We have used Park and Marcotte's procedure to design similar testing datasets using six other human PPI databases. Among the most widely known human PPI databases we have chosen six that appear to be the most widely used: Biogrid, HPRD, InnateDB (experimentally validated and manually curated PPIs), IntAct, and MINT. We have used 20,160 human protein sequences downloaded from UniProt. The protein sequences and interactions were downloaded in Oct.~2016. We perform four tests for each program on each dataset: 10 fold cross-validation using all PPIs and C1, C2, and C3 tests, the datasets for which are built as explained above, with the ratio between training and testing pairs of 10:1. The details of all datasets are given in Table~\ref{table_datasets}. 

\begin{table}
\centering
\caption[The datasets used for comparing PPI prediction methods]{The datasets used for comparing PPI prediction methods. The second column contains the total number of PPIs, while the third the fourth columns give the number of PPIs used for training and testing, respectively, in the C1, C2, and C3 tests.}
\label{table_datasets}
\begin{tabular}{@{}lrrrl@{}}
\toprule
Dataset & \multicolumn{3}{c}{PPIs} & Website  \\ \cmidrule{2-4}
	 & All & Training & Testing &  \\ \midrule
Park and Marcotte & 24,718 & 14,186 & 1,250 & \texttt{www.marcottelab.org} \\ 
Biogrid  & 215,029 & 100,000 & 10,000 & \texttt{thebiogrid.org} \\
HPRD Release 9 & 34,044 & 10,000 & 1,000 & \texttt{www.hprd.org} \\ 
InnateDB experim.~validated & 165,655 & 65,000 & 6,500 & \texttt{www.innatedb.com} \\ 
InnateDB manually curated & 9,913 &3,600 & 360 & \texttt{www.innatedb.com} \\ 
IntAct & 111,744 & 52,500 & 5,250 & \texttt{www.ebi.ac.uk/intact} \\ 
MINT & 16,914 & 7,000 & 700 & \texttt{mint.bio.uniroma2.it} \\ \bottomrule
\end{tabular}
\end{table}

\subsection{Competing methods}
We have compared SPRINT with five methods Martin's\cite{Martin05_PPIpred}, Shen's \cite{Shen07_PPIpred}, Guo's\cite{Guo08_PPIpred}, PIPE2\cite{Pitre08_PIPE2}, and Ding's\cite{ding2016predicting}.  Since the three methods do not have names, we use the first author's name to identify them: Martin \cite{Martin05_PPIpred}, Shen \cite{Shen07_PPIpred}, Guo \cite{Guo08_PPIpred}, and Ding \cite{ding2016predicting}.

Note that PIPE2 and SPRINT do not require negative training data as they do not use machine learning algorithms. All the other programs require both positive and negative training sets. All programs are trained and tested on the same data, and the results on the testing data are reported. 

\subsection{Comparative Analysis on Park \& Marcotte’s Datasets}
We present first the comparison of all five methods considered on the datasets of Park and Marcotte in Figure~\ref{fig_Park_Marcotte_curves}. The receiver operating characteristic (ROC) and precision-recall (PR) curves for the four tests, CV, C1, C2, and C3, are presented. 

\begin{figure}[h!]
\centering
\includegraphics{img/fig_PM_plots.pdf}
  \caption[Performance comparison on Park and Marcotte datasets.]{Performance comparison on Park and Marcotte datasets. The ROC curves (top row) and PR curves (bottom row) for CV, C1, C2, and C3 datasets, from left to right. \label{fig_Park_Marcotte_curves}}
\end{figure}

The prediction performance on CV and C1 is very similar. The performance decreases from C1 to C2 and again to C3, both for ROC and PR curves. This is expected due to the way the datasets are constructed. The ROC curves do not distinguish very well between the prediction performance of the five methods. The difference is more clear in the PR curves. The SPRINT curve is almost always on top, especially at the beginning of the curve, where it matters the most for prediction. Ding's and Martin's are very close for CV and C1 datasets, followed by PIPE2. For C2 and C3 tests, the performance of Ding's and Martin's programs deteriorates and PIPE2 advances in second position.
\subsection{Comparative Analysis on Seven Human Datasets}
For a comprehensive comparison, we have compared the top four programs on six datasets, computed as mentioned above from six databases: Biogrid, HPRD Release 9, InnateDB (experimentally validated and manually curated PPIs), IntAct, and MINT. Since the prediction on the CV datasets is similar with C1, we use only C1, C2 and C3 datasets.

For the purpose of predicting new PPIs, the behaviour at high specificity is important. We therefore compare the sensitivity, precision and $F_1$-score for several high specificity values. The table with all values is given in the supplementary material. We present here in Table~\ref{table_cutoff_avg} the average values for each dataset type (C1, C2, and C3) over all datasets for each specificity value. At the bottom of the table we give also the average over all three dataset types. The performance of SPRINT with respect to all three measures, sensitivity, precision, and $F_1$-score is the highest. Only Ding comes close for C1 datasets. the overall average of SPRINT is much higher than Ding's. PIPE2 comes third and Martin last. The performance of PIPE2 decreases much less from C1 to C3 compared with Ding's. It should be noted that a weighted overall average, where the contribution of each dataset type C1,2,3 is proportional with its share of the general population, would place PIPE2 slightly ahead of Ding.

The area under the ROC and PR curves is given in Table~\ref{table_AUC} for all seven datasets, including the C1-, C2-, and C3-average, as well as the overall average across types. Ding is the winner for the C1 tests and SPRINT is the winner for the C2 and C3 tests. In the overall average, SPRINT comes on top. Martin is third and PIPE2 last.

All ROC and PR curves are included in the supplementary material.

\begin{table} 
  \centering
  
\caption[Performance comparison at high specificity]{Performance comparison at high specificity. Sensitivity, precision, and $F_1$-score averages for seven datasets are given for each dataset type C1, C2 and C3, as well as overall averages across types. Darker colours represent better results. The best results are in bold. \label{table_cutoff_avg}} 
\fontsize{9.5}{8}\selectfont
\begin{adjustwidth}{-2.5cm}{}
    \begin{tabular}{|@{}c@{}|@{\ }c@{\ }|cccc|cccc|cccc|}
    \toprule
    \multirow{2}[4]{*}{\textbf{Dataset}} & \multirow{2}[4]{*}{\textbf{Specificity}} & \multicolumn{4}{c|}{\textbf{Sensitivity}} & \multicolumn{4}{c|}{\textbf{Precision}} & \multicolumn{4}{c|}{\textbf{F1-score}} \\
\cmidrule{3-14}          &       & \textbf{$\!\!\!$Martin$\!\!\!$} & \textbf{$\!\!\!$PIPE2$\!\!\!$} & \textbf{$\!\!\!$Ding$\!\!\!$} & \textbf{$\!\!\!$SPRINT$\!\!\!$} & \textbf{$\!\!\!$Martin$\!\!\!$} & \textbf{$\!\!\!$PIPE2$\!\!\!$} & \textbf{$\!\!\!$Ding$\!\!\!$} & \textbf{$\!\!\!$SPRINT$\!\!\!$} & \textbf{$\!\!\!$Martin$\!\!\!$} & \textbf{$\!\!\!$PIPE2$\!\!\!$} & \textbf{$\!\!\!$Ding$\!\!\!$} & \textbf{$\!\!\!$SPRINT$\!\!\!$} \\
    \midrule
    \multirow{5}[2]{*}{\begin{tabular}{c}\textbf{C1}\\ \textbf{average}\end{tabular}} & 99.95\% & \cellcolor[rgb]{ .886,  .937,  .855} 6.07 & \cellcolor[rgb]{ .773,  .851,  .722} 7.60 & \cellcolor[rgb]{ .439,  .596,  .337} 11.93 & \cellcolor[rgb]{ .329,  .51,  .208} 13.35 & \cellcolor[rgb]{ .373,  .545,  .259} 98.52 & \cellcolor[rgb]{ .357,  .533,  .239} 98.82 & \cellcolor[rgb]{ .886,  .937,  .855} 88.05 & \cellcolor[rgb]{ .329,  .51,  .208} 99.37 & \cellcolor[rgb]{ .886,  .937,  .855} 11.06 & \cellcolor[rgb]{ .773,  .851,  .722} 13.55 & \cellcolor[rgb]{ .451,  .604,  .349} 20.39 & \cellcolor[rgb]{ .329,  .51,  .208} 22.93 \\
          & 99.90\% & \cellcolor[rgb]{ .886,  .937,  .855} 6.53 & \cellcolor[rgb]{ .729,  .82,  .675} 9.20 & \cellcolor[rgb]{ .431,  .588,  .325} 14.24 & \cellcolor[rgb]{ .329,  .51,  .208} 15.91 & \cellcolor[rgb]{ .455,  .608,  .353} 97.36 & \cellcolor[rgb]{ .376,  .545,  .263} 98.61 & \cellcolor[rgb]{ .886,  .937,  .855} 90.65 & \cellcolor[rgb]{ .329,  .51,  .208} 99.29 & \cellcolor[rgb]{ .886,  .937,  .855} 11.88 & \cellcolor[rgb]{ .722,  .812,  .663} 16.45 & \cellcolor[rgb]{ .439,  .596,  .333} 24.10 & \cellcolor[rgb]{ .329,  .51,  .208} 27.03 \\
          & 99.50\% & \cellcolor[rgb]{ .886,  .937,  .855} 17.27 & \cellcolor[rgb]{ .706,  .8,  .643} 21.41 & \cellcolor[rgb]{ .329,  .51,  .208} 29.90 & \cellcolor[rgb]{ .349,  .525,  .231} 29.50 & \cellcolor[rgb]{ .886,  .937,  .855} 96.66 & \cellcolor[rgb]{ .596,  .718,  .518} 97.52 & \cellcolor[rgb]{ .365,  .537,  .247} 98.20 & \cellcolor[rgb]{ .329,  .51,  .208} 98.30 & \cellcolor[rgb]{ .886,  .937,  .855} 28.62 & \cellcolor[rgb]{ .682,  .78,  .62} 34.73 & \cellcolor[rgb]{ .333,  .514,  .212} 45.19 & \cellcolor[rgb]{ .329,  .51,  .208} 45.22 \\
          & 99.00\% & \cellcolor[rgb]{ .886,  .937,  .855} 25.48 & \cellcolor[rgb]{ .765,  .843,  .714} 28.73 & \cellcolor[rgb]{ .384,  .553,  .271} 38.72 & \cellcolor[rgb]{ .329,  .51,  .208} 40.14 & \cellcolor[rgb]{ .886,  .937,  .855} 95.55 & \cellcolor[rgb]{ .647,  .753,  .576} 96.40 & \cellcolor[rgb]{ .396,  .561,  .286} 97.28 & \cellcolor[rgb]{ .329,  .51,  .208} 97.52 & \cellcolor[rgb]{ .886,  .937,  .855} 39.14 & \cellcolor[rgb]{ .741,  .827,  .686} 43.69 & \cellcolor[rgb]{ .392,  .557,  .278} 54.69 & \cellcolor[rgb]{ .329,  .51,  .208} 56.58 \\
          & 95.00\% & \cellcolor[rgb]{ .659,  .761,  .588} 55.35 & \cellcolor[rgb]{ .886,  .937,  .855} 48.07 & \cellcolor[rgb]{ .329,  .51,  .208} 65.68 & \cellcolor[rgb]{ .447,  .6,  .345} 62.02 & \cellcolor[rgb]{ .616,  .729,  .537} 91.44 & \cellcolor[rgb]{ .886,  .937,  .855} 90.19 & \cellcolor[rgb]{ .329,  .51,  .208} 92.72 & \cellcolor[rgb]{ .4,  .565,  .29} 92.41 & \cellcolor[rgb]{ .643,  .753,  .573} 68.37 & \cellcolor[rgb]{ .886,  .937,  .855} 62.09 & \cellcolor[rgb]{ .329,  .51,  .208} 76.41 & \cellcolor[rgb]{ .427,  .588,  .322} 73.90 \\
    \midrule
    \multirow{5}[2]{*}{\begin{tabular}{c}\textbf{C2}\\ \textbf{average}\end{tabular}} & 99.95\% & \cellcolor[rgb]{ .886,  .937,  .855} 5.55 & \cellcolor[rgb]{ .71,  .804,  .651} 10.65 & \cellcolor[rgb]{ .761,  .839,  .706} 9.22 & \cellcolor[rgb]{ .329,  .51,  .208} 21.45 & \cellcolor[rgb]{ .886,  .937,  .855} 96.33 & \cellcolor[rgb]{ .365,  .537,  .247} 99.42 & \cellcolor[rgb]{ .62,  .733,  .541} 97.92 & \cellcolor[rgb]{ .329,  .51,  .208} 99.62 & \cellcolor[rgb]{ .886,  .937,  .855} 9.78 & \cellcolor[rgb]{ .671,  .773,  .604} 18.91 & \cellcolor[rgb]{ .773,  .851,  .722} 14.69 & \cellcolor[rgb]{ .329,  .51,  .208} 33.16 \\
          & 99.90\% & \cellcolor[rgb]{ .886,  .937,  .855} 5.88 & \cellcolor[rgb]{ .718,  .808,  .659} 11.28 & \cellcolor[rgb]{ .765,  .843,  .714} 9.78 & \cellcolor[rgb]{ .329,  .51,  .208} 23.40 & \cellcolor[rgb]{ .886,  .937,  .855} 93.66 & \cellcolor[rgb]{ .369,  .541,  .251} 98.96 & \cellcolor[rgb]{ .647,  .753,  .576} 96.11 & \cellcolor[rgb]{ .329,  .51,  .208} 99.34 & \cellcolor[rgb]{ .886,  .937,  .855} 10.40 & \cellcolor[rgb]{ .682,  .78,  .616} 19.98 & \cellcolor[rgb]{ .773,  .851,  .722} 15.70 & \cellcolor[rgb]{ .329,  .51,  .208} 36.08 \\
          & 99.50\% & \cellcolor[rgb]{ .886,  .937,  .855} 11.73 & \cellcolor[rgb]{ .682,  .78,  .616} 19.52 & \cellcolor[rgb]{ .761,  .839,  .706} 16.59 & \cellcolor[rgb]{ .329,  .51,  .208} 32.73 & \cellcolor[rgb]{ .886,  .937,  .855} 93.86 & \cellcolor[rgb]{ .475,  .62,  .376} 97.11 & \cellcolor[rgb]{ .859,  .914,  .82} 94.11 & \cellcolor[rgb]{ .329,  .51,  .208} 98.22 & \cellcolor[rgb]{ .886,  .937,  .855} 20.17 & \cellcolor[rgb]{ .651,  .757,  .584} 31.86 & \cellcolor[rgb]{ .757,  .839,  .706} 26.59 & \cellcolor[rgb]{ .329,  .51,  .208} 47.77 \\
          & 99.00\% & \cellcolor[rgb]{ .886,  .937,  .855} 15.03 & \cellcolor[rgb]{ .643,  .753,  .573} 24.93 & \cellcolor[rgb]{ .702,  .796,  .643} 22.55 & \cellcolor[rgb]{ .329,  .51,  .208} 37.60 & \cellcolor[rgb]{ .886,  .937,  .855} 91.85 & \cellcolor[rgb]{ .482,  .627,  .388} 95.64 & \cellcolor[rgb]{ .71,  .804,  .651} 93.52 & \cellcolor[rgb]{ .329,  .51,  .208} 97.07 & \cellcolor[rgb]{ .886,  .937,  .855} 25.26 & \cellcolor[rgb]{ .616,  .729,  .541} 38.84 & \cellcolor[rgb]{ .694,  .788,  .631} 34.94 & \cellcolor[rgb]{ .329,  .51,  .208} 52.97 \\
          & 95.00\% & \cellcolor[rgb]{ .886,  .937,  .855} 37.41 & \cellcolor[rgb]{ .765,  .843,  .71} 40.95 & \cellcolor[rgb]{ .663,  .765,  .592} 43.83 & \cellcolor[rgb]{ .329,  .51,  .208} 53.17 & \cellcolor[rgb]{ .886,  .937,  .855} 86.45 & \cellcolor[rgb]{ .631,  .741,  .561} 88.43 & \cellcolor[rgb]{ .655,  .757,  .584} 88.27 & \cellcolor[rgb]{ .329,  .51,  .208} 90.76 & \cellcolor[rgb]{ .886,  .937,  .855} 51.17 & \cellcolor[rgb]{ .733,  .82,  .678} 55.33 & \cellcolor[rgb]{ .647,  .753,  .576} 57.69 & \cellcolor[rgb]{ .329,  .51,  .208} 66.18 \\
    \midrule
    \multirow{5}[2]{*}{\begin{tabular}{c}\textbf{C3}\\ \textbf{average}\end{tabular}} & 99.95\% & \cellcolor[rgb]{ .886,  .937,  .855} 1.04 & \cellcolor[rgb]{ .847,  .91,  .812} 1.46 & \cellcolor[rgb]{ .851,  .91,  .812} 1.44 & \cellcolor[rgb]{ .329,  .51,  .208} 6.96 & \cellcolor[rgb]{ .663,  .769,  .596} 94.80 & \cellcolor[rgb]{ .761,  .843,  .71} 93.56 & \cellcolor[rgb]{ .886,  .937,  .855} 91.97 & \cellcolor[rgb]{ .329,  .51,  .208} 99.01 & \cellcolor[rgb]{ .886,  .937,  .855} 2.05 & \cellcolor[rgb]{ .847,  .906,  .808} 2.85 & \cellcolor[rgb]{ .851,  .91,  .812} 2.78 & \cellcolor[rgb]{ .329,  .51,  .208} 12.85 \\
          & 99.90\% & \cellcolor[rgb]{ .886,  .937,  .855} 1.20 & \cellcolor[rgb]{ .843,  .902,  .804} 1.78 & \cellcolor[rgb]{ .851,  .91,  .816} 1.65 & \cellcolor[rgb]{ .329,  .51,  .208} 8.04 & \cellcolor[rgb]{ .635,  .745,  .565} 91.31 & \cellcolor[rgb]{ .706,  .8,  .643} 89.73 & \cellcolor[rgb]{ .886,  .937,  .855} 85.41 & \cellcolor[rgb]{ .329,  .51,  .208} 98.50 & \cellcolor[rgb]{ .886,  .937,  .855} 2.37 & \cellcolor[rgb]{ .839,  .902,  .8} 3.46 & \cellcolor[rgb]{ .851,  .91,  .816} 3.18 & \cellcolor[rgb]{ .329,  .51,  .208} 14.76 \\
          & 99.50\% & \cellcolor[rgb]{ .886,  .937,  .855} 4.12 & \cellcolor[rgb]{ .867,  .922,  .831} 4.74 & \cellcolor[rgb]{ .859,  .918,  .824} 4.92 & \cellcolor[rgb]{ .329,  .51,  .208} 19.50 & \cellcolor[rgb]{ .859,  .918,  .824} 85.62 & \cellcolor[rgb]{ .694,  .792,  .631} 89.05 & \cellcolor[rgb]{ .886,  .937,  .855} 85.01 & \cellcolor[rgb]{ .329,  .51,  .208} 96.63 & \cellcolor[rgb]{ .886,  .937,  .855} 7.83 & \cellcolor[rgb]{ .863,  .918,  .827} 8.96 & \cellcolor[rgb]{ .859,  .918,  .824} 9.03 & \cellcolor[rgb]{ .329,  .51,  .208} 31.65 \\
          & 99.00\% & \cellcolor[rgb]{ .875,  .929,  .839} 7.40 & \cellcolor[rgb]{ .796,  .867,  .749} 9.89 & \cellcolor[rgb]{ .886,  .937,  .855} 6.92 & \cellcolor[rgb]{ .329,  .51,  .208} 24.81 & \cellcolor[rgb]{ .851,  .91,  .812} 83.64 & \cellcolor[rgb]{ .678,  .776,  .612} 87.41 & \cellcolor[rgb]{ .886,  .937,  .855} 82.80 & \cellcolor[rgb]{ .329,  .51,  .208} 94.99 & \cellcolor[rgb]{ .867,  .922,  .831} 13.51 & \cellcolor[rgb]{ .784,  .859,  .737} 17.32 & \cellcolor[rgb]{ .886,  .937,  .855} 12.48 & \cellcolor[rgb]{ .329,  .51,  .208} 38.32 \\
          & 95.00\% & \cellcolor[rgb]{ .867,  .922,  .831} 24.82 & \cellcolor[rgb]{ .776,  .851,  .725} 27.35 & \cellcolor[rgb]{ .886,  .937,  .855} 24.18 & \cellcolor[rgb]{ .329,  .51,  .208} 39.79 & \cellcolor[rgb]{ .886,  .937,  .855} 80.99 & \cellcolor[rgb]{ .769,  .847,  .718} 82.36 & \cellcolor[rgb]{ .878,  .929,  .843} 81.13 & \cellcolor[rgb]{ .329,  .51,  .208} 87.38 & \cellcolor[rgb]{ .859,  .918,  .824} 37.59 & \cellcolor[rgb]{ .773,  .851,  .722} 40.28 & \cellcolor[rgb]{ .886,  .937,  .855} 36.73 & \cellcolor[rgb]{ .329,  .51,  .208} 53.82 \\
    \midrule
    \multirow{5}[2]{*}{\begin{tabular}{c}\textbf{Overall}\\ \textbf{AVERAGE}\end{tabular}} & 99.95\% & \cellcolor[rgb]{ .886,  .937,  .855} 4.22 & \cellcolor[rgb]{ .753,  .835,  .702} 6.57 & \cellcolor[rgb]{ .698,  .792,  .635} 7.53 & \cellcolor[rgb]{ .329,  .51,  .208} 13.92 & \cellcolor[rgb]{ .565,  .69,  .478} 96.55 & \cellcolor[rgb]{ .502,  .643,  .408} 97.27 & \cellcolor[rgb]{ .886,  .937,  .855} 92.65 & \cellcolor[rgb]{ .329,  .51,  .208} 99.33 & \cellcolor[rgb]{ .886,  .937,  .855} 7.63 & \cellcolor[rgb]{ .737,  .824,  .682} 11.77 & \cellcolor[rgb]{ .706,  .8,  .647} 12.62 & \cellcolor[rgb]{ .329,  .51,  .208} 22.98 \\
          & 99.90\% & \cellcolor[rgb]{ .886,  .937,  .855} 4.54 & \cellcolor[rgb]{ .745,  .831,  .69} 7.42 & \cellcolor[rgb]{ .69,  .788,  .627} 8.56 & \cellcolor[rgb]{ .329,  .51,  .208} 15.79 & \cellcolor[rgb]{ .663,  .765,  .592} 94.11 & \cellcolor[rgb]{ .549,  .678,  .463} 95.77 & \cellcolor[rgb]{ .886,  .937,  .855} 90.73 & \cellcolor[rgb]{ .329,  .51,  .208} 99.04 & \cellcolor[rgb]{ .886,  .937,  .855} 8.22 & \cellcolor[rgb]{ .729,  .816,  .671} 13.30 & \cellcolor[rgb]{ .698,  .792,  .635} 14.33 & \cellcolor[rgb]{ .329,  .51,  .208} 25.96 \\
          & 99.50\% & \cellcolor[rgb]{ .886,  .937,  .855} 11.04 & \cellcolor[rgb]{ .745,  .827,  .69} 15.23 & \cellcolor[rgb]{ .678,  .776,  .612} 17.14 & \cellcolor[rgb]{ .329,  .51,  .208} 27.24 & \cellcolor[rgb]{ .886,  .937,  .855} 92.05 & \cellcolor[rgb]{ .643,  .749,  .569} 94.56 & \cellcolor[rgb]{ .851,  .91,  .812} 92.44 & \cellcolor[rgb]{ .329,  .51,  .208} 97.71 & \cellcolor[rgb]{ .886,  .937,  .855} 18.87 & \cellcolor[rgb]{ .733,  .82,  .678} 25.18 & \cellcolor[rgb]{ .69,  .788,  .627} 26.94 & \cellcolor[rgb]{ .329,  .51,  .208} 41.54 \\
          & 99.00\% & \cellcolor[rgb]{ .886,  .937,  .855} 15.97 & \cellcolor[rgb]{ .729,  .816,  .671} 21.19 & \cellcolor[rgb]{ .682,  .78,  .616} 22.73 & \cellcolor[rgb]{ .329,  .51,  .208} 34.18 & \cellcolor[rgb]{ .886,  .937,  .855} 90.35 & \cellcolor[rgb]{ .635,  .745,  .565} 93.15 & \cellcolor[rgb]{ .812,  .878,  .769} 91.20 & \cellcolor[rgb]{ .329,  .51,  .208} 96.52 & \cellcolor[rgb]{ .886,  .937,  .855} 25.97 & \cellcolor[rgb]{ .714,  .804,  .655} 33.28 & \cellcolor[rgb]{ .694,  .792,  .631} 34.04 & \cellcolor[rgb]{ .329,  .51,  .208} 49.29 \\
          & 95.00\% & \cellcolor[rgb]{ .871,  .925,  .835} 39.19 & \cellcolor[rgb]{ .886,  .937,  .855} 38.79 & \cellcolor[rgb]{ .639,  .749,  .565} 44.56 & \cellcolor[rgb]{ .329,  .51,  .208} 51.66 & \cellcolor[rgb]{ .886,  .937,  .855} 86.30 & \cellcolor[rgb]{ .788,  .863,  .741} 86.99 & \cellcolor[rgb]{ .733,  .82,  .678} 87.37 & \cellcolor[rgb]{ .329,  .51,  .208} 90.18 & \cellcolor[rgb]{ .886,  .937,  .855} 52.38 & \cellcolor[rgb]{ .878,  .933,  .847} 52.57 & \cellcolor[rgb]{ .682,  .78,  .616} 56.94 & \cellcolor[rgb]{ .329,  .51,  .208} 64.63 \\
    \bottomrule
    \end{tabular}%
    \end{adjustwidth}
\end{table}%



\begin{table}[] 
%   \centering
\caption[Area under curves]{Area under curves. AUROC and AUPR curves are given for seven datasets and three types, C1, C2, C3, for each, as well as averages for each type and overall average across types. Darker colours represent better results. The best results are in bold. \label{table_AUC}} 
    \begin{tabular}{|c|cccc|cccc|}
    \toprule
    \multirow{3}[6]{*}{\textbf{Dataset}} & \multicolumn{4}{c|}{\textbf{AUROC}} & \multicolumn{4}{c|}{\textbf{AUPR}} \\
\cmidrule{2-9}          & \textbf{$\!\!\!$Martin$\!\!\!$} & \textbf{$\!\!\!$PIPE2$\!\!\!$} & \textbf{$\!\!\!$Ding$\!\!\!$} & \textbf{$\!\!\!$SPRINT$\!\!\!$} & \textbf{$\!\!\!$Martin$\!\!\!$} & \textbf{$\!\!\!$PIPE2$\!\!\!$} & \textbf{$\!\!\!$Ding$\!\!\!$} & \textbf{$\!\!\!$SPRINT$\!\!\!$} \\
\cmidrule{2-9}          & \multicolumn{8}{c|}{\textbf{C1}} \\
    \midrule
    \textbf{Biogrid} & \cellcolor[rgb]{ .549,  .678,  .463} 87.54 & \cellcolor[rgb]{ .886,  .937,  .855} 79.01 & \cellcolor[rgb]{ .329,  .51,  .208} 93.06 & \cellcolor[rgb]{ .529,  .663,  .439} 88.11 & \cellcolor[rgb]{ .592,  .714,  .514} 87.20 & \cellcolor[rgb]{ .886,  .937,  .855} 80.52 & \cellcolor[rgb]{ .329,  .51,  .208} 93.08 & \cellcolor[rgb]{ .502,  .643,  .408} 89.24 \\
    \textbf{HPRD} & \cellcolor[rgb]{ .51,  .651,  .42} 86.83 & \cellcolor[rgb]{ .886,  .937,  .855} 81.53 & \cellcolor[rgb]{ .329,  .51,  .208} 89.34 & \cellcolor[rgb]{ .514,  .655,  .424} 86.76 & \cellcolor[rgb]{ .639,  .749,  .569} 86.93 & \cellcolor[rgb]{ .886,  .937,  .855} 84.31 & \cellcolor[rgb]{ .329,  .51,  .208} 90.20 & \cellcolor[rgb]{ .416,  .576,  .306} 89.32 \\
    \textbf{Innate\_Exp} & \cellcolor[rgb]{ .537,  .671,  .451} 90.18 & \cellcolor[rgb]{ .886,  .937,  .855} 83.98 & \cellcolor[rgb]{ .329,  .51,  .208} 93.83 & \cellcolor[rgb]{ .471,  .62,  .373} 91.34 & \cellcolor[rgb]{ .576,  .702,  .494} 90.31 & \cellcolor[rgb]{ .886,  .937,  .855} 85.48 & \cellcolor[rgb]{ .329,  .51,  .208} 94.14 & \cellcolor[rgb]{ .455,  .604,  .353} 92.25 \\
    \textbf{Innate\_Man} & \cellcolor[rgb]{ .424,  .584,  .318} 94.11 & \cellcolor[rgb]{ .886,  .937,  .855} 90.26 & \cellcolor[rgb]{ .329,  .51,  .208} 94.89 & \cellcolor[rgb]{ .549,  .678,  .463} 93.09 & \cellcolor[rgb]{ .459,  .608,  .357} 94.93 & \cellcolor[rgb]{ .886,  .937,  .855} 92.22 & \cellcolor[rgb]{ .329,  .51,  .208} 95.73 & \cellcolor[rgb]{ .486,  .631,  .392} 94.75 \\
    \textbf{IntAct} & \cellcolor[rgb]{ .533,  .667,  .443} 88.02 & \cellcolor[rgb]{ .886,  .937,  .855} 80.72 & \cellcolor[rgb]{ .329,  .51,  .208} 92.18 & \cellcolor[rgb]{ .502,  .643,  .408} 88.69 & \cellcolor[rgb]{ .584,  .706,  .502} 87.51 & \cellcolor[rgb]{ .886,  .937,  .855} 81.68 & \cellcolor[rgb]{ .329,  .51,  .208} 92.31 & \cellcolor[rgb]{ .467,  .616,  .369} 89.71 \\
    \textbf{MINT} & \cellcolor[rgb]{ .478,  .624,  .38} 90.86 & \cellcolor[rgb]{ .886,  .937,  .855} 83.41 & \cellcolor[rgb]{ .329,  .51,  .208} 93.54 & \cellcolor[rgb]{ .58,  .702,  .498} 89.03 & \cellcolor[rgb]{ .537,  .671,  .451} 91.08 & \cellcolor[rgb]{ .886,  .937,  .855} 85.93 & \cellcolor[rgb]{ .329,  .51,  .208} 94.11 & \cellcolor[rgb]{ .533,  .667,  .447} 91.13 \\
    \textbf{Park \& Marcotte} & \cellcolor[rgb]{ .416,  .576,  .31} 81.49 & \cellcolor[rgb]{ .886,  .937,  .855} 76.74 & \cellcolor[rgb]{ .365,  .537,  .251} 82.00 & \cellcolor[rgb]{ .329,  .51,  .208} 82.35 & \cellcolor[rgb]{ .643,  .749,  .573} 82.32 & \cellcolor[rgb]{ .886,  .937,  .855} 79.90 & \cellcolor[rgb]{ .573,  .698,  .49} 83.00 & \cellcolor[rgb]{ .329,  .51,  .208} 85.39 \\
    \midrule
          & \multicolumn{8}{c|}{\textbf{C2}} \\
    \midrule
    \textbf{Biogrid} & \cellcolor[rgb]{ .627,  .737,  .553} 81.33 & \cellcolor[rgb]{ .886,  .937,  .855} 76.66 & \cellcolor[rgb]{ .329,  .51,  .208} 86.57 & \cellcolor[rgb]{ .439,  .592,  .333} 84.67 & \cellcolor[rgb]{ .714,  .808,  .655} 80.76 & \cellcolor[rgb]{ .886,  .937,  .855} 78.25 & \cellcolor[rgb]{ .345,  .522,  .224} 86.12 & \cellcolor[rgb]{ .329,  .51,  .208} 86.30 \\
    \textbf{HPRD} & \cellcolor[rgb]{ .675,  .776,  .608} 83.30 & \cellcolor[rgb]{ .886,  .937,  .855} 81.55 & \cellcolor[rgb]{ .49,  .635,  .396} 84.78 & \cellcolor[rgb]{ .329,  .51,  .208} 86.09 & \cellcolor[rgb]{ .886,  .937,  .855} 82.85 & \cellcolor[rgb]{ .773,  .851,  .725} 83.98 & \cellcolor[rgb]{ .686,  .784,  .624} 84.85 & \cellcolor[rgb]{ .329,  .51,  .208} 88.37 \\
    \textbf{Innate\_Exp} & \cellcolor[rgb]{ .71,  .804,  .651} 83.96 & \cellcolor[rgb]{ .886,  .937,  .855} 81.46 & \cellcolor[rgb]{ .424,  .584,  .318} 87.98 & \cellcolor[rgb]{ .329,  .51,  .208} 89.31 & \cellcolor[rgb]{ .804,  .875,  .761} 83.74 & \cellcolor[rgb]{ .886,  .937,  .855} 82.57 & \cellcolor[rgb]{ .506,  .647,  .416} 87.91 & \cellcolor[rgb]{ .329,  .51,  .208} 90.37 \\
    \textbf{Innate\_Man} & \cellcolor[rgb]{ .639,  .749,  .565} 85.87 & \cellcolor[rgb]{ .886,  .937,  .855} 84.43 & \cellcolor[rgb]{ .835,  .898,  .796} 84.74 & \cellcolor[rgb]{ .329,  .51,  .208} 87.64 & \cellcolor[rgb]{ .784,  .859,  .733} 87.71 & \cellcolor[rgb]{ .867,  .922,  .831} 87.22 & \cellcolor[rgb]{ .886,  .937,  .855} 87.10 & \cellcolor[rgb]{ .329,  .51,  .208} 90.33 \\
    \textbf{IntAct} & \cellcolor[rgb]{ .608,  .722,  .529} 81.68 & \cellcolor[rgb]{ .886,  .937,  .855} 77.64 & \cellcolor[rgb]{ .329,  .51,  .208} 85.63 & \cellcolor[rgb]{ .506,  .643,  .412} 83.14 & \cellcolor[rgb]{ .725,  .816,  .671} 80.68 & \cellcolor[rgb]{ .886,  .937,  .855} 78.69 & \cellcolor[rgb]{ .361,  .533,  .243} 85.20 & \cellcolor[rgb]{ .329,  .51,  .208} 85.58 \\
    \textbf{MINT} & \cellcolor[rgb]{ .384,  .553,  .271} 86.66 & \cellcolor[rgb]{ .886,  .937,  .855} 81.76 & \cellcolor[rgb]{ .329,  .51,  .208} 87.17 & \cellcolor[rgb]{ .431,  .588,  .325} 86.20 & \cellcolor[rgb]{ .6,  .718,  .522} 86.37 & \cellcolor[rgb]{ .886,  .937,  .855} 84.08 & \cellcolor[rgb]{ .459,  .608,  .357} 87.47 & \cellcolor[rgb]{ .329,  .51,  .208} 88.47 \\
    \textbf{Park \& Marcotte} & \cellcolor[rgb]{ .82,  .886,  .776} 60.67 & \cellcolor[rgb]{ .51,  .647,  .416} 63.76 & \cellcolor[rgb]{ .886,  .937,  .855} 60.00 & \cellcolor[rgb]{ .329,  .51,  .208} 65.52 & \cellcolor[rgb]{ .867,  .922,  .831} 60.43 & \cellcolor[rgb]{ .486,  .631,  .388} 67.41 & \cellcolor[rgb]{ .886,  .937,  .855} 60.00 & \cellcolor[rgb]{ .329,  .51,  .208} 70.25 \\
    \midrule
          & \multicolumn{8}{c|}{\textbf{C3}} \\
    \midrule
    \textbf{Biogrid} & \cellcolor[rgb]{ .565,  .69,  .482} 76.20 & \cellcolor[rgb]{ .886,  .937,  .855} 71.38 & \cellcolor[rgb]{ .365,  .537,  .251} 79.16 & \cellcolor[rgb]{ .329,  .51,  .208} 79.67 & \cellcolor[rgb]{ .639,  .749,  .565} 74.89 & \cellcolor[rgb]{ .886,  .937,  .855} 70.25 & \cellcolor[rgb]{ .51,  .651,  .42} 77.24 & \cellcolor[rgb]{ .329,  .51,  .208} 80.59 \\
    \textbf{HPRD} & \cellcolor[rgb]{ .678,  .776,  .612} 79.46 & \cellcolor[rgb]{ .886,  .937,  .855} 77.14 & \cellcolor[rgb]{ .855,  .914,  .816} 77.51 & \cellcolor[rgb]{ .329,  .51,  .208} 83.27 & \cellcolor[rgb]{ .706,  .8,  .647} 78.51 & \cellcolor[rgb]{ .718,  .808,  .659} 78.28 & \cellcolor[rgb]{ .886,  .937,  .855} 75.32 & \cellcolor[rgb]{ .329,  .51,  .208} 85.08 \\
    \textbf{Innate\_Exp} & \cellcolor[rgb]{ .765,  .843,  .71} 78.10 & \cellcolor[rgb]{ .886,  .937,  .855} 75.89 & \cellcolor[rgb]{ .616,  .729,  .541} 80.69 & \cellcolor[rgb]{ .329,  .51,  .208} 85.70 & \cellcolor[rgb]{ .784,  .859,  .733} 76.65 & \cellcolor[rgb]{ .886,  .937,  .855} 74.42 & \cellcolor[rgb]{ .694,  .788,  .631} 78.55 & \cellcolor[rgb]{ .329,  .51,  .208} 86.23 \\
    \textbf{Innate\_Man} & \cellcolor[rgb]{ .584,  .706,  .502} 71.75 & \cellcolor[rgb]{ .506,  .647,  .412} 73.25 & \cellcolor[rgb]{ .886,  .937,  .855} 65.96 & \cellcolor[rgb]{ .329,  .51,  .208} 76.57 & \cellcolor[rgb]{ .612,  .725,  .533} 73.49 & \cellcolor[rgb]{ .549,  .678,  .463} 74.95 & \cellcolor[rgb]{ .886,  .937,  .855} 66.81 & \cellcolor[rgb]{ .329,  .51,  .208} 80.17 \\
    \textbf{IntAct} & \cellcolor[rgb]{ .533,  .667,  .443} 76.94 & \cellcolor[rgb]{ .886,  .937,  .855} 73.61 & \cellcolor[rgb]{ .329,  .51,  .208} 78.81 & \cellcolor[rgb]{ .8,  .871,  .753} 74.44 & \cellcolor[rgb]{ .69,  .788,  .627} 74.88 & \cellcolor[rgb]{ .886,  .937,  .855} 73.11 & \cellcolor[rgb]{ .561,  .686,  .475} 76.03 & \cellcolor[rgb]{ .329,  .51,  .208} 78.08 \\
    \textbf{MINT} & \cellcolor[rgb]{ .49,  .635,  .396} 81.25 & \cellcolor[rgb]{ .886,  .937,  .855} 78.06 & \cellcolor[rgb]{ .78,  .855,  .729} 78.94 & \cellcolor[rgb]{ .329,  .51,  .208} 82.54 & \cellcolor[rgb]{ .667,  .769,  .6} 80.07 & \cellcolor[rgb]{ .729,  .816,  .671} 79.28 & \cellcolor[rgb]{ .886,  .937,  .855} 77.14 & \cellcolor[rgb]{ .329,  .51,  .208} 84.55 \\
    \textbf{Park \& Marcotte} & \cellcolor[rgb]{ .757,  .835,  .702} 57.86 & \cellcolor[rgb]{ .596,  .714,  .514} 58.90 & \cellcolor[rgb]{ .886,  .937,  .855} 57.00 & \cellcolor[rgb]{ .329,  .51,  .208} 60.60 & \cellcolor[rgb]{ .808,  .878,  .765} 57.07 & \cellcolor[rgb]{ .604,  .722,  .525} 59.84 & \cellcolor[rgb]{ .886,  .937,  .855} 56.00 & \cellcolor[rgb]{ .329,  .51,  .208} 63.49 \\
    \midrule
          & \multicolumn{8}{c|}{\textbf{AVERAGES}} \\
    \midrule
    \textbf{C1 average} & \cellcolor[rgb]{ .506,  .647,  .412} 88.43 & \cellcolor[rgb]{ .886,  .937,  .855} 82.24 & \cellcolor[rgb]{ .329,  .51,  .208} 91.26 & \cellcolor[rgb]{ .502,  .643,  .408} 88.48 & \cellcolor[rgb]{ .569,  .694,  .486} 88.61 & \cellcolor[rgb]{ .886,  .937,  .855} 84.29 & \cellcolor[rgb]{ .329,  .51,  .208} 91.80 & \cellcolor[rgb]{ .447,  .6,  .341} 90.26 \\
\cmidrule{1-1}    \textbf{C2 average} & \cellcolor[rgb]{ .631,  .741,  .561} 80.50 & \cellcolor[rgb]{ .886,  .937,  .855} 78.18 & \cellcolor[rgb]{ .42,  .58,  .314} 82.41 & \cellcolor[rgb]{ .329,  .51,  .208} 83.23 & \cellcolor[rgb]{ .882,  .937,  .851} 80.36 & \cellcolor[rgb]{ .886,  .937,  .855} 80.32 & \cellcolor[rgb]{ .643,  .753,  .573} 82.67 & \cellcolor[rgb]{ .329,  .51,  .208} 85.67 \\
\cmidrule{1-1}    \textbf{C3 average} & \cellcolor[rgb]{ .675,  .773,  .608} 74.51 & \cellcolor[rgb]{ .886,  .937,  .855} 72.60 & \cellcolor[rgb]{ .729,  .816,  .671} 74.01 & \cellcolor[rgb]{ .329,  .51,  .208} 77.54 & \cellcolor[rgb]{ .796,  .867,  .749} 73.65 & \cellcolor[rgb]{ .855,  .914,  .82} 72.88 & \cellcolor[rgb]{ .886,  .937,  .855} 72.44 & \cellcolor[rgb]{ .329,  .51,  .208} 79.74 \\
    \midrule
    \textbf{Overall AVERAGE} & \cellcolor[rgb]{ .529,  .667,  .443} 81.15 & \cellcolor[rgb]{ .886,  .937,  .855} 77.67 & \cellcolor[rgb]{ .384,  .553,  .271} 82.56 & \cellcolor[rgb]{ .329,  .51,  .208} 83.08 & \cellcolor[rgb]{ .729,  .82,  .675} 80.87 & \cellcolor[rgb]{ .886,  .937,  .855} 79.16 & \cellcolor[rgb]{ .6,  .718,  .522} 82.30 & \cellcolor[rgb]{ .329,  .51,  .208} 85.22 \\
    \bottomrule
    \end{tabular}%
\end{table}%
\subsection{Comparative Analysis on Human Interactome Prediction}
The goal of all PPI prediction methods is to predict new interactions from existing reliable ones. That means, in practice we input all known interactions -- the entire interactome of an organism -- and predict new ones. Of the newly predicted interactions, only those that are the most likely to be true interactions are kept. 

For predicting the entire interactome, we need to predict the probability of interaction between any two proteins. Recall that for $P$ proteins, that means we need to consider $(1+P)\times P\div2$ protein pairs. For our 20,160 proteins, that is about 203 million potential interactions. %; 203,222,880
For example, predicting one pair per second results in over six years of computation time. 

We have tested the four programs, Martin's, PIPE2, Ding's, and SPRINT, on the entire human interactome, considering as given PPIs each of the six datasets in Table~\ref{table_datasets}. The tests were performed on a DELL PowerEdge R620 computer with 12 cores Intel Xeon at 2.0 GHz and 256 GB of RAM, running Linux Red Hat, CentOS 6.3. 

The time and memory values are shown in Table~\ref{table_interactome_time} for all three stages: preprocessing, training, and predicting. For each dataset, training is performed on all PPIs in that dataset and then predictions are made for all 203 million protein pairs. 

Note that PIPE2 and SPRINT do not require any training. Also, preprocessing is performed only once for all protein sequences. As long as no protein sequences are added, no preprocessing needs to be done. For SPRINT, we provide all necessary similarities for all reviewed human proteins in UniProt. If new protein sequences are added, the program has an option (``\texttt{-add}'') that is able to compute only the new similarities, which is very fast.

Therefore, the comparison is between predicting time of PIPE2 and SPRINT and training plus predicting time of Martin and Ding. PIPE2 and Martin are very slow and the predicting times are estimated by running the programs for 100 hours and then estimating according to the number of protein pairs left to process. Both take too long to be used on the entire human interactome.

Ding's program is faster than the other two but uses a large amount of memory. It ran out of 256GB of memory when training on the two largest datasets: Biogrid and InnateDB experimentally validated. It seems able to train on the IntAct dataset but it could not finish training in 14 days, which is the longest we can run a job on our system.

SPRINT is approximately five orders of magnitude faster than PIPE2 and Martin. It is over two orders of magnitude faster than Ding but this is based on the small datasets. The results on IntAct seem to indicate that the difference increases for large datasets. 

Another interesting property of SPRINT is that it appears to scale sublinearly with the size of the datasets, that is, the larger the datasets, the faster it runs (per PPI). This means SPRINT will continue to be fast as the datasets will grow, which it is to be expected.

It should be noted that SPRINT runs in parallel whereas the other are serial. Martin's and PIPE2 are much slower, so parallelizing the prediction would not make any difference. Ding's program on the other hand uses a considerable amout of time for training, which cannot be easily parallelized. The very large difference in speed is due to the fact that while Martin, PIPE2, and Ding consider one protein pair at the time, out of the 203 million, SPRINT simply computes all 203 million scores at the same time; see the Methods section for details.

In terms of memory, SPRINT requires a very modest amount of memory to predict. We successfully ran SPRINT on all entire human interactome tests in serial mode on an older MacBook (1.4GHz processor, 4 GB RAM); the running time was between 35 minutes for Innate manually curated to 11 hours for Biogriod. 

The comparison is more visually clear in Figure~\ref{fig_time_memory} where the time (in hours) and memory are plotted together for the four programs compared and those datasets for which we have either a value or at least an estimate. Note the logarithmic scale for time. The point with the highest memory for Ding's program (for the IntAct dataset) has time value fourteen days, which is the only lower bound we have. The real time may be  much larger. 


\begin{figure}[h!]
\centering
\includegraphics[width=8.5cm]{img/fig_time_memory.pdf}
\caption{Time and memory comparison for predicting the entire human interactome. \label{fig_time_memory}}
\end{figure}

\begin{table}[!h!] 
\centering
\caption[Human interactome comparison]{Human interactome comparison: running time and peak memory. The predicting time for Martin's and PIPE2 was estimated by running it for 100 hours and then estimating the total time according to the number of pairs left to predict. Note that PIPE2 and SPRINT do not require training as they are not using machine learning. For the entries marked with a dash, the program ran out of (256 GB) memory or ran for more than 14 days. Times marked with a dagger${}^\dag$ are estimated.\label{table_interactome_time}}
\begin{tabular}{@{}llrrrrrr@{}} \toprule
Dataset & Program & \multicolumn{3}{c}{Time (s)} &  \multicolumn{3}{|c}{Memory (GB)}   \\ \cmidrule{3-5} \cmidrule{6-8}
 &	& Preprocess & Train & Predict &  Preprocess & Train & Predict \\ \midrule
Biogrid & %%%%%%%%%%%%%%%%%%%%%%%%
	Martin & 32,400 & $>$ 1,209,600  & {\scriptsize  -- } & 2.5 & 6.1 & {\scriptsize  -- } \\ 
&	PIPE2 & 312,120 & {\scriptsize N$\!$/$\!$A} & ${}^\dag$1,150,675,200 & 2.1 & {\scriptsize N$\!$/$\!$A} & 18.9\\ 
&	Ding	   & 37,708 & {\scriptsize  -- } & {\scriptsize  -- }  & 3.3 &  $>$ 256 & {\scriptsize  -- }\\
&	SPRINT & 105,480 & {\scriptsize N$\!$/$\!$A} & 6,120 & 11.2 & {\scriptsize N$\!$/$\!$A} & 3.0 \\ \midrule
HPRD  & %%%%%%%%%%%%%%%%%%%%%%%% 
	Martin & 32,400 & 584,640 & ${}^\dag$107,222,400  & 2.5 & 3.2 & 1.5 \\ 
Release 9 &	PIPE2 & 312,120 & {\scriptsize N$\!$/$\!$A} & ${}^\dag$435,628,800 &  2.1 & {\scriptsize N$\!$/$\!$A} & 18.9 \\ 
&	Ding	   & 37,708 & 236,551 & 374,360 & 3.3 & 79.5 & 79.5\\
&	SPRINT & 105,480 & {\scriptsize N$\!$/$\!$A} & 1,257 & 11.2 & {\scriptsize N$\!$/$\!$A} & 3.0 \\ \midrule
Innate   & %%%%%%%%%%%%%%%%%%%%%%%%
	Martin & 32,400 & $>$ 1,209,600 & {\scriptsize  -- }  & 2.5 & 5.7 & {\scriptsize  -- } \\ 
experim. &	PIPE2 & 312,120 & {\scriptsize N$\!$/$\!$A} & ${}^\dag$872,294,400 &  2.1 &  {\scriptsize N$\!$/$\!$A} & 18.9 \\ 
validated &	Ding	   & 37,708 & {\scriptsize  -- } & {\scriptsize  -- }  & 3.3 &  $>$ 256 & {\scriptsize  -- } \\
&	SPRINT & 105,480 & {\scriptsize N$\!$/$\!$A} & 3,600 & 11.2 & {\scriptsize N$\!$/$\!$A} & 3.0  \\ \midrule
Innate  &  %%%%%%%%%%%%%%%%%%%%%%%%
	Martin & 32,400 & 26,280 & ${}^\dag$30,888,000 & 2.5 & 1.9 & 1.5 \\ 
manually &	PIPE2 & 312,120 & {\scriptsize N$\!$/$\!$A} & ${}^\dag$230,342,400  &  2.1 &  {\scriptsize N$\!$/$\!$A} & 18.9 \\ 
curated &	Ding	   & 37,708 & 55,532 & 285,323 & 3.3 & 25.4 & 25.4 \\
&	SPRINT & 105,480 & {\scriptsize N$\!$/$\!$A} & 930 & 11.2 & {\scriptsize N$\!$/$\!$A} & 3.0  \\ \midrule
IntAct & %%%%%%%%%%%%%%%%%%%%%%%%
	Martin & 32,400 & $>$ 1,209,600 & {\scriptsize  -- }    & 2.5 & 3.5 & {\scriptsize  -- } \\ 
&	PIPE2 & 312,120 & {\scriptsize N$\!$/$\!$A} & ${}^\dag$616,464,000 &  2.1 &  {\scriptsize N$\!$/$\!$A} & 18.9\\ 
&	Ding	   & 37,708 &  $>$ 1,209,600 & {\scriptsize  -- }   & 3.3 & 220 & {\scriptsize  -- } \\
&	SPRINT & 105,480 & {\scriptsize N$\!$/$\!$A} & 2,672 & 11.2 & {\scriptsize N$\!$/$\!$A} & 3.0  \\ \midrule
MINT & %%%%%%%%%%%%%%%%%%%%%%%%
	Martin & 32,400 & 101,160  & ${}^\dag$52,557,120 & 2.5 & 2.3 & 1.5\\ 
&	PIPE2 & 312,120 & {\scriptsize N$\!$/$\!$A} & ${}^\dag$372,902,400 &  2.1 &  {\scriptsize N$\!$/$\!$A} & 18.9 \\ 
&	Ding	   & 37,708 & 120,720 & 331,865 & 3.3 & 41.1 & 41.1\\
&	SPRINT & 105,480 & {\scriptsize N$\!$/$\!$A} & 952 & 11.2 & {\scriptsize N$\!$/$\!$A} & 3.0  \\ \bottomrule
\end{tabular}
\end{table}

\subsection{Availability}
SPRINT is freely available at  \texttt{https://github.com/lucian-ilie/SPRINT/}\\
Any restrictions to use by non-academics: None.\\
The UniProt protein sequences we used, precomputed similarities for these sequences, the datasets, and the top 1\% predicted PPIs for the entire human interactome can be found at \texttt{www.csd.uwo.ca/faculty/ilie/SPRINT/}.\\

\section{Conclusion}
We introduced our program SPRINT and comprehensively compared it with five state-of-the-art programs on seven human datasets. SPRINT is more accurate and running orders of magnitudes faster than the competing methods. The contributions of the SPRINT study are as follows.
First, an end-to-end PPI perdition program is provided, freely to the public. SPRINT is easy to use, and we hope it will make PPI prediction for entire interactomes a routine task. Second, the fast design and the implementation of sequence indexing, similarities detection and scoring PPI are made available. The HSP computation component can be easily used in connection with other tool, such as DELPHI \cite{li2020delphi}. Third, six new human PPI benchmark datasets are constructed and the entire human interactome prediction on each of them are also made available.
\chapter{Conclusion and Future Research \label{chap_4}}
We have described DELPHI and SPRINT, two bioinformatics programs for predicting PPI biding sites and predicting PPI. Both programs are more accurate than the state-of-the-art methods at the time of the publication. SPRINT is orders of magnitudes faster. 

In this chapter, we discuss some of the common practises in developing bioinformatics tools. We hope these techniques and tips are helpful to others. Then, future research areas are introduced.
\section{Common Deep Learning Practises in Bioinformatics}
As shown in Figure \ref{fig_deep_learning_step}, common deep learning application development steps include defining a problem, preparing data, designing and improving models, and result reporting. Unlike traditional algorithmic approach, most deep learning frameworks such as TensorFlow and PyTorch have a black box design where it is hard to debug line by line. Therefore, the development process is to some level, experience driven; one needs to detect problems by looking at the end results. 
\begin{figure}[h!]
\begin{center}
\includegraphics[height=9.5cm]{img/machine_learning_in_bioinf.png}
\caption[Common deep learning application development steps]{Common deep learning application development steps. From machinelearningmastery.com \label{fig_deep_learning_step}}
\end{center}
\end{figure}
\subsection{Data Preparation}
Data preparation is a step often neglected by engineers. Computer scientists tend to focus on the algorithmic aspect of the application, but a trend in deep learning is the increasing importance of having good data. 

Good data has several folds of meanings. First, good training and validation data is the key to train a good model. In general, in deep learning, the more data the better. However, too much data also implies longer training time, so it is also a trade off. Training data should be clean, consistent, and obtained from trustworthy source. Second, several gold standard testing data should be picked at the beginning of the development cycle. Enough attention should be focused on selecting good independent testing dataset. This includes using benchmark testing dataset from previous publications or design your own testing dataset.  Third, data similarities should be checked carefully at the very beginning. The training and validation data should be filtered to contain no similarities to testing data. There should also be no similarities between and among training and validation data. Many high impact publications now require the submitted machine learning manuscripts to have a separate section discussing how training and testing data are dissimilar. Precious time could be wasted if reviewers are concerned about the dataset similarities. Because this could mean revisit the whole development process.

\subsubsection{Comparative Analysis}
To perform comparative analysis, running competing methods is often needed. Sanity check is crucial but often omitted. Sanity check means by executing others' programs, one obtain identical result as reported in previous publications. This ensures running others' program correctly and evaluating result using same way. There are commonly seen mistake due to not having sanity checks. For example, evaluation metrics such as sensitivity, MCC are threshold dependent. That is, the original output of programs are regression values, and a threshold is needed to classify each value into a class, for instance, binding site and non-binding site. All programs should have a uniformed way of selecting the threshold (see Section \ref{sec_evaluation_scheme} as an example). Another common mistake is comparing the evaluation metrics on a complete testing dataset to the average of leave-one-out result on the same testing dataset.

\subsubsection{Improving Results}
After the data is ready, trying different architectures, tuning parameters, and improving results often take several development cycles. It is worthwhile spending time on the infrastructure code. Good code infrastructure, such as scripting, clear system design, makes it easier to change sub-component and try ideas quickly. Following good coding standard and git practise will eventually save time and prevent accumulating technical debt.

Results can be improved by having better data, better features, better model architecture, better regularization (less overfitting). Besides collecting initial training data, sampling and shuffling the data also play a role in obtaining good model. The initial architecture usually comes from literature, and it servers as the baseline performance. Adding new elements that are proven to work in some other field into the baseline model is a good way to improve the performance. For example, ensemble learning and position information are shown to improve performances in areas like image classification and language translation, so modifying and applying them into a bioinformatics problem is also worth trying.

Regularization techniques such as dropout, L1/L2 are critical and effective in reducing model variance. It convenient to parameterize the regularization option in the development code for almost most layers, so we can try many of them quickly.
\section{Future Research}
For PPI sites prediction, many recent published deep learning techniques could potentially further improve the prediction performance. These include better protein sequence embedding using ELMo, the transformer architecture, graph neural network, residual architecture, more sophisticated data sampling. Benchmark data can be improved as well. Three of the testing dataset used in DELPHI are relatively old and new publications are still using them because no better ones exist. The ideas in DELPHI can be also transferred to protein binding sites prediction with other molecules such as DNA, RNA and ligand. Binding-partner specific prediction is also interesting. That is, not only the binding residues are predicted, also the biding proteins. Better features can be potential discovered. Section \ref{sec_arch_fea} shows that the newly added three features are more helpful than the architecture. 

For PPI prediction, our research lab is trying similar deep learning ideas to improve the prediction. The deep learning computational complexity of PPI prediction is higher than site precision. Quantization techniques and mixed precision could be used to accelerate the process. Classifying more specific area of PPI site could also be interesting. For example,  permanent and transient can be predicted separately \cite{perkins2010transient}. 


%% This adds a line for the Bibliography in the Table of Contents.
\addcontentsline{toc}{chapter}{Bibliography}
%% ***   Set the bibliography style.   ***
\bibliographystyle{plain} % (change according to your preference)
%%% ***   Set the bibliography file.   ***
\bibliography{reference}{}
%% ***   NOTE   ***
%% If you don't use bibliography files, comment out the previous line
%% and use \begin{thebibliography}...\end{thebibliography}.  (In that
%% case, you should probably put the bibliography in a separate file
%% and \include or \input it here).

%Appendices.
% \begin{appendices}
% \include{appendixa}
% \end{appendices}

%CV only relevant stuff... not full CV.
\addcontentsline{toc}{chapter}{Curriculum Vitae}
\chapter*{Curriculum Vitae}
\begin{table}[ht]
\begin{tabular}{ll}
\textbf{Name:} & \firstname{} \lastname\\\\
\textbf{Post-Secondary} & University of Western Ontario\\
\textbf{Education and}& London, ON, Canada\\
\textbf{Degrees:}& 2014 - 2020 Ph.D.\\\\
& University of Western Ontario\\
& London, ON, Canada\\
& 2012 - 2013 M.Sc.\\\\
& Anhui University\\
& Hefei, Anhui, China\\
& 2008 - 2012 B.Eng.\\\\
\textbf{Honours and}& Robert and Ruth Lumsden Graduate Awards\\
\textbf{Awards:}& March 2018\\\\
& Second Prize at University of Western Ontario Research in Computer Science\\
& April 2018\\\\
& First Prize at University of Western Ontario Research in Computer Science\\
& April 2017\\\\
& Society OF Graduate Student Travel Scholarship\\
& July 2017\\\\
\textbf{Related Work}& Senior Software Engineer\\
\textbf{Experience:}& Huawei Toronto Research Center\\
& 2018 - Current\\\\
& Research and Teaching assistant\\
& The University of Western Ontario\\
& 2012 - 2017\\
\end{tabular}
\end{table}
\subsubsection*{Publications:}
Li, Y. and Ilie, L., 2020. DELPHI: accurate deep ensemble model for protein interaction sites prediction, \\\\
Li, Y. and Ilie, L., 2019. Predicting Protein–Protein Interactions Using SPRINT. Protein-Protein Interaction Networks. Humana, New York, NY, 2020. 1-11.\\\\
Li, Y. and Ilie, L., 2017. SPRINT: ultrafast protein-protein interaction prediction of the entire human interactome. BMC bioinformatics 18.1 (2017): 485.

\subsubsection*{Referred Conference Presentations:}
DELPHI: accurate deep ensemble model for protein interaction sites prediction
(ISMB’20), 3DSIG, Montreal, 2020. (abstract and poster)\\\\
Y. Li, L. Ilie, SPRINT: Ultrafast protein-protein interaction prediction of the entire human interactome, 25th Annual International Conference on Intelligent Systems for Molecular Biology
(ISMB’17), 3DSIG, Prague, 2017. (poster)


\end{spacing}
\end{document}

